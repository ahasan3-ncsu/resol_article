\documentclass[12pt]{article}
\usepackage[margin=1in]{geometry}
\usepackage{graphicx, subcaption, booktabs}
\usepackage{placeins, xspace, hyperref}
\usepackage{amsmath, amssymb, siunitx}
\usepackage[version=4]{mhchem}

\newcommand{\Y}{\ce{_{39}^{97}Y}\xspace}
\newcommand{\I}{\ce{_{53}^{136}I}\xspace}

\title{Xe gas bubble re-solution in U-10Mo nuclear fuel}
\author{ATM Jahid Hasan}

\begin{document}

\maketitle

\begin{abstract}
	The U.S. High-Performance Research Reactor (USHPRR) program
	aims to convert high enriched fuel in high-power research reactors
	to low enriched fuel.
	The choice for this conversion is a fuel type
	based on the U-10Mo alloy.
	The behavior of fission product gases, such as Xe, in the fuel
	needs to be well understood
	for evaluating the performance of the U-10Mo fuel.
	Xe gas bubbles play a pivotal role in this fuel type
	by trapping more fission product gas atoms and growing in size.
	Also, parts of the bubbles can disintegrate under irradiation
	by a process called re-solution.
	The interplay between the trapping rate and the re-solution rate
	governs the evolution of these gas bubbles.
	In this study,
	binary collision approximation (BCA) and molecular dynamics (MD) simulations
	are performed to quantify the re-solution rate of Xe gas bubbles
	in U-10Mo fuel.
	First, the energy loss of fission fragments (FFs)
	due to electronic and nuclear stopping is evaluated.
	Electronic stopping can initiate thermal spikes along FF tracks.
	To quantify re-solution due to thermal spikes,
	MD simulations coupled with the two-temperature model (TTM) are performed.
	It is found that thermal spikes cannot bring about re-solution in U-10Mo.
	For re-solution due to nuclear stopping,
	FF behavior in U-10Mo is simulated using BCA
	to get average FF incidence probability, energy and angle
	as a function of distance from the FF origin.
	Afterward, interactions between
	FFs of different energies and Xe gas bubbles in U-10Mo
	are assessed.
	Based on all the information gathered from BCA simulations,
	an overall Xe gas bubble re-solution rate $b$ is computed.
	$b/\dot{F}$
	ranges from \num{4.4e-26} m$^3$/fission for intergranular bubbles
	to \num{8.8e-25} m$^3$/fission for intragranular bubbles,
	where $\dot{F}$ represents the fission rate in the fuel.
	The effect of bubble pressure on re-solution rate is also evaluated,
	revealing an inverse relation between the two.
\end{abstract}


\section{Introduction}

A U-10Mo alloy-based monolithic fuel design was identified
as the fuel type for converting U.S. High-Performance Research Reactors (HPRRs)
\cite{meyer2014} from high enriched fuel to low enriched fuel.
To understand the fuel's behavior under irradiation,
mesoscale and engineering-level fuel performance models require
knowledge of the fundamental mechanistic behavior of fission products
within the fuel to describe key phenomena,
such as swelling \cite{beeler2018gb, annualreport2021}.
Specifically, understanding the progression of Xe gas bubbles in the fuel
is crucial for optimizing reactor performance and safety.
These Xe gas bubbles act as a sink for individual Xe atoms,
trapping them and causing the bubbles to grow after absorption.
Under irradiation, the Xe atoms in the gas bubble are reintroduced
into the fuel matrix through fission-product-induced cascades
and thermal spikes---a process known as re-solution.
The relative rates of the re-solution affect the overall size and density
of the bubbles \cite{ye2023, olander2006re, parfitt2008},
in turn impacting bubble evolution and subsequent fuel swelling.
Re-solution of fission gas in nuclear fuels involves
two commonly accepted mechanisms:
homogeneous re-solution and heterogeneous re-solution \cite{olander2006re}.
In the homogeneous model proposed by Nelson \cite{nelson1968},
atoms from the gas bubbles are ejected individually
through collisions with fission products
or the recoil atoms that traverse the bubbles.
These atomic collision cascades are primarily
governed by the nuclear stopping power of the material.
In the heterogeneous model proposed by Turnbull \cite{turnbull1971},
a portion of gas bubbles are dissolved
by a passing fission fragment (FF) in the vicinity.
The driving mechanism is
the local heating of the material containing the gas bubbles,
through the electronic stopping of the FFs \cite{setyawan2018}.
The fact that both these mechanisms occur on a short timescale
makes it challenging to conduct experiments for determining re-solution rates
that contribute to the fission gas release models.
Thus, atomistic-scale modeling is necessary
for determining the re-solution rate
and for elucidating the fundamental mechanism
behind re-solution in U-10Mo.

In the literature up to this point,
atomistic simulations have been widely used
to evaluate the re-solution rate in various nuclear materials.
For instance, in 2008, Parfitt et al. \cite{parfitt2008}
used simulations of primary knock-on atoms (PKAs) in uranium dioxide (UO$_2$)
to assess the re-solution of helium gas bubbles.
In 2009, Schwen et al. \cite{schwen2009md} investigated
the homogeneous re-solution of Xe gas bubbles in UO$_2$,
using binary collision approximation (BCA) and molecular dynamics (MD).
The following year, Huang et al. \cite{huang2010md}
examined the impact of thermal spikes on Xe re-solution in UO$_2$.
In 2012, Govers et al. \cite{govers2012}
performed PKA and thermal spike simulations of Xe gas bubbles in UO$_2$
and proposed a mathematical model for the re-solution rate.
However, the most comprehensive work on Xe gas bubble re-solution in UO$_2$
was conducted  in 2018 by Setyawan et al. \cite{setyawan2018}.
They reconciled the inconsistencies found in the conclusions of previous works
on Xe bubble re-solution in UO$_2$
and evaluated the re-solution rate as a function of bubble radius.
Their findings suggest that heterogeneous re-solution of gas bubbles
is the dominant method of re-solution in UO$_2$.
In addition to UO$_2$,
the re-solution rate of fission gas bubbles was also evaluated
in uranium carbide (UC) by Matthews et al. \cite{matthews2015} in 2015
and in uranium zirconium (U-Zr) alloys by Mao et al. \cite{mao2025} in 2025,
using BCA.
Thermal spikes are not expected to occur in UC and U-Zr systems
due to higher electronic conductivity and thermal diffusivity
compared to UO$_2$ \cite{matthews2015, ronchi1986, mao2025}.
Thus, only heterogeneous re-solution has been studied in UC and U-Zr.
In summary, BCA and MD simulations were both used
to determine the re-solution rate in nuclear fuels.

In MD simulations of homogeneous re-solution,
a regular lattice atom is typically endowed with high kinetic energy
to emulate a PKA.
The PKA then interacts ballistically with other atoms,
initiating a collision cascade near the gas bubble
and inducing disorder \cite{parfitt2008, govers2012}.
One alternative approach in MD is to simulate
only a portion of the cascade (i.e., a subcascade)
by imparting energy to a random gas atom within the bubble.
In doing so, the simulation avoids unnecessary cascade events
that may not significantly influence the re-solution process.
However, BCA must be utilized in this approach
to first obtain an energy spectrum of the gas atom PKAs \cite{schwen2009md}.
One challenge in using MD simulations to model homogeneous re-solution
is the channeling of PKAs or their recoils over long distances,
without any collisions \cite{jarrin2021}.
This can make collecting statistics
on the interactions between PKAs and gas bubble atoms computationally demanding,
especially when the PKA direction is random.
A potential solution is to direct the PKAs
toward a high index lattice direction \cite{stoller2000}.
Moreover, collision cascades end up in heat spikes due to nuclear stopping.
This can lead to damage assisted re-solution
due to the high concentration of defects caused by these cascades.
To identify atoms that are re-solved ballistically, a threshold atomic speed,
above which it is improbable to find atoms in thermal equilibrium,
can be utilized \cite{parfitt2008}.
For MD simulations of heterogeneous re-solution due to swift heavy ions,
the thermal spike model is normally employed.
This model is useful for describing
the interaction between the FFs and the fuel.
These interactions occur primarily via
electronic stopping of the energetic particles
that initially raise the electronic subsystem temperature.
The energy deposited in the electronic subsystem can then transfer
to the lattice as thermal energy via electron-phonon coupling.
Finally, the energy is transferred among the atoms,
leading to a rapid increase in lattice temperature
within a cylindrical zone of typically a few nm in radius.
This increase is known as a thermal spike
\cite{wang1994, toulemonde2002, patra2019}.
In MD simulations, electronic interactions cannot be treated directly.
However, the thermal spike process can be emulated
either by raising the temperature of atoms within a cylindrical region
\cite{govers2012, setyawan2018}
or by coupling the two-temperature model (TTM) with MD
\cite{duffy2006, huang2010md}.

For qualification of U-10Mo fuel,
the ability to accurately predict the fission gas atom evolution
under various operational and transient conditions is crucial.
To that end, the Dispersion Analysis Research Tool (DART),
a mesoscale code developed by Argonne National Laboratory \cite{ye2023},
has been equipped with the ability to calculate
fission gas swelling in U-10Mo under different operational situations.
One of the many parameters required to model the swelling behavior
is the re-solution rate of fission gas bubbles.
DART employs a re-solution model
that includes a piecewise function to account for
both intergranular and intragranular bubbles.
The parameters in this function are calibrated
by fitting the computed swelling value to experimental data,
meaning that we only have a rough estimation of the re-solution rate.
A physics-based model of the re-solution rate of fission gas bubbles would make
the swelling calculations of higher-length-scale models more rigorous.
Thus, in the present study, we utilize BCA and MD simulations
to investigate the re-solution of Xe gas bubbles in U-10Mo fuel,
considering both the homogeneous and heterogeneous re-solution mechanisms.


\section{Computational methods}

In this section, we discuss the general computational methods.
Specific simulation details will be discussed along with the results
for better readability.

\subsection{Binary collision approximation}

RustBCA, a free and open-source software,
has been used for all BCA simulations \cite{drobny2021}.
This software can simulate ion-material interactions
including sputtering, implantation, and reflection.
Out of the box, RustBCA supports infinite homogeneous 0D targets,
finite-depth layered inhomogeneous 1D targets,
inhomogeneous 2D targets through a triangular mesh,
and homogeneous 3D triangular mesh geometry.
However, it does not support inhomogeneous 3D geometry,
which is necessary to emulate gas bubbles embedded in solids.
For that reason, we implemented a specific 3D geometry in RustBCA
called \texttt{SPHEREINCUBOID}
that allows the user to specify a spherical material inside another cuboid one.
The implementation is available at \url{https://github.com/ATM-Jahid/RustBCA}.
Like most BCA codes, RustBCA assumes an amorphous, static material,
neglecting crystal structures and accumulation of irradiation damage.
It also cannot take into account temperature effects.

The electronic stopping in the BCA simulations performed in this work
is described by the Biersack-Varelas interpolation \cite{varelas1970},
and the nuclear interactions are described by the universal Kr-C potential
\cite{moller1984, eckstein2013}.
An exponentially distributed mean-free-path model is used for gaseous regions
and a constant mean-free-path model for others.
A threshold number density of $\num{1.5e28}$ m$^{-3}$ is used to distinguish
between gaseous and solid regions.

\subsection{Molecular dynamics}

The LAMMPS software package \cite{lammps} was utilized for the MD work,
in conjunction with a U-Mo-Xe angular-dependent potential (ADP)
\cite{smirnova2013, starikov2018, beelerADP}.
This ADP can accurately describe
the body-centered cubic phase of $\gamma$U-Mo alloys,
reproducing their stable structure, modulus of elasticity,
room temperature density, and melting point.
To describe the electronic stopping due to FFs,
all MD simulations were coupled with the TTM
using the ttm/mod command in LAMMPS \cite{norman2013, pisarev2014}.
In this approach, the electronic subsystem is considered as a continuum
while the ionic subsystem is described by standard MD.
Energy transfer within the electronic subsystem
is handled using the heat diffusion equation along with source terms
representing the heat transfer between the two subsystems:
\begin{align}
	C_e (T_e) \rho_e \frac{\partial T_e}{\partial t}
		&= \nabla (\kappa_e \nabla T_e) + g_p (T_e - T_a)
\end{align}
where $C_e$ is the specific heat as a function of $T_e$,
$\rho_e$ is the electronic density,
$\kappa_e$ is the thermal conductivity, $T$ is the temperature,
$g_p$ is the coupling constant for electron-ion iteraction,
and the `$e$' and `$a$' subscripts represent
electronic and atomic subsystems, respectively
\cite{duffy2006, rutherford2007}.
The electronic specific heat is used in the form $C_e = \gamma T_e$,
where $\gamma = \num{4e-9}$ eV/(K$^2$e).
The electronic density is set to $\rho_e = 625$ e/nm$^3$,
and the thermal conductivity is equated to $\kappa_e = D_e \rho_e C_e$,
where $D_e = 100$ nm$^2$/ps is the thermal diffusion coefficient
\cite{li2017, kolotova2017}.
Even though these TTM parameter values have only been used
for pure U or U-5at.\%Mo,
our simulations showed they provide
sufficiently accurate thermal conductivity values in U-10Mo.

Electronic pressure effects are included in the model
to account for the blast force acting on ions
due to the electronic pressure gradient \cite{chen2006, norman2013}.
The total force acting on an ion is:
\begin{align}
	\vec{F}_i
		&= -\frac{\partial U}{\partial \vec{r}_i}
		+ \vec{F}_{langevin} - \nabla P_e / n_{ion}
\end{align}
where $\vec{F}_{langevin}$ is a force from Langevin thermostat
simulation electron-phonon coupling,
$U$ is the potential energy of the system,
$\nabla P_e / n_{ion}$ is electron blast force,
and $n_{ion}$ is the ion concentration.
The electronic pressure is used in the form $P_e = 0.5 \rho_e C_e T_e$
\cite{norman2013, pisarev2014, kolotova2017}.


\section{Energy loss of fission fragments in U-10Mo}

To understand and quantify re-solution in U-Mo,
the behavior of FFs need to be studied first.
The fission of \ce{_{92}^{235}U} can produce many different isotopes.
To keep computational complexity manageable,
two isotopes, \Y and \I, are chosen as representative of light and heavy FFs.
These two isotopes are created in the fuel due to the following reaction:
\begin{align}
	\ce{
		_0^1n + _{92}^{235}U -> _{39}^{97}Y + _{53}^{136}I + Q
	}
	\label{eq:iso}
\end{align}
While \Y has an initial kinetic energy of about $101.3$ MeV,
\I has about $74.6$ MeV.
The rationale behind choosing these two as representative FFs is that
they appear close to the two peaks in a typical fission product yield graph,
and these isotopes have a fission product yield of about $0.12$
\cite{setyawan2018, mills1995}.

\begin{figure}[!ht]
	\begin{subfigure}{0.49\textwidth}
		\centering
		\caption{}
		\includegraphics[width=8cm]{images/Y_stopping.pdf}
	\end{subfigure}
	\begin{subfigure}{0.49\textwidth}
		\centering
		\caption{}
		\includegraphics[width=8cm]{images/I_stopping.pdf}
	\end{subfigure}
	\caption{
		Nuclear and electronic stopping power of (a) \Y and (b) \I.
	}
	\label{fig:stopping}
\end{figure}

To evaluate the energy loss of these FFs in U-10Mo,
BCA simulations are set up using RustBCA's \texttt{0D} geometry option.
In these simulations, U-10Mo extends from $0$ to $\infty$ in the $x$ direction,
and $-\infty$ to $\infty $ in $y$ and $z$ directions.
The FFs are created at $(0, 0, 0)$ with a direction of $(1, 0, 0)$.
For each FF, $2,000$ independent ion-material simulations are performed.
FF positions and velocities from every single binary collision
are processed and binned together
to obtain electronic and nuclear stopping profiles.
These profiles are shown in Figure \ref{fig:stopping}.
At the beginning, the electronic stopping power
is bounded by $20$ keV/nm for both FFs.
As expected, the nuclear stopping power
has a characteristic Bragg peak towards the end of FF trajectory (cite).
The nuclear energy loss of \Y is about $5\%$ of its total initial energy,
whereas the nuclear energy loss of \I is about $10\%$ of the total.

Based on the information from the stopping power profiles,
a two-pronged approach is employed to assess the gas bubble re-solution rate.
To analyze the effect of electronic stopping,
the two-temperature model (TTM) is utilized to emulate the transfer of energy
from the electronic subsystem to the ionic subsystem,
and this is discussed in section \ref{sec:elec}.
For nuclear stopping,
the interactions among FFs, U-10Mo, and Xe gas bubbles are simulated
using the newly implemented \texttt{SPHEREINCUBOID} geometry in RustBCA,
and this is discussed in detail in section \ref{sec:nuke}.


\section{Re-solution due to electronic stopping}
\label{sec:elec}

To simulate thermal spikes, a simulation cell of size
$120 \alpha_0 \times 120 \alpha_0 \times 50 \alpha_0$
($\alpha_0 = 0.343$ nm is the lattice parameter of U-10Mo)
is created with periodic boundary condition in all directions.
A random distribution of U and Mo atoms in a bcc lattice
is first generated such that the Mo concentration equals $22$ at.\%.
Utilizing the resulting configuration, a spherical bubble of radius $2$ nm
is created by removing U and Mo atoms
and depositing gaseous Xe atoms at a Xe/vacancy ratio of $0.2$
(about $330$ Xe atoms) within the resulting void.
The system is then equilibrated at $400$ K
at a pressure of $0$ bar in an NPT ensemble for $10$ ps
with a timestep size of $1$ fs,
using the Nos\'e-Hoover barostat and thermostat.

Electronic cells of size
$2 \alpha_0 \times 2 \alpha_0 \times 50 \alpha_0$
are specified across whole the simulation cell.
The electronic subsystem is also periodic across all boundaries.
A thermal spike is then emulated
by initiating temperature values for these electronic cells.
The electronic temperature profile is set according to:
\begin{align}
	T_e &= T_{init} + T_{spike} \exp\left( -r / R \right)
\end{align}
where $r$ is the distance from the thermal spike axis,
$R$ is equal $10 \alpha_0$, and $T_{init} = 400$ K.
The thermal spike axis is aligned with the shortest dimension of the supercell,
and $T_{spike}$ and $R$ are chosen according to
the desired electronic stopping power.
The simulations are then performed using an NVE ensemble
with a canonical sampling thermostat \cite{bussi2007}
at the edges of the simulation box to act as a heat sink.
Only the edges that do not intersect the thermal spike axis
are part of the heat sink.
A variable timestep size is implemented for the NVE runs,
such that any atom's maximum displacement
between two successive timesteps is less than or equal to $0.001$ nm.
$100,000$ timesteps are run for each simulation,
yielding \qtyrange{80}{90}{ps} total simulation time.

\begin{figure}[!ht]
	\centering
	\begin{subfigure}{0.49\textwidth}
		\centering
		\caption{}
		\includegraphics[width=8cm]{images/ttm1.png}
	\end{subfigure}
	\begin{subfigure}{0.49\textwidth}
		\centering
		\caption{}
		\includegraphics[width=8cm]{images/ttm2.png}
	\end{subfigure}
	\begin{subfigure}{0.49\textwidth}
		\centering
		\caption{}
		\includegraphics[width=8cm]{images/ttm3.png}
	\end{subfigure}
	\begin{subfigure}{0.49\textwidth}
		\centering
		\caption{}
		\includegraphics[width=8cm]{images/ttm4.png}
	\end{subfigure}
	\caption{
		Snapshots of a thermal spike (30 keV/nm) simulation
		at (a) 0, (b) 2.4, (c) 10.1, and (d) 29.1 ps.
		The Xe atoms are shown in black, along with U (red) and Mo (blue) atoms.
	}
	\label{fig:ttm}
\end{figure}

Simulations are performed with three values of $T_{spike}$
($28,000$ K, $31,500$ K, and $34,500$ K),
emulating electronic stopping power ranging from $20$ keV/nm to $30$ keV/nm.
No re-solution has been observed in any of these simulations.
Figure \ref{fig:ttm} shows a few snapshots
of a $30$ keV/nm thermal spike simulation as visualized in OVITO \cite{ovito}.
The local ionic subsystem temperature rises for about $1.5$ ps
after the initiation of the thermal spike before starting to cool down.
After $30$ ps, only a few defects are observable in the system.
It takes about $60$ ps for the local temperature to fall down
to the level of $T_{init}$.

Kolotova et al. reported threshold electronic stopping power
for defect formation and melting at various temperatures \cite{kolotova2017}.
At $400$ K, the threshold stopping powers for defect formation and melting
are about $22$ keV/nm and $26$ keV/nm in U-5at.\%Mo, respectively.
Given that we do not observe any re-solution
in the MD simulations of $30$ keV/nm thermal spikes,
peak FF stopping power is about $20$ keV/nm (Figure \ref{fig:stopping}),
and threshold stopping power for defect formation is above $20$ keV/nm,
it is highly unlikely any gas bubble re-solution can occur in U-10Mo
through the heterogeneous mechanism.
Since U-10Mo is a metallic system, this behavior is expected.


\section{Re-solution due to nuclear stopping}
\label{sec:nuke}

\subsection{Model for re-solution calculation}

\begin{figure}[!ht]
	\begin{subfigure}{\textwidth}
		\centering
		\caption{}
		\includegraphics[width=12cm]{images/rotation.pdf}
	\end{subfigure}
	\begin{subfigure}{\textwidth}
		\centering
		\caption{}
		\includegraphics[width=8cm]{images/coord.pdf}
	\end{subfigure}
	\caption{
		(a) Rotation of fission event positions and velocities around the origin
		such that all velocities point to the $-x$ direction.
		Red dots represent positions and gray arrows represent velocities.
		(b) A coordinate system where a Xe gas bubble is at the origin,
		and fission fragments are all pointing toward the $-x$ direction.
	}
	\label{fig:coord}
\end{figure}

The re-solution rate can be defined as the probability
of a Xe atom going out of the gas bubble and into the fuel matrix,
per unit time.
To represent the overall re-solution behavior in the material,
the contributions from all the FFs,
originating at different distances from a bubble
and oriented toward a random direction,
need to be summed up by means of a volume integral.
Consider a 3D coordinate system in which the gas bubble is at the origin.
For ease of calculation,
FF origins can be rotated around the coordinate system origin
so that all FFs point in the same direction initially (Figure \ref{fig:coord}a).
Given that the FF generation is uniform and isotropic in the material,
the aforementioned rotational transformation will lead to
a uniform distribution of unidirectional fission products
as shown in Figure \ref{fig:coord}b.
The axial coordinate $x$ is then defined as parallel to the FF velocities,
and the radial coordinate $w$ as perpendicular to them
(Figure \ref{fig:coord}b).
If we denote the fission rate as $\dot{F}$,
the number of fission events per second
in an infinitesimal volume $dV = 2 \pi w \: dw  \: dx$ would be $\dot{F} dV$.
Also, we assume that when a FF isotope $k$ originating at $(x, w)$
interacts with a Xe gas bubble at the origin,
it re-solves $\xi_k \equiv \xi_k(x, w)$ fraction of the Xe atoms.
All $k$ isotopes from fission events in $dV$ thus contributes
$\xi_k(x, w) \dot{F} dV$ to the total re-solution of the bubble.
The re-solution rate $b$ can then be expressed as:
\begin{align}
	b &= \sum_{k = Y, I} \int_V \xi_k(x, w) \dot{F} dV \label{eq:b} \\
	  &= \dot{F} \sum_{k = Y, I} \int_V \xi_k(x, w) dV
	   = \dot{F} \left( \int_V \xi_Y(x, w) dV + \int_V \xi_I(x, w) dV \right) \\
	  &= \dot{F} \sum_{k = Y, I} \int_{x=0}^{\infty} \int_{w=0}^{\infty}
		\xi_k(x, w) 2 \pi w dw dx
\end{align}
where it is assumed that all fission events produce \Y and \I
according to Eq. \ref{eq:iso}.

\subsection{Reference fission fragment simulations}

The most straightforward way to calculate $\xi_k(x, w)$ would be
simulating a FF and a Xe gas bubble for that specific $(x, w)$ value.
However, doing this for all necessary $(x, w)$ values
is computationally expensive and leads to statistically unreliable results.
As the distance between the origin of a FF and a Xe gas bubble increases,
the probability of any interaction between them diminishes greatly.
The interaction probability is also dependent of bubble size.
The smaller the bubble, the lower the probability.
Therefore, for some bubble sizes and $(x, w)$ values,
a few hundred thousand BCA simulations might only lead to few interactions,
making most of the simulations a poor use of computational resources.

\begin{figure}[!ht]
	\centering
	\begin{subfigure}{0.49\textwidth}
		\centering
		\caption{}
		\includegraphics[height=5cm]{images/ff_track.png}
	\end{subfigure}
	\begin{subfigure}{0.49\textwidth}
		\centering
		\caption{}
		\includegraphics[height=5cm]{images/surf_grid.pdf}
	\end{subfigure}
	\caption{
		(a) Trajectories of 100 simulated \Y ions in U-10Mo.
		(b) Surface discretization of fission fragment information across volume.
	}
	\label{fig:fftrack}
\end{figure}

\begin{figure}[!ht]
	\centering
	\begin{subfigure}{0.49\textwidth}
		\centering
		\caption{}
		\includegraphics
			[width=8cm, trim={0.8cm 0 1.5cm 0.4cm}, clip]
			{images/Y_p.pdf}
	\end{subfigure}
	\begin{subfigure}{0.49\textwidth}
		\centering
		\caption{}
		\includegraphics
			[width=8cm, trim={0.8cm 0 1.5cm 0.4cm}, clip]
			{images/I_p.pdf}
	\end{subfigure}
	\begin{subfigure}{0.49\textwidth}
		\centering
		\caption{}
		\includegraphics
			[width=8cm, trim={0.8cm 0 1.5cm 0.4cm}, clip]
			{images/Y_e.pdf}
	\end{subfigure} \begin{subfigure}{0.49\textwidth}
		\centering
		\caption{}
		\includegraphics
			[width=8cm, trim={0.8cm 0 1.5cm 0.4cm}, clip]
			{images/I_e.pdf}
	\end{subfigure}
	\begin{subfigure}{0.49\textwidth}
		\centering
		\caption{}
		\includegraphics
			[width=8cm, trim={0.8cm 0 1.5cm 0.4cm}, clip]
			{images/Y_a.pdf}
	\end{subfigure}
	\begin{subfigure}{0.49\textwidth}
		\centering
		\caption{}
		\includegraphics
			[width=8cm, trim={0.8cm 0 1.5cm 0.4cm}, clip]
			{images/I_a.pdf}
	\end{subfigure}
	\caption{
		Ion incidence probability per unit surface area of (a) \Y and (b) \I.
		Average incidence energy of (c) \Y and (d) \I.
		Average incidence angle of (e) \Y and (f) \I.
	}
	\label{fig:ref}
\end{figure}

An alternative of the brute-force approach is to first assess
the behavior of FFs in the fuel without the presence of any bubbles.
If the probability of an FF
going through a certain point in the fuel with a certain energy is known,
obtaining re-solution behavior from local FF-bubble interactions
at that exact point would be easy.
Thus, FF simulations in U-10Mo are performed to obtain three key properties:
the probability of an FF passing through a particular point per unit area,
its average incidence energy and average incidence angle.
In these reference simulations, the FF starts from the origin
and is directed toward the $x$ axis in a U-10Mo matrix
as displayed in Figure \ref{fig:fftrack}a.
Notice that this setup is different
from the one described in Figure \ref{fig:coord}b,
where the Xe gas bubble was at the origin.

To gather the data from the simulations,
an annular surface discretization scheme,
as shown in Figure \ref{fig:fftrack}b, is used.
The discretized surface elements are associated with specific $(x, w)$ values,
and the distance between two successive surface elements
in $x$ or $w$ direction is set to $\Delta x = \Delta w = 50$ nm.
This discretization is sufficient enough
to provide smooth profiles of FF properties.
To check the convergence of these FF profiles,
we selected $6$ points for both \Y and \I.
$1,000$ ion simulations were performed at a time,
When the relative changes in probability, energy, and angle
between two consecutive batches of simulations reached below $0.001$
at each of the 6 points, simulations were stopped.
The $6$ points used for convergence were
$(3, 0)$, $(5, 0)$, $(7, 0)$,
$(4.5, 0.5)$, $(6.5, 0.5)$, and $(6, 1)$ for \Y,
and $(2, 0)$, $(3.5, 0)$, $(5, 0)$,
$(3, 0.5)$, $(4.5, 0.5)$, and $(4, 1)$ for \I.
All the coordinates are in $\mu$m.
\Y profiles converged after $30,000$ simulations and \I profiles after $40,000$.

The results from the simulations and subsequent discretization
are shown in Figure \ref{fig:ref}.
The incidence probability per unit area
is displayed in Figures \ref{fig:ref}a and \ref{fig:ref}b.
The probability profiles widen as $x$ increases, creating a plume like pattern.
Figures \ref{fig:ref}c and \ref{fig:ref}d show the incidence ion energies.
The circular pattern clearly depicts how the ions lose energy
the farther they move from the origin.
The energy loss as a function of distance is mostly linear.
Finally, Figures \ref{fig:ref}e and \ref{fig:ref}f display
the incidence angle of ions with respect to the $x$ axis.
As expected, ions closer to $w = 0$ have a low incidence angle,
while ions farther away have higher incidence angles.
In all the incidence energy and the incidence angle plots,
a few discrete FF paths are also visible in the top-left region.
These are highly unlikely occurrences as evident from the probability figures.
The ion profiles also show the ranges of \Y and \I ions in U-10Mo
as approximately $8.5$ $\mu$m and $6.5$ $\mu$m, respectively.
The observed ranges in Figure \ref{fig:ref} are consistent
with the data in Figure \ref{fig:stopping}.

\subsection{Fission fragment interactions with Xe gas bubbles}

\begin{figure}[!ht]
	\centering
	\includegraphics[width=8cm]{images/n_vdw.pdf}
	\caption{
		Equilibrium Xe number density in gas bubbles
		calculated from the van der Waals equation of state.
	}
	\label{fig:vdw}
\end{figure}

With the reference FF properties available,
local FF-bubble interactions now need to be simulated
to quantify the gas bubble re-solution rate.
For these simulations, equilibrium Xe number densities are used
for bubbles of all sizes.
The van der Waals equation of state (EOS),
as described by Olander in \cite{olander1975},
is used to compute the equilibrium number densities:
\begin{align}
	n &= \left( B + \frac{kT}{p} \right)^{-1} \\
	n_{eq} &= \left( B + \frac{kT}{p_{eq}} \right)^{-1}
	        = \left( B + \frac{kT R_b}{2 \gamma} \right)^{-1}
\end{align}
The equilibrium bubble pressure $p_{eq}$ utilized in that EOS
comes from the Young-Laplace equation:
\begin{align}
	p_{eq} &= \frac{2 \gamma}{R_b}
\end{align}
where $\gamma = 1.55$ J/m$^2$ is the surface energy in U-10Mo \cite{beelerADP}.
The resultant equilibrium number densities
are plotted against bubble radius in Figure \ref{fig:vdw}.
Notice that the van der Waals EOS predicts
a plateauing equilibrium number density for smaller bubbles.

\begin{figure}[!ht]
	\centering
	\includegraphics[width=8cm]{images/recoil_dr.pdf}
	\caption{
		Recoil displacement against recoil energy
		in $5$ simulations of \Y in U-10Mo.
	}
	\label{fig:recoil}
\end{figure}

FF-bubble interactions should be simulated in a way such that
all pertinent recoils are accounted for.
To that end, $5$ BCA simulations of \Y in U-10Mo
are analyzed to assess recoil behavior.
Figure \ref{fig:recoil} shows a scatter plot
of the recoil displacements and their energies.
The largest amount of energy transferred to the recoils is about 1 MeV,
and the recoil displacement at that energy is at most about 100 nm.
It is possible to transfer more than 1 MeV to a recoil in a head-on collision,
but it is highly unlikely.
Therefore, it is reasonable to assume
the recoils generated by FFs more than $\delta = 100$ nm away from a bubble
cannot interact with that bubble.
Therefore, in the FF-bubble BCA simulations,
the FFs are placed $D = R_b + \delta$ distance away from the bubble center.
Recoil trajectories from one such simulation are visualized
in Figure \ref{fig:ffbub} using VisPy \cite{vispy}.

\begin{figure}[!ht]
	\centering
	\includegraphics[width=8cm]{images/ff_bubble.png}
	\caption{
		Recoil trajectories in a simulation of
		a 5 MeV \Y incident on a 64 nm radius Xe bubble.
		Red, blue, black, and cyan represent U, Mo, Xe, and \Y, respectively.
		Total Xe recoils: 3823.
		Xe outside the bubble: 155.
		Re-solved Xe atoms: 9.
	}
	\label{fig:ffbub}
\end{figure}

\begin{figure}[!ht]
	\begin{subfigure}{0.49\textwidth}
		\centering
		\caption{}
		\includegraphics[width=8cm]{images/xe_dr.pdf}
	\end{subfigure}
	\begin{subfigure}{0.49\textwidth}
		\centering
		\caption{}
		\includegraphics[width=8cm]{images/xe_hist.pdf}
	\end{subfigure}
	\caption{
		Xe recoils from the simulation of
		a 5 MeV \Y incident on a 64 nm radius Xe bubble.
		(a) Xe recoil final position against recoil energy.
		(b) Xe recoil final position histogram.
	}
	\label{fig:xeres}
\end{figure}

Now comes the problem of determining which Xe recoils are re-solved.
In this work, we consider Xe atoms that end up
at least $\lambda = 1$ nm away from the bubble surface as re-solved Xe atoms.
In reality, bubble surface is not a static structure
due to the thermal fluctuations of the Xe atoms that make up the surface.
Also, a Xe atom that is just outside the surface is highly likely
to end up in the bubble in a short timeframe.
Thus, there exists a finite annular region outside the bubble surface
that needs to be cleared by a Xe atom to be considered fully re-solved.
Our choice of $\lambda$ is also informed by the literature
\cite{schwen2009md, govers2012, setyawan2018}.
Figure \ref{fig:xeres} depicts the displacements of Xe recoils
from a FF-bubble simulation.
From a recoil displacement vs energy graph such as Figure \ref{fig:xeres}a,
it is also possible to find a minimum threshold energy $E_{min}$
required for re-solution.
$\lambda$ and $E_{min}$ are thus interconnected.
While some re-solution studies made a subjective choice on $\lambda$,
some did it on $E_{min}$ \cite{ronchi1986, matthews2015}.
From our simulations, we find $E_{min}$ to be about $25$ eV for $\lambda=1$ nm.
On the other hand, Figure \ref{fig:xeres}b depicts
how most Xe recoils originate close to the bubble surface
and end up just outside the surface.

\begin{figure}[!ht]
	\centering
	\begin{subfigure}{0.49\textwidth}
		\centering
		\caption{}
		\includegraphics[width=8cm]{images/chi_2nm_Y.pdf}
	\end{subfigure}
	\begin{subfigure}{0.49\textwidth}
		\centering
		\caption{}
		\includegraphics[width=8cm]{images/chi_2nm_I.pdf}
	\end{subfigure}
	\begin{subfigure}{0.49\textwidth}
		\centering
		\caption{}
		\includegraphics[width=8cm]{images/chi_64nm_Y.pdf}
	\end{subfigure}
	\begin{subfigure}{0.49\textwidth}
		\centering
		\caption{}
		\includegraphics[width=8cm]{images/chi_64nm_I.pdf}
	\end{subfigure}
	\caption{
		$\chi(E, \ell)$ for bubble radius of 2 nm
		with incident (a) \Y and (b) \I.
		$\chi(E, \ell)$ for bubble radius of 64 nm
		with incident (c) \Y and (d) \I.
	}
	\label{fig:chi}
\end{figure}

For FF-bubble simulations, both energy $E$ and off-centered distance $\ell$
are discretized.
Here, $\ell$ is measured as the minimum distance between
the bubble center and the initial FF velocity vector.
The energy discretization scheme depends on which FF is simulated,
and $\ell$ discretization depends on the bubble radius.
The $\ell$ values are chosen in such a way that
the region around $R_b$ is sampled properly.
$5,000$ BCA simulations are performed for each configuration.
From these simulations, the number of re-solved Xe atoms is first calculated.
This number is then divided by the initial number of Xe atoms in the bubble
to obtain the re-solved bubble fraction $\chi$.
In other words, if a Xe gas bubble is at the origin,
$\chi(E', \ell')$ is the re-solved bubble fraction
due to the interaction of the bubble with an FF that originated at $(D, \ell')$
with an energy $E'$ directed towards the $x$ axis.
The results for bubbles of two different sizes
are shown in Figure \ref{fig:chi}.
The $2$ nm and $64$ nm bubbles are representative
of intragranular and intergranular bubbles, respectively.
The error bars indicate $2\sigma$ deviations from the mean.

Using the simulation results and interpolation,
it is now possible to obtain any reasonable $\chi(E', \ell')$ value.
If $\mathcal{I}$ is an interpolator that uses the mapping $X \rightarrow Y$,
we can use the notation $y' = \mathcal{I}_X (x', X, Y)$ to mean that
the interpolator returns $y'$ when $x'$ is provided as an input.
Any arbitrary $\chi(E', \ell')$ is then defined the following way:
\begin{align}
	\chi(E', \ell)
	   &= \mathcal{I}_{\mathcal{E}}
	   (E', \mathcal{E}, [\chi(E, \ell)]_{E \in \mathcal{E}}) \\
	\chi(E', \ell')
	   &= \mathcal{I}_{\mathcal{L}}
	   (\ell', \mathcal{L}, [\chi(E', \ell)]_{\ell \in \mathcal{L}})
\end{align}
where $\mathcal{E}$ and $\mathcal{L}$ are the sets of discrete energies
and off-centered distances that are used in the simulations,
and $E$ and $\ell$ are elements of those sets.
In this work,
$\mathcal{I}_{\mathcal{E}}$ is a Pchip interpolator \cite{fritsch1984},
and $\mathcal{I}_{\mathcal{L}}$ is a linear interpolator.

\subsection{Calculation of \texorpdfstring{$\xi$}{xi}}

\begin{figure}[!ht]
	\centering
	\includegraphics[width=13cm]{images/surf_mesh.pdf}
	\caption{
		Illustration of a fission fragment going through a surface mesh element
		in the vicinity of a Xe gas bubble.
	}
	\label{fig:surf_mesh}
\end{figure}

Equipped with $\chi$ values, we proceed to calculate $\xi$.
Imagine an FF at the origin directed toward $x$, and a bubble at $(x, w)$.
At $(x, w)$, the FF has an average incidence angle $\alpha(x, w)$.
The incidence angle changes slowly with changes in $x$ and $w$,
and thus it is reasonable to assume the incidence angle
in the vicinity of $(x, w)$ is simply $\alpha(x, w)$.
Next, we find a point $(x', w')$ such that
it is $D$ distance away from $(x, w)$
and the line connecting these two points has a slope $\tan \alpha(x, w)$.
A surface $S$ perpendicular to that connecting line can then be constructed.
$S$ is square in shape and has a side length $2D = 2(R_b + \delta)$.
This surface is then meshed into small elements
as displayed in Figure \ref{fig:surf_mesh}.
Now, the FF has certain probabilities of going through each mesh element.
If the FF goes through a mesh element $m$
with probability per unit area $p(r_m)$ and incidence energy $E(r_m)$,
$\xi(x, w)$ can be calculated using the following equation:
\begin{align}
	\xi(x, w) &= \sum_{m \in S}
		p(r_m) \frac{A_m}{\cos \alpha(x, w)}
		\chi(E(r_m), ||r_m - r_c||)
	\label{eq:mesh}
\end{align}
where $r_m$ denotes the coordinate of the mesh element center,
and $r_c$ denotes the center of the surface $S$.
The total probability of the FF going through the mesh element $m$
is calculated by the product of $p(r_m)$
and the area of the mesh element $A_m$
projected to the direction perpendicular of $x$.
Notice that all FF trajectories that do not go through $S$ are ignored
because of the extremely low probability of either the FF or its recoils
reaching the bubble.

Even though the mesh elements displayed in Figure \ref{fig:surf_mesh}
are of constant size,
such a scheme is not efficient for the calculation of $\xi$.
This is due to the fact that $R_b$ can be a thousand times smaller than $D$.
Thus, we implemented an adaptive meshing scheme
that ensures a maximum mesh size of $R_b / 2$ for $\ell \in [0, 2 R_b]$
and $35$ nm for $\ell \in (2 R_b, D]$.
The upper bounds of the mesh size are chosen
in such a way that any finer resolution does not lead to
a relative change of $\xi$ by more than $0.001$.

\begin{figure}[!ht]
	\centering
	\begin{subfigure}{0.49\textwidth}
		\centering
		\caption{}
		\includegraphics
			[width=8cm, trim={0.8cm 0 1.5cm 0.7cm}, clip]
			{images/2nm_Y_xi.pdf}
	\end{subfigure}
	\begin{subfigure}{0.49\textwidth}
		\centering
		\caption{}
		\includegraphics
			[width=8cm, trim={0.8cm 0 1.4cm 0.7cm}, clip]
			{images/2nm_Y_db.pdf}
	\end{subfigure}
	\begin{subfigure}{0.49\textwidth}
		\centering
		\caption{}
		\includegraphics
			[width=8cm, trim={0.8cm 0 1.4cm 0.7cm}, clip]
			{images/64nm_Y_xi.pdf}
	\end{subfigure}
	\begin{subfigure}{0.49\textwidth}
		\centering
		\caption{}
		\includegraphics
			[width=8cm, trim={0.8cm 0 1.4cm 0.7cm}, clip]
			{images/64nm_Y_db.pdf}
	\end{subfigure}
	\caption{
		(a) $\xi$ and (b) $\xi \Delta V$
		for bubble radius of 2 nm with incident \Y.
		(c) $\xi$ and (d) $\xi \Delta V$
		for bubble radius of 64 nm with incident \Y.
	}
	\label{fig:xi}
\end{figure}

$\xi$ is calculated for all combinations of FF isotopes and bubble radii.
$\xi_Y$ for $2$ nm and $64$ nm bubbles
are shown in Figures \ref{fig:xi}a and \ref{fig:xi}b.
These $\xi$ profiles can also be interpreted in another way
that makes the calculation of the overall re-solution rate straightforward.
If a bubble is at the origin and a FF is at $(x, w)$ pointing toward $-x$,
the $\xi$ profiles we would get are exactly the same.
A discretized version of Eq. \ref{eq:b} can then be formulated as follows:
\begin{align}
	b / \dot{F}
		&= \sum_{k=Y,I} \sum \xi_k \Delta V \\
		&= \sum_{k=Y,I} \sum_{i,j \in V} \xi_k(x_{i,j}, w_{i,j})
			\left[\pi (w_{i,j+1}^2 - w_{i,j}^2) (x_{i+1,j} - x_{i,j})\right]
	\label{eq:xitob}
\end{align}
where $i, j$ denote the indices of the discrete grid points
where $\xi$ is evaluated.
Since the $\xi$ profiles are already discretized, this leads to an easy compute.
$\xi \Delta V$ profiles are also depicted in Figure \ref{fig:xi}
to show the significant effect of $\Delta V$ on re-solution.

\subsection{Homogeneous re-solution rate}

\begin{figure}[!ht]
	\centering
	\includegraphics[width=8cm]{images/bhom.pdf}
	\caption{
		Homogeneous re-solution rate as a function of bubble radius $R_b$
		at equilibrium Xe number density $n_{eq}$ in U-10Mo.
	}
	\label{fig:res}
\end{figure}

Summing up all the values in $\xi_Y \Delta V$ profiles
provides the re-solution rate of a bubble due to \Y.
The re-solution rate due to \I is also calculated similarly
and then used to get the total re-solution rate of that bubble.
Figure \ref{fig:res} displays the re-solution rates
for bubbles with radius ranging from $1$ nm to $128$ nm.
The error bars here describe the uncertainty in $b$
only due to the uncertainty in $\chi$.
Since the ion profiles and $xi$ have been computed
using stringent convergence criteria,
it is reasonable to assume almost all of the uncertainty propagates from $\chi$.
Thus, the error bars for $b$ also roughly denote $2\sigma$ deviations.
We also observe vanishingly small deviations for larger bubbles.
This is simply because the probability of an interaction
between a FF and larger bubbles is higher.

Even though simple linear interpolations are sufficient
to obtain re-solution rates for arbitrary bubble radii,
an approximate analytical function may better serve higher-length-scale models.
To that end, we propose the following functional form:
\begin{align}
	b / \dot{F} = a R_b^k + c
\end{align}
where $a = \num{8.43e-25}, k = -0.926$ and $c = \num{3.46e-26}$.
$R_b$ is in nm, and $b / \dot{F}$ is in m$^3$/fsn.
The functional fit has a root mean squared error of $\num{3.17e-27}$ m$^3$/fsn
and a $R^2$ score of $0.99986$.

\subsection{Effect of bubble pressure}

\begin{figure}[!ht]
	\centering
	\includegraphics[width=8cm]{images/pressure.pdf}
	\caption{
		Effect of Xe number density on homogeneous re-solution rate.
	}
	\label{fig:pres}
\end{figure}

The effect of pressure on re-solution is investigated
by varying the Xe number density of bubbles of radii $8$ nm and $64$ nm.
FF-bubble interaction simulations are performed
for \Y and \I ions of energies 1 MeV and 20 MeV.
$\chi$ values calculated from these simulations
are plotted in Figure \ref{fig:pres}.
An inversely proportional relationship between $\chi$ and $n$ is observed.
The equation $\chi / \chi_{eq} = n_{eq} / n$,
which is used to make the lines in Figure \ref{fig:pres},
fits the data with an R$^2$ score of 0.97.
To understand the origin of this inverse relationship,
the simulation data has been analyzed further.
It turns out the number of re-solved Xe atoms is almost an invariant
with respect to the initial Xe number density in the bubble.
Since $\chi$ is the ratio of the number of re-solved atoms
and the total number of Xe atoms in the bubble,
an inverse relation between $\chi$ and $n$ naturally arise.
One possible explanation for the invariance of the number of re-solved atoms
is that the FF-bubble interactions primarily affecting re-solution behavior
occurs close to the bubble surface,
and the gas bubble surface area is independent of $n$.

So far, the symbol $b$ has been used to denote the re-solution rate
at equilibrium Xe number density $n_{eq}$.
Here, $b_{eq}$ will be used to describe re-solution
at equilibrium Xe density,
whereas $b$ will be reserved for the general re-solution rate.
The re-solved bubble fraction $\chi$ is related to $\xi$
through equation \ref{eq:mesh},
and $\xi$ is related to $b$ through equation \ref{eq:xitob}.
These relations can thus be used
to relate the re-solution rate directly to the Xe number density:
\begin{align}
	b &\propto \xi \propto \chi \\
	b / b_{eq} &= n_{eq} / n \\
	b &= \left( a R_b^k + c \right) \left( \frac{n_{eq}}{n} \right) \dot{F}
	\label{eq:model}
\end{align}
where $R_b$ is in nm, $n$ is in m$^{-3}$,
$\dot{F}$ is in fsn m$^{-3}$ s$^{-1}$, and $b$ is in s$^{-1}$.
Equation \ref{eq:model} now describes the fission gas bubble re-solution rate
as a function of bubble size, bubble pressure, and fission rate.


\section{Discussion}

\begin{figure}[!ht]
	\centering
	\includegraphics[width=8cm]{images/comp.pdf}
	\caption{
		Comparison of the DART model prediction for re-solution
		with the results from this work.
	}
	\label{fig:comp}
\end{figure}

The re-solution rate computed in this work is compared
against the literature values for UO$_2$, UC, and U-10Mo
in Figure \ref{fig:comp}.
Setyawan et al. performed MD simulations of gas bubbles
ranging from $0.6$ nm in radius to $3$ nm \cite{setyawan2018}.
The slope of $\ln b$ against $\ln R_b$ in UO$_2$
is similar to that in our work,
albeit the re-solution rate in UO$_2$ is almost
one order of magnitude lower than in U-10Mo.
Matthews et al. conducted BCA simulations to evaluate the re-solution rate
for a wide range of bubble radii in UC \cite{matthews2015}.
The slope of $\ln b$ against $\ln R_b$ in UC is flatter than that in U-10Mo.
An intersection between the two rates is also observed,
showing the re-solution rate is higher in U-10Mo for smaller bubbles
but not for larger bubbles.
Even with all the differences, the re-solution rates in UO$_2$, UC, and U-10Mo
are very close to each other,
and the difference is often less than one order of magnitude.
We could not compare our results against U-Zr
because Mao et al. simulated all interactions at a fixed distance of $2$ $\mu$m
between the FFs and the bubbles \cite{mao2025}.
Thus, their data do not lead to an overall re-solution rate.

Finally, the re-solution rate evaluated in this work is compared
against the existing re-solution model of U-10Mo as implemented in DART.
The model is as follows:
\begin{align}
	b_{dart} &= b_0 \cdot \dot{F} \cdot G \\
	G &=
	\begin{cases}
		1 & ,R_{bubble} \leq \lambda \\
		1 - (\frac{R_{bubble}-R_{resol}}{R_{bubble}})^3
		  & ,R_{bubble} > \lambda
	\end{cases}
\end{align}
where $b_0$ is the bubble destruction probability,
and $G$ is a piecewise function representing
different re-solution modes for small and large gas bubbles.
In the piecewise function G, $R_{bubble}$ is the bubble radius,
$\lambda$ is the gas-atom knock out distance,
and $R_{resol}$ is the thickness of the annulus
within which all gas-atoms are knocked out.
The parameters $b_0$, $\lambda$, and $R_{resol}$ are considered adjustable,
and the optimized values are $b_0 = \num{2e-18}$ cm$^3$,
$\lambda = \num{5e-7}$ cm, and $R_{resol} = \num{3e-9}$ cm.
Since the parameters $\lambda$ and $R_{resol}$ are not coupled,
the re-solution rate is discontinuous at a bubble radius $\lambda$.
According to Ye et al. \cite{ye2023},
this is to account for the strong trapping effects of grain boundaries
on intergranular bubbles.

The re-solution rate computed in this work is slightly lower than
the DART model prediction for intragranular bubbles
but higher than the prediction for intergranular bubbles.
In fact, the difference for intergranular bubbles can be
as large as two orders of magnitude.
This large difference most likely stemmed from
the mixing of trapping effects to the re-solution rate.
However, accounting for trapping in the re-solution rate
contradicts the definition of re-solution.
To model gas bubble evolution, the re-solution rate and trapping rate
should be implemented separately in higher-length-scale programs,
thereby affording greater freedom in modeling complex behavior
of gas bubbles in the fuel under different contexts.
For example, the grain boundary diffusion coefficient of Xe in U-10Mo can be
15 orders of magnitude higher than the intrinsic diffusion coefficient
at around 600 K \cite{hasan2024gb}.
Thus, in a gas bubble evolution model,
the Xe trapping rate for intergranular bubbles can be set
15 times higher than that for intragranular bubbles,
and the re-solution rate can be set
using the analytical model prescribed in this work.
This way the re-solution rate will not be confounded with the trapping rate.


\section{Conclusion}

This study utilizes BCA and MD simulations
to determine the re-solution rate of Xe gas bubbles in U-10Mo nuclear fuel.
Both the homogeneous and heterogeneous re-solution mechanisms are investigated.
Thermal spikes initiated by electronic stopping cannot cause re-solution.
Thus, the occurrence of heterogeneous re-solution in U-10Mo is not probable.
Homogeneous re-solution, which is brought about by nuclear stopping,
is found to be the only mechanism of re-solution in U-10Mo.
Therefore, the re-solution rate is calculated
by first profiling FF behavior in the fuel,
then evaluating the FF interactions with Xe gas bubbles,
and finally putting all the information together in a physical model.
The computed re-solution rate $b$ is
$\num{4.4e-26} \dot{F}$ s$^{-1}$ for the largest intergranular bubble
and $\num{8.8e-25} \dot{F}$ s$^{-1}$ for the smallest intragranular bubble,
where the unit of $\dot{F}$ is fission/m$^3$/s.
Furthermore, BCA simulations with varying Xe number density in the bubble
revealed that
the re-solution rate is inversely proportional to the Xe number density.
Thus, higher bubble pressure leads to a lower re-solution rate.
The results of this study will inform higher-length-scale models of U-10Mo
with a physics-based description of the Xe gas bubble re-solution rate.


\bibliographystyle{unsrt}
\bibliography{ref.bib}

\end{document}

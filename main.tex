\documentclass[12pt]{article}
\usepackage[margin=1in]{geometry}
\usepackage{graphicx, subcaption, booktabs}
\usepackage{placeins, xspace, hyperref}
\usepackage{amsmath, amssymb, siunitx}
\usepackage[version=4]{mhchem}

\newcommand{\Y}{\ce{_{39}^{97}Y}\xspace}
\newcommand{\I}{\ce{_{53}^{136}I}\xspace}

\title{Xe gas bubble re-solution in U-10Mo nuclear fuel}
\author{ATM Jahid Hasan}

\begin{document}

\maketitle

\begin{abstract}
	The U.S. High-Performance Research Reactor (USHPRR) program
	aims to convert high enriched fuel in high-power research reactors
	to low enriched fuel.
	The choice for this conversion is a fuel type
	based on the U-10Mo alloy.
	The behavior of fission product gases, such as Xe, in the fuel
	needs to be well understood
	for evaluating the performance of the U-10Mo fuel.
	Xe gas bubbles play a pivotal role in this fuel type
	by trapping more fission product gas atoms and growing in size.
	Also, parts of the bubbles can disintegrate under irradiation
	by a process called re-solution.
	The interplay between the trapping rate and the re-solution rate
	governs the evolution of these gas bubbles.
	In this study,
	binary collision approximation (BCA) and molecular dynamics (MD) simulations
	are performed to quantify the re-solution rate of Xe gas bubbles
	in U-10Mo fuel.
	First, the energy loss of fission fragments (FFs)
	due to electronic and nuclear stopping is evaluated.
	Electronic stopping can initiate thermal spikes along FF tracks.
	To quantify re-solution due to thermal spikes,
	MD simulations coupled with the two-temperature model (TTM) are performed.
	It is found that thermal spikes cannot bring about re-solution in U-10Mo.
	For re-solution due to nuclear stopping,
	FF behavior in U-10Mo is simulated using BCA
	to get average FF incidence probability, energy and angle
	as a function of distance from the FF origin.
	Afterward, interactions between
	FFs of different energies and Xe gas bubbles in U-10Mo
	are assessed.
	Based on all the information gathered from BCA simulations,
	an overall Xe gas bubble re-solution rate $b$ is computed.
	$b/\dot{F}$
	ranges from \num{4.4e-26} m$^3$/fission for intergranular bubbles
	to \num{8.7e-25} m$^3$/fission for intragranular bubbles,
	where $\dot{F}$ represents the fission rate in the fuel.
	The effect of bubble pressure on re-solution rate is also evaluated.
\end{abstract}


\section{Introduction}

A U-10Mo alloy-based monolithic fuel design was identified
as the fuel type for converting U.S. High-Performance Research Reactors (HPRRs)
\cite{meyer2014} from high enriched fuel to low enriched fuel.
To understand the fuel's behavior under irradiation,
mesoscale and engineering-level fuel performance models require
knowledge of the fundamental mechanistic behavior of fission products
within the fuel to describe key phenomena,
such as swelling \cite{beeler2018gb, annualreport2021}.
Specifically, understanding the progression of Xe gas bubbles in the fuel
is crucial for optimizing reactor performance and safety.
These Xe gas bubbles act as a sink for individual Xe atoms,
trapping them and causing the bubbles to grow after absorption.
Under irradiation, the Xe atoms in the gas bubble are reintroduced
into the fuel matrix through fission-product-induced cascades
and thermal spikes---a process known as re-solution.
The relative rates of the re-solution affect the overall size and density
of the bubbles \cite{ye2023, olander2006re, parfitt2008},
in turn impacting bubble evolution and subsequent fuel swelling.
Re-solution of fission gas in nuclear fuels involves
two commonly accepted mechanisms:
homogeneous re-solution and heterogeneous re-solution.
In homogeneous re-solution, atoms from the gas bubbles are ejected individually
through collisions with fission products
or the recoil atoms that traverse the bubbles.
These atomic collision cascades are primarily
governed by the nuclear stopping power of the material.
In the heterogeneous model, a portion of gas bubbles are dissolved
by a passing fission fragment (FF) in the vicinity.
The driving mechanism is
the local heating of the material containing the gas bubbles,
through the electronic stopping of the FFs \cite{setyawan2018}.
The fact that both these mechanisms occur on a short timescale
makes it challenging to conduct experiments for determining re-solution rates
that contribute to the fission gas release models.
Thus, atomistic-scale modeling is necessary
for determining the re-solution rate
and for elucidating the fundamental mechanism
behind re-solution in U-10Mo.

In the literature up to this point,
atomistic simulations have been widely used
to evaluate the re-solution rate in various nuclear materials.
For instance, in 2008, Parfitt et al. \cite{parfitt2008}
used simulations of primary knock-on atoms (PKAs) in uranium dioxide (UO$_2$)
to assess the re-solution of helium gas bubbles.
In 2009, Schwen et al. \cite{schwen2009md} investigated
the homogeneous re-solution of Xe gas bubbles in UO$_2$,
using binary collision approximation (BCA) and molecular dynamics (MD).
The following year, Huang et al. \cite{huang2010md}
examined the impact of thermal spikes on Xe re-solution in UO$_2$.
In 2012, Govers et al. \cite{govers2012}
performed PKA and thermal spike simulations of Xe gas bubbles in UO$_2$
and proposed a mathematical model for the re-solution rate.
However, the most comprehensive work on Xe gas bubble re-solution in UO$_2$
was conducted  in 2018 by Setyawan et al. \cite{setyawan2018}.
They reconciled the inconsistencies found in the conclusions of previous works
on Xe bubble re-solution in UO$_2$
and evaluated the re-solution rate as a function of bubble radius.
Their findings suggest that heterogeneous re-solution of gas bubbles
is the dominant method of re-solution in UO$_2$.
In addition to UO$_2$,
the re-solution rate of fission gas bubbles was also evaluated
in uranium carbide (UC) by Matthews et al. \cite{matthews2015diss},
using BCA.
The thermal spike model was not employed in UC
because it was assumed that the local heating does not exceed
the melting temperature \cite{matthews2015diss, ronchi1986}.
In summary, BCA and MD simulations were both used
to determine the re-solution rate in nuclear fuels.

In MD simulations of homogeneous re-solution,
a regular lattice atom is typically endowed with high kinetic energy
to emulate a PKA.
The PKA then interacts ballistically with other atoms,
initiating a collision cascade near the gas bubble
and inducing disorder \cite{parfitt2008, govers2012}.
One alternative approach in MD is to simulate
only a portion of the cascade (i.e., a subcascade)
by imparting energy to a random gas atom within the bubble.
In doing so, the simulation avoids unnecessary cascade events
that may not significantly influence the re-solution process.
However, BCA must be utilized in this approach
to first obtain an energy spectrum of the gas atom PKAs \cite{schwen2009md}.
One challenge in using MD simulations to model homogeneous re-solution
is the channeling of PKAs or their recoils over long distances,
without any collisions \cite{jarrin2021}.
This can make collecting statistics
on the interactions between PKAs and gas bubble atoms computationally demanding,
especially when the PKA direction is random.
A potential solution is to direct the PKAs
toward a high index lattice direction \cite{stoller2000}.
Moreover, collision cascades end up in heat spikes due to nuclear stopping.
In that regard, the simulation of PKAs
also encompasses the heterogeneous re-solution.
To identify atoms that are re-solved ballistically, a threshold atomic speed,
above which it is improbable to find atoms in thermal equilibrium,
can be utilized \cite{parfitt2008}.
For MD simulations of heterogeneous re-solution due to swift heavy ions,
the thermal spike model is normally employed.
This model is useful for describing
the interaction between the FFs and the fuel.
These interactions occur primarily via
electronic stopping of the energetic particles
that initially raise the electronic subsystem temperature.
The energy deposited in the electronic subsystem can then transfer
to the lattice as thermal energy via electron-phonon coupling.
Finally, the energy is transferred among the atoms,
leading to a rapid increase in lattice temperature
within a cylindrical zone of typically a few nm in radius.
This increase is known as a thermal spike
\cite{wang1994, toulemonde2002, patra2019}.
In MD simulations, electronic interactions cannot be treated directly.
However, the final step described above can be emulated
by raising the temperature of atoms within a cylindrical region.

For qualification of U-10Mo fuel,
the ability to accurately predict the fission gas atom evolution
under various operational and transient conditions is crucial.
The Dispersion Analysis Research Tool (DART),
developed by Argonne National Laboratory \cite{ye2023},
is a mesoscale code that can calculate fission gas swelling
in U-10Mo under different operational situations.
One of the many parameters required to model swelling behavior
is the re-solution rate of fission gas bubbles.
DART employs a re-solution model
that includes a piecewise function to account for the bubble radius.
The parameters in this function are calibrated
by fitting the computed swelling value to experimental data,
meaning that we can only roughly estimate the re-solution rate.
A physics-based re-solution rate for fission gas bubbles would make
the swelling calculations of higher-length-scale models more rigorous.
In the present study, we utilize BCA and MD simulations
to investigate the re-solution of Xe gas bubbles in U-10Mo fuel,
considering both the homogeneous and heterogeneous re-solution mechanisms.


\section{Computational methods}

In this section, we discuss the general computational methods.
Specific simulation details will be discussed along with the results
for better readability.

\subsection{Binary collision approximation}

RustBCA, a free and open-source software, has been used for all BCA simulations.
This software can simulate ion-material interactions
including sputtering, implantation, and reflection (cite).
Out of the box, RustBCA supports infinite homogeneous 0D targets,
finite-depth layered inhomogeneous 1D targets,
inhomogeneous 2D targets through a triangular mesh,
and homogeneous 3D triangular mesh geometry.
However, it does not support inhomogeneous 3D geometry,
which is necessary to emulate gas bubbles embedded in solids.
For that reason, we implemented a specific 3D geometry in RustBCA
called \texttt{SPHEREINCUBOID}
that allows the user to specify a spherical material inside another cuboid one
(cite).
The electronic stopping in the BCA simulations performed in this work
is described by the Biersack-Varelas interpolation (cite),
and the nuclear interactions are described by the universal Kr-C potential
(cite).

% talk about amorphous U-10Mo

\subsection{Molecular dynamics}

The LAMMPS software package \cite{lammps} was utilized for the MD work,
in conjunction with a U-Mo-Xe angular-dependent potential (ADP)
\cite{smirnova2013, starikov2018, beelerADP}.
This ADP can accurately describe
the body-centered cubic phase of $\gamma$U-Mo alloys,
reproducing their stable structure, modulus of elasticity,
room temperature density, and melting point.
Given that the present work focuses on
high-energy interactions pertaining to thermal spike events,
the distances between the atoms can become small.
On its own, the classical interatomic potential
does not accurately describe extremely small distance interactions.
To address this issue,
the Ziegler, Biersack, and Littmark (ZBL) potential \cite{ziegler1985},
which provides a realistic depiction of ion-ion repulsive interactions,
was combined with the ADP.
The inner and outer cutoff distances for the ZBL potential
are chosen to be 1 \r{A} and 2 \r{A}, respectively.


\section{Energy loss of fission fragments in U-10Mo}

To understand and quantify re-solution in U-Mo,
the behavior of FFs need to be studied first.
The fission of \ce{_{92}^{235}U} can produce many different isotopes.
To keep computational complexity manageable,
two isotopes, \Y and \I, are chosen as representative of light and heavy FFs.
These two isotopes are created in the fuel due to the following reaction:
\begin{align}
\ce{
	_0^1n + _{92}^{235}U -> _{39}^{97}Y + _{53}^{136}I + Q
}
\end{align}
While \Y has an initial kinetic energy of about 101.3 MeV,
\I has about 74.6 MeV.
The rationale behind choosing these two as representative FFs is that
they appear close to the two peaks in a typical fission product yield graph,
and these isotopes have a fission product yield of about 0.12
\cite{setyawan2018, mills1995}.

To evaluate the energy loss of these FFs in U-10Mo,
BCA simulations are set up using RustBCA's 0D geometry option.
In these simulations, U-10Mo extends from $0$ to $\infty$ in the $x$ direction,
and $-\infty$ to $\infty $ in $y$ and $z$ directions.
The FFs are created at $(0, 0, 0)$ with a direction of $(1, 0, 0)$.
For each FF, 2,000 independent ion-material simulations are performed.
FF positions and velocities from every single binary collision
are processed and binned together
to obtain electronic and nuclear stopping profiles.
These profiles are shown in Figure \ref{fig:stopping}.
At the beginning, the electronic stopping power
is bounded by 20 keV/nm for both FFs.
As expected, the nuclear stopping power
has a characteristic Bragg peak towards the end of FF trajectory (cite).
The nuclear energy loss of \Y is about $5\%$ of its total initial energy,
whereas the nuclear energy loss of \I is about $10\%$ of the total.

\begin{figure}[!ht]
\begin{subfigure}{0.49\textwidth}
	\centering
	\caption{}
	\includegraphics[width=8cm]{images/Y_stopping.pdf}
\end{subfigure}
\begin{subfigure}{0.49\textwidth}
	\centering
	\caption{}
	\includegraphics[width=8cm]{images/I_stopping.pdf}
\end{subfigure}
\caption{
	Nuclear and electronic stopping power of (a) \Y and (b) \I.
}
\label{fig:stopping}
\end{figure}

Based on the information from the stopping power profiles,
a two-pronged approach is employed to assess the gas bubble re-solution rate.
To analyze the effect of electronic stopping,
the two-temperature model (TTM) is utilized to emulate the transfer of energy
from the electronic subsystem to the ionic subsystem,
and this is discussed in section \ref{sec:elec}.
For nuclear stopping,
the interactions among FFs, U-10Mo, and Xe gas bubbles are simulated
using the newly implemented \texttt{SPHEREINCUBOID} geometry in RustBCA,
and this is discussed in detail in section \ref{sec:nuke}.


\FloatBarrier
\section{Re-solution due to electronic stopping}
\label{sec:elec}

% add TTM equations here
% add TTM parameters as well

MD simulations incorporating TTM have been performed
to quantify re-solution due to electonic stopping.
No re-solution has been observed even at 30 keV/nm.
This is consistent with the findings of Kolotova et al.


\section{Re-solution due to nuclear stopping}
\label{sec:nuke}

\subsection{Model for re-solution calculation}

\begin{figure}[ht]
	\centering
	\includegraphics[width=13cm]{images/rotation.pdf}
	\caption{
		Rotation of fission event positions and velocities around the origin
		such that all velocities point to the $-x$ direction.
		Red dots represent positions and gray arrows represent velocities.
	}
	\label{fig:rot}
\end{figure}

\begin{figure}[ht]
	\centering
	\includegraphics[width=5cm]{oldimg/coordSystem.pdf}
	\caption{
		A coordinate system where a Xe gas bubble is at the origin,
		and fission fragments are all pointing at the $-x$ direction.
	}
	\label{fig:coord}
\end{figure}

The re-solution rate can be defined as the probability
of a Xe atom going out of the gas bubble and into the fuel matrix,
per unit time.
To represent the overall re-solution behavior in the material,
the contributions from all the fission products,
originating at different distances from a bubble
and oriented toward a random direction,
need to be summed up by means of a volume integral.
Consider a 3D coordinate system in which the gas bubble is at the origin.
For ease of calculation,
fission product origins can be rotated around the coordinate system origin
so that all fission tracks point in the same direction (Figure \ref{fig:rot}).
Given that the fission product generation is
uniform and isotropic in the material,
the aforementioned rotational transformation will lead to
a uniform distribution of unidirectional fission products
as shown in Figure \ref{fig:coord}a.
The axial coordinate $x$ is then defined as parallel to the fission tracks,
and the radial coordinate $w$ as perpendicular to the tracks
(Figure \ref{fig:coord}b).
If we denote the fission rate as $\dot{F}$,
the number of fission events per second
in an infinitesimal volume $dV = 2 \pi w \: dw  \: dx$ would be $\dot{F} dV$.
Also, we assume that
when a fission track produced by isotope $k$ originating at $(x, w)$
interacts with a Xe gas bubble at the origin,
it re-solves $\xi_k \equiv \xi_k(x, w)$ fraction of the Xe atoms.
Each fission product isotope $k$ from fission events thus contributes
$\xi_k(x, w) \dot{F} dV$ to the total re-solution of the bubble.
The re-solution rate $b$ can then be expressed as:
\begin{align}
	b &= \sum_{k = Y, I} \int_V \xi_k(x, w) \dot{F} dV \label{eq:b} \\
	  &= \dot{F} \sum_{k = Y, I} \int_V \xi_k(x, w) dV
	   = \dot{F} \left( \int_V \xi_Y(x, w) dV + \int_V \xi_I(x, w) dV \right) \\
	  &= \dot{F} \sum_{k = Y, I} \int_{x=0}^{\infty} \int_{w=0}^{\infty}
		\xi_k(x, w) 2 \pi w dw dx
\end{align}

\FloatBarrier
\subsection{Reference fission fragment simulations}

% simulation visuals using vispy here
\begin{figure}[ht]
	\centering
	\includegraphics[width=8cm]{images/surface_discr.png}
	\caption{
		Surface discretization
	}
	\label{fig:surf_discr}
\end{figure}

% the colorbars need labels
\begin{figure}[ht]
	\centering
	\begin{subfigure}{0.49\textwidth}
		\centering
		\caption{}
		\includegraphics
			[width=8cm, trim={0.8cm 0 1.5cm 0.4cm}, clip]
			{images/Y_p.pdf}
	\end{subfigure}
	\begin{subfigure}{0.49\textwidth}
		\centering
		\caption{}
		\includegraphics
			[width=8cm, trim={0.8cm 0 1.5cm 0.4cm}, clip]
			{images/I_p.pdf}
	\end{subfigure}
	\begin{subfigure}{0.49\textwidth}
		\centering
		\caption{}
		\includegraphics
			[width=8cm, trim={0.8cm 0 1.5cm 0.4cm}, clip]
			{images/Y_e.pdf}
	\end{subfigure} \begin{subfigure}{0.49\textwidth}
		\centering
		\caption{}
		\includegraphics
			[width=8cm, trim={0.8cm 0 1.5cm 0.4cm}, clip]
			{images/I_e.pdf}
	\end{subfigure}
	\begin{subfigure}{0.49\textwidth}
		\centering
		\caption{}
		\includegraphics
			[width=8cm, trim={0.8cm 0 1.5cm 0.4cm}, clip]
			{images/Y_a.pdf}
	\end{subfigure}
	\begin{subfigure}{0.49\textwidth}
		\centering
		\caption{}
		\includegraphics
			[width=8cm, trim={0.8cm 0 1.5cm 0.4cm}, clip]
			{images/I_a.pdf}
	\end{subfigure}
	\caption{
		Ion incidence probability per unit surface area of (a) \Y and (b) \I.
		Average incidence energy of (c) \Y and (d) \I.
		Average incidence angle of (e) \Y and (f) \I.
	}
	\label{fig:ref}
\end{figure}

The most straightforward way to calculate $\xi_k(x, w)$ would be
simulating a FF and a Xe gas bubble for that specific $(x, w)$ value.
However, doing this for all necessary $(x, w)$ values
is computationally expensive and leads to statistically unreliable results.
As the distance between the origin of a FF and a Xe gas bubble increases,
the probability of any interaction between them diminishes greatly.
The interaction probability is also dependent of bubble size.
The smaller the bubble, the lower the probability.
Therefore, for some bubble sizes and $(x, w)$ values,
a few hundred thousand BCA simulations might only lead to few interactions,
making most of the simulations a poor use of computational resources.

An alternative of the brute-force approach is to first assess
the behavior of FFs in the fuel without the presence of any bubbles.
If the probability of an FF
going through a certain point in the fuel with a certain energy is known,
simulating local FF-bubble interactions at that exact point would be easy.
Thus, FF simulations in U-10Mo are performed to obtain three key properties:
the probability of an FF passing through a particular point per unit area,
its average incidence energy and average incidence angle.
In these reference simulations,
the FF starts from the origin and is directed toward the $x$ direction.
Notice that this setup is different
from the one described in Figure \ref{fig:coord}b,
where the Xe gas bubble was at the origin.
To gather the data from the simulations,
an annular surface discretization scheme,
as shown in Figure \ref{fig:surf_discr}, is used.
The discretized surface elements are associated
with specific discretized $(x, w)$ values,
and $\Delta x = \Delta w = 500$ \r{A}.
This discretization is fine enough to provide smooth profiles of FF properties.
The results from the simulations and subsequent processing
are shown in \ref{fig:ref}.

% haven't talked about convergence; the 6-point convergence test

% need a paragraph here describing the FF profiles

\FloatBarrier
\subsection{Fission fragment interactions with Xe gas bubbles}

Now that reference FF properties are available,
local FF-bubble interactions need to be simulated.
For these simulations, equilibrium Xe number densities are used for all bubbles.
The van der Waals equation of state (EOS), as described in Olander ??,
is used to compute the equilibrium number densities (Eq. \ref{eq:vdw}).
The equilibrium bubble pressure $p_{eq}$ utilized in that EOS
comes from the Young-Laplace equation (cite).
The surface tension $\gamma$ in U-10Mo is 1.55 J/m$^2$ (cite).
The resultant equilibrium number densities
are plotted against bubble radius in Figure \ref{fig:vdw}.
The van der Waals EOS predicts a plateauing equilibrium number density
for smaller bubbles.
\begin{align}
	n &= \bigg( B + \frac{kT}{p} \bigg)^{-1} \label{eq:vdw} \\
	p_{eq} &= \frac{2 \gamma}{R_b} \\
	n_{eq} &= \bigg( B + \frac{kT R_b}{2 \gamma} \bigg)^{-1}
\end{align}

\begin{figure}[ht]
	\centering
	\includegraphics[width=8cm]{images/n_vdw.pdf}
	\caption{
		Equilibrium Xe number density in gas bubbles
		calculated from the van der Waals equation of state.
	}
	\label{fig:vdw}
\end{figure}

FF-bubble interactions should be simulated in a way such that
all pertinent recoils are accounted for.
To that end, 10 FF BCA simulations are analyzed to assess recoil behavior.
Figure \ref{fig:recoil} shows a scatter plot
of the recoil displacements and their energies.
The largest amount of energy transferred to the recoils is about 1 MeV,
and the recoil displacement at that energy is at most about 100 nm.
% fix this angstrom vs nm problem; just go with nm everywhere
It is possible to transfer more than 1 MeV to a recoil in a head-on collision,
but it is highly unlikely.
Therefore, it is reasonable to assume
the recoils generated by FFs more than $\delta = 100$ nm away from a bubble
cannot interact with that bubble.
As such, in the FF-bubble BCA simulations,
the FFs are placed at least $D = R_b + \delta$ distance away
from the bubble center.

\begin{figure}[ht]
	\centering
	\includegraphics[width=8cm]{images/recoil_dr.pdf}
	\caption{
		Recoil displacements as a function of recoil energy.
	}
	\label{fig:recoil}
\end{figure}

% We need a simulation setup here showing where the FF starts

% simulation visuals here using vispy

Now comes the problem of determining which Xe recoils are re-solved.
In this work, we consider Xe atoms that end up
at least $\lambda = 1$ nm away from the bubble surface as re-solved Xe atoms.
In reality, bubble surface is not a static structure
due to the thermal fluctuations of the Xe atoms that make up the surface.
Also, a Xe atom that is just outside the surface is highly likely
to end up in the bubble in a short timeframe.
Thus, there exists a finite annular region outside the bubble surface
that needs to cleared by a Xe atom to be considered fully re-solved.
Our choice of $\lambda$ is also informed by the literature (cite).
Figure \ref{fig:xeres} depicts the displacements of Xe recoils
from a FF-bubble simulation (details please, 200 sims).
From a recoil displacement vs energy graph such as Figure \ref{fig:xeres}a,
it is also possible to find a minimum threshold energy $E_{min}$
required for re-solution.
$\lambda$ and $E_{min}$ are thus interconnected.
While some re-solution studies made a subjective choice on $\lambda$ (cite),
some did it on $E_{min}$ (cite).
From our simulations, we find $E_{min}$ to be about 20 eV for $\lambda=1$ nm.

\begin{figure}[!ht]
\begin{subfigure}{0.49\textwidth}
	\centering
	\caption{}
	\includegraphics[width=8cm]{images/xe_dr.pdf}
\end{subfigure}
\begin{subfigure}{0.49\textwidth}
	\centering
	\caption{}
	\includegraphics[width=8cm]{images/xe_hist.pdf}
\end{subfigure}
\caption{
	(a) Xe recoil final position against recoil energy.
	(b) Xe recoil final position histogram.
}
\label{fig:xeres}
\end{figure}

For FF-bubble simulations, both energy $E$ and off-centered distance $\ell$
are discretized.
The energy values are discretized based on which FF is simulated.
% add a table of discrete energy values? or just state them?
$\ell$ discretization is based on the bubble radius.
The $\ell$ values are chosen in such a way that
the region around $R_b$ is sampled properly.
% ell values are also important. supplementary information?
The results for two 2 nm and 64 nm bubbles are shown in Figure \ref{fig:chi}.
2 nm and 64 nm bubbles are representative
of intragranular and intergranular bubbles, respectively.

% talk about convergence; number of simulations

% use scientific notation for tick labels and make them consistent
\begin{figure}[ht]
	\centering
	\begin{subfigure}{0.49\textwidth}
		\centering
		\caption{}
		\includegraphics[width=8cm]{images/chi_2nm_Y.pdf}
	\end{subfigure}
	\begin{subfigure}{0.49\textwidth}
		\centering
		\caption{}
		\includegraphics[width=8cm]{images/chi_2nm_I.pdf}
	\end{subfigure}
	\begin{subfigure}{0.49\textwidth}
		\centering
		\caption{}
		\includegraphics[width=8cm]{images/chi_64nm_Y.pdf}
	\end{subfigure}
	\begin{subfigure}{0.49\textwidth}
		\centering
		\caption{}
		\includegraphics[width=8cm]{images/chi_64nm_I.pdf}
	\end{subfigure}
	\caption{
		$\chi(E, \ell)$ for bubble radius of 2 nm
		with incident (a) \Y and (b) \I.
		$\chi(E, \ell)$ for bubble radius of 64 nm
		with incident (c) \Y and (d) \I.
	}
	\label{fig:chi}
\end{figure}

\FloatBarrier
Now, let's define $\chi(E', \ell')$ as the fraction of Xe atoms
that get re-solved due to the interaction with an FF
that is $(D, \ell')$ ?? away from the bubble center and has an energy $E'$.
With the simulation results and interpolation,
it is now possible to obtain any reasonable $\chi(E', \ell')$ value.
If $\mathcal{I}$ is an interpolator that uses the mapping $X \rightarrow Y$,
we can use the notation $y' = \mathcal{I}_X (x', X, Y)$ to mean that
the interpolator returns $y'$ when $x'$ is provided as an input.
Any arbitrary $\chi(E', \ell')$ is then defined the following way:
\begin{align}
	\chi(E', \ell)
	   &= \mathcal{I}_{\mathcal{E}}
	   (E', \mathcal{E}, [\chi(E, \ell)]_{E \in \mathcal{E}}) \\
	\chi(E', \ell')
	   &= \mathcal{I}_{\mathcal{L}}
	   (\ell', \mathcal{L}, [\chi(E', \ell)]_{\ell \in \mathcal{L}})
\end{align}
where $\mathcal{E}$ and $\mathcal{L}$ are the sets of discrete energies
and off-centered distances that are used in the simulations,
and $E$ and $\ell$ are elements of those sets.
$\mathcal{I}_{\mathcal{E}}$ is a Pchip interpolator (cite),
and $\mathcal{I}_{\mathcal{L}}$ is a linear interpolator (cite).

\subsection{Calculation of \texorpdfstring{$\xi$}{xi}}

Equipped with $\chi$ values, we proceed to calculate $\xi$.
Imagine an FF at the origin and directed toward $x$, and a bubble at $(x, w)$.
At $(x, w)$, the FF has an average incidence angle $\alpha(x, w)$.
The incidence angle changes slowly with changes in $x$ and $w$,
and thus it is reasonable to assume the incidence angle
in the vicinity of $(x, w)$ is simply $\alpha(x, w)$.
Next, we find a point $(x', w')$ such that
it is $D$ distance away from $(x, w)$
and the line connecting these two points has a slope $\tan \alpha(x, w)$.
A surface $S$ perpendicular to that connecting line can then be constructed.
$S$ is square in shape and has a side length $2D = 2(R_b + \delta)$.
This surface is then meshed into small elements
as displayed in Figure \ref{fig:surf_mesh}.
Now, the FF has certain probabilities of going through each mesh element.
If the FF goes through a mesh element $m$
with probability per unit area $p(r_m)$ and incidence energy $E(r_m)$,
$\xi(x, w)$ can be calculated using Eq. \ref{eq:mesh}.
Here, $r_m$ denotes the coordinate of the mesh element center,
and $r_c$ denotes the center of the surface $S$.
The total probability of the FF going through the mesh element $m$
is calculated by the product of $p(r_m)$
and the area of the mesh element $A_m$
projected to the direction perpendicular of $x$.
\begin{align}
	\xi(x, w) &= \sum_{m \in S}
		p(r_m) \frac{A_m}{\cos \alpha(x, w)}
		\chi(E(r_m), ||r_m - r_c||) \label{eq:mesh}
\end{align}

\begin{figure}[ht]
	\centering
	\includegraphics[width=12cm]{images/surface_mesh.png}
	\caption{
		Surface discretization
	}
	\label{fig:surf_mesh}
\end{figure}

% adaptive mesh element size; convergence

$\xi$ is calculated for all combinations of FF isotopes and bubble radii.
$\xi_Y$ for 2 nm and 64 nm bubbles
are shown in Figures \ref{fig:xi}a and \ref{fig:xi}b.
These $\xi$ profiles can also be interpreted in a way
that enables the calculation of the overall re-solution rate.
If a bubble is at the origin and a FF is at $(x, w)$ pointing toward $-x$,
the $\xi$ profiles we would get are exactly the same.
A discretized version of Eq. \ref{eq:b} can then be formulated as follows:
\begin{align}
	b / \dot{F} &= \sum_{k=Y,I} \sum \xi_k \Delta V \\
		&= \sum_{k=Y,I} \sum_{i,j \in V} \xi_k(c_{i,j})
		\pi (w_{i,j+1}^2 - w_{i,j}^2) (x_{i+1,j} - x_{i,j})
\end{align}
Since our $\xi$ profiles are already discretized,
this leads to an easy calculation.
$\xi \Delta V$ profiles are also depicted in Figure \ref{fig:xi}
to show the volumetric effect on re-solution.

\begin{figure}[ht]
	\centering
	\begin{subfigure}{0.49\textwidth}
		\centering
		\caption{}
		\includegraphics
			[width=8cm, trim={0.8cm 0 1.5cm 0.7cm}, clip]
			{images/2nm_Y_xi.pdf}
	\end{subfigure}
	\begin{subfigure}{0.49\textwidth}
		\centering
		\caption{}
		\includegraphics
			[width=8cm, trim={0.8cm 0 1.4cm 0.7cm}, clip]
			{images/2nm_Y_db.pdf}
	\end{subfigure}
	\begin{subfigure}{0.49\textwidth}
		\centering
		\caption{}
		\includegraphics
			[width=8cm, trim={0.8cm 0 1.4cm 0.7cm}, clip]
			{images/64nm_Y_xi.pdf}
	\end{subfigure}
	\begin{subfigure}{0.49\textwidth}
		\centering
		\caption{}
		\includegraphics
			[width=8cm, trim={0.8cm 0 1.4cm 0.7cm}, clip]
			{images/64nm_Y_db.pdf}
	\end{subfigure}
	\caption{
		(a) $\xi$ and (b) $\xi \Delta V$
		for bubble radius of 2 nm with incident \Y.
		(c) $\xi$ and (d) $\xi \Delta V$
		for bubble radius of 64 nm with incident \Y.
	}
	\label{fig:xi}
\end{figure}

\FloatBarrier
\subsection{Homogeneous re-solution rate}

Summing up all the values in $\xi_Y \Delta V$ profiles
provides the re-solution rate of a bubble due to \Y.
The re-solution rate due to \I is also calculated similarly
and then used to get the total re-solution rate of that bubble.
Figure \ref{fig:res} displays the re-solution rates
for bubbles with radius ranging from 1 nm to 128 nm.

\begin{figure}[ht]
	\centering
	\includegraphics[width=8cm]{images/bhom.pdf}
	\caption{
		Homogeneous re-solution rate as a function of bubble radius $R_b$
		at equilibrium Xe number density $n_{eq}$ in U-10Mo.
	}
	\label{fig:res}
\end{figure}

\subsection{Effect of bubble pressure}

The effect of pressure on re-solution is investigated
by varying the Xe number density of bubbles of two different radii.
FF-bubble interaction simulations are performed
for \Y and \I of energies 1 MeV and 20 MeV.
$\chi$ values calculated from these simulations
are plotted in Figure \ref{fig:pres}.
An inversely proportional relationship between $\chi$ and $n$ is observed.
% Thus, $b / b_{eq} = n_{eq} / n$.

\begin{figure}[ht]
	\centering
	\includegraphics[width=8cm]{images/pressure.pdf}
	\caption{
		Effect of Xe number density on homogeneous re-solution rate.
	}
	\label{fig:pres}
\end{figure}

% \subsection{Re-solution rate as a function of bubble size and pressure}


\section{Discussion}

The results from this work are compared against the model prediction
used in Dispersion Analysis Research Tool (DART) \cite{ye2023}
and shown in Figure \ref{fig:comp}.

% also need to compare with uo2, uc and uzr

\begin{figure}[ht]
	\centering
	\includegraphics[width=8cm]{images/comp.pdf}
	\caption{
		Comparison of the DART model prediction for re-solution
		with the results from this work.
	}
	\label{fig:comp}
\end{figure}


\FloatBarrier
\section{Conclusion}

This study utilizes BCA and MD simulations
to determine the re-solution rate of Xe gas bubbles in U-10Mo nuclear fuel.
Both the homogeneous and heterogeneous re-solution mechanisms are investigated.
Thermal spikes initiated by electronic stopping cannot cause re-solution.
Thus, the occurrence of heterogeneous re-solution in U-10Mo is not probable.
Homogeneous re-solution, which is brought about by nuclear stopping,
is found to be the only mechanism of re-solution in U-10Mo.
Therefore, the re-solution rate is calculated
by first obtaining FF behavior in the fuel,
then evaluating the FF interactions with Xe gas bubbles,
and finally putting all the information together in a physical model.
The computed re-solution rate $b$ for intergranular bubbles
is $\num{4.4e-26} \dot{F}$ s$^{-1}$
and for intragranular bubbles $\num{8.7e-25} \dot{F}$ s$^{-1}$,
where the unit of $\dot{F}$ is fission/m$^3$/s.
Furthermore, BCA simulations with varying Xe number density in the bubble
revealed that
the re-solution rate is inversely proportional to the Xe number density.
Thus, higher bubble pressure leads to a lower re-solution rate.
The results of this study will inform higher-length-scale models of U-10Mo
with a physics-based description of the Xe gas bubble re-solution rate.


\bibliographystyle{unsrt}
\bibliography{ref.bib}

\end{document}

\documentclass{elsarticle}
\usepackage[margin=1in]{geometry}
\usepackage{graphicx, xcolor, placeins}
\usepackage{amsmath, booktabs, subcaption}
\usepackage{mathtools, lineno}

\begin{document}


\begin{frontmatter}

\title{Re-solution of Xe gas bubbles in $\gamma$U-10Mo fuel}
\author[inl,ncsu]{ATM Jahid Hasan}
\author[inl]{Linu Malakkal}
\author[inl]{Mathew Swisher}
\author[inl,ncsu]{Benjamin Beeler}
\address[inl]{Idaho National Laboratory,
Idaho Falls, ID 83415, United States}
\address[ncsu]{North Carolina State University,
Raleigh, NC 27695, United States}

\begin{abstract}
	The U.S. High-Performance Research Reactor (USHPRR) program aims to convert
	high enriched fuel in high-power research reactors to low enriched fuel.
	The choice for this conversion is a fuel type based on the $\gamma$U-10Mo
	alloy. The behavior of fission product gases, such as Xe, in the fuel needs
	to be well understood for evaluating the performance of the $\gamma$U-10Mo
	fuel. Xe gas bubbles play a pivotal role in this fuel type by trapping more
	fission product gas atoms and growing in size. Also, parts of the bubbles
	can disintegrate under irradiation by a process called re-solution. The
	interplay between the trapping rate and the re-solution rate governs the
	evolution of these gas bubbles. In this study, molecular dynamics (MD)
	simulations were performed to quantify the re-solution rate of Xe gas
	bubbles in $\gamma$U-10Mo fuel. Two models---namely, the primary knock-on
	atom (PKA) and thermal spike models---were employed to evaluate homogeneous
	and heterogeneous re-solution, respectively. Notably, homogeneous
	re-solution was found to be negligible compared to heterogeneous
	re-solution. In the thermal spike simulations, Xe gas bubbles of varying
	radii and pressure were considered. The thermal spike axis was also
	adjusted to account for both on- and off-centered collisions with the gas
	bubble. Subsequently, the probability of Xe gas atoms being re-solved into
	the $\gamma$U-10Mo matrix was evaluated as a function of bubble radius,
	bubble pressure, thermal spike energy ($S_{e,eff}$), and off-centered
	distance. Additionally, the electronic stopping power ($S_e$) of fission
	products was simulated in order to determine the amount of energy imparted
	by the fission products at specific distances. To further refine the
	analysis, a fraction ($\zeta$) of the energy imparted by fission products
	was equated to the thermal spike energy ($\zeta S_e = S_{e,eff}$). Then
	finally, all this information was used to compute an overall Xe gas bubble
	re-solution rate in the $\gamma$U-10Mo fuel as a function of bubble radius,
	bubble pressure, and fission rate, considering reasonable $\zeta$ values.
\end{abstract}

\end{frontmatter}


%\linenumbers
\section{Introduction}

A $\gamma$U-10Mo alloy-based monolithic fuel design was identified as the fuel
type for converting U.S. High-Performance Research Reactors (HPRRs)
\cite{meyer2014} from high enriched fuel to low enriched fuel. To understand
the fuel's behavior under irradiation, mesoscale and engineering-level fuel
performance models require knowledge of the fundamental mechanistic behavior of
fission products within the fuel to describe key phenomena, such as swelling
\cite{beeler2018gb, annualreport2021}. Specifically, understanding the
progression of Xe gas bubbles in the fuel is crucial for optimizing reactor
performance and safety. These Xe gas bubbles act as a sink for individual Xe
atoms, trapping them and causing the bubbles to grow after absorption. Under
irradiation, the Xe atoms in the gas bubble are reintroduced into the fuel
matrix through fission-product-induced cascades and thermal spikes---a process
known as re-solution. The relative rates of the re-solution affect the overall
size and density of the bubbles \cite{ye2023, olander2006re, parfitt2008}, in
turn impacting bubble evolution and subsequent fuel swelling. Re-solution of
fission gas in nuclear fuels involves two commonly accepted mechanisms:
homogeneous re-solution and heterogeneous re-solution. In homogeneous
re-solution, atoms from the gas bubbles are ejected individually through
collisions with fission products or the recoil atoms that traverse the bubbles.
These atomic collision cascades are primarily governed by the nuclear stopping
power of the material. In the heterogeneous model, a portion of gas bubbles are
dissolved by a passing fission fragment in the vicinity. The driving mechanism
is the local heating of the material containing the gas bubbles, through the
electronic stopping of the fission fragments \cite{setyawan2018}. The fact that
both these mechanisms occur on a short timescale makes it challenging to
conduct experiments for determining re-solution rates that contribute to the
fission gas release models. Thus, atomistic-scale modeling is necessary for
determining the re-solution rate and for elucidating the fundamental mechanism
behind re-solution in $\gamma$U-10Mo.

% TODO: explanation homogeneous/cascades, low and high energy interactions

In the literature up to this point, atomistic simulations have been widely used
to evaluate the re-solution rate in various nuclear materials. For instance, in
2008, Parfitt et al. \cite{parfitt2008} used simulations of primary knock-on
atoms (PKAs) in uranium dioxide (UO$_2$) to assess the re-solution of helium
gas bubbles. In 2009, Schwen et al. \cite{schwen2009md} investigated the
homogeneous re-solution of Xe gas bubbles in UO$_2$, using a binary collision
model and molecular dynamics (MD) simulations. The following year, Huang et al.
\cite{huang2010md} examined the impact of thermal spikes on Xe re-solution in
UO$_2$. In 2012, Govers et al. \cite{govers2012} evaluated both the PKA and
thermal spike models for Xe gas bubbles in UO$_2$, and proposed a mathematical
model for the re-solution rate. However, the most comprehensive work on Xe gas
bubble re-solution in UO$_2$ was conducted  in 2018 by Setyawan et al.
\cite{setyawan2018}. They reconciled the inconsistencies found in the
conclusions of previous works on Xe bubble re-solution in UO$_2$ and evaluated
the re-solution rate as a function of bubble radius. Their findings suggest
that heterogeneous re-solution of gas bubbles is the dominant method of
re-solution in UO$_2$. In addition to UO$_2$, the re-solution rate of fission
gas bubbles was also evaluated in uranium carbide (UC) by Matthews et al.
\cite{matthews2015diss}, using binary collision methods. The thermal spike
model was not employed in UC because it was assumed that the local heating does
not exceed the melting temperature \cite{matthews2015diss, ronchi1986}. In
summary, the binary collision model and MD simulations were used to determine
the re-solution rate in nuclear fuels.

In MD simulations of homogeneous re-solution, a regular lattice atom is
typically endowed with high kinetic energy to emulate a PKA. The PKA then
interacts ballistically with other atoms, initiating a collision cascade near
the gas bubble and inducing disorder \cite{parfitt2008, govers2012}. One
alternative approach in MD is to simulate only a portion of the cascade (i.e.,
a subcascade) by imparting energy to a random gas atom within the bubble. In
doing so, the simulation avoids unnecessary cascade events that may not
significantly influence the re-solution process. However, a binary collision
model must be utilized in this approach to first obtain an energy spectrum of
the gas atom PKAs \cite{schwen2009md}. One challenge in using MD simulations to
model homogeneous re-solution is the channeling of PKAs or their recoils over
long distances, without any collisions \cite{jarrin2021}. This can make
collecting statistics on the interactions between PKAs and gas bubble atoms
computationally demanding, especially when the PKA direction is random. A
potential solution is to direct the PKAs toward a high index lattice direction
\cite{stoller2000}. For MD simulations of heterogeneous re-solution, the
thermal spike model is useful for describing the interaction between the
fission fragments and the fuel. These interactions occur primarily via
electronic stopping of the energetic particles that initially raise the
electronic subsystem temperature. The energy deposited in the electronic
subsystem can then transfer to the lattice as thermal energy via
electron-phonon coupling. Finally, the energy is transferred among the atoms,
leading to a rapid increase in lattice temperature within a cylindrical zone of
typically a few nm in radius. This increase is known as a thermal spike
\cite{wang1994, toulemonde2002, patra2019}. In MD simulations, electronic
interactions cannot be treated directly. However, the final step described
above can be emulated by raising the temperature of atoms within a cylindrical
region.

For qualification of $\gamma$U-Mo fuel, the ability to accurately predict the
fission gas atom evolution under various operational and transient conditions
is crucial. The Dispersion Analysis Research Tool (DART), developed by Argonne
National Laboratory \cite{ye2023}, is a mesoscale code that can calculate
fission gas swelling in $\gamma$U-Mo under different operational situations.
One of the many parameters required to model swelling behavior is the
re-solution rate of fission gas bubbles. DART employs a simple re-solution
model that includes a piecewise function to account for the bubble radius. The
parameters in this function are calibrated by fitting the computed swelling
value to experimental data, meaning that we can only roughly estimate the
re-solution rate. A physics-based re-solution rate for fission gas bubbles
would make the swelling calculations of higher-length-scale models more
rigorous. 

In the present study, we utilized MD simulations to investigate the re-solution
of Xe gas bubbles in $\gamma$U-Mo fuel---considering both the homogeneous and
heterogeneous re-solution mechanisms by simulating PKAs and thermal spikes,
respectively. We also calculated the electronic stopping power of
representative fission products so as to determine the energy imparted on gas
bubbles by fission events at a certain distance. Finally, we evaluated an
overall re-solution rate of Xe gas bubbles as a function of bubble radius,
bubble pressure, and fission rate.


\FloatBarrier
\section{Computational details}

The LAMMPS software package \cite{lammps} was utilized for the MD work, in
conjunction with a U-Mo-Xe angular-dependent potential (ADP)
\cite{smirnova2013, starikov2018, beelerADP}. This ADP can accurately describe
the body-centered cubic phase of $\gamma$U-Mo alloys, reproducing their stable
structure, modulus of elasticity, room temperature density, and melting point.
Given that the present work focuses on high-energy interactions pertaining to
PKAs and thermal spike events, the distance between the ions and atoms can
become extremely small. On its own, the classical interatomic potential does
not accurately describe these interactions. To address this issue in all the
simulations, the Ziegler, Biersack, and Littmark (ZBL) \cite{ziegler1985}
repulsive potential, which provides a realistic depiction of ion-ion repulsive
interactions, was combined with the ADP. The inner and outer cutoff distances
for the ZBL potential were chosen to be 1 \r{A} and 2 \r{A}, respectively.

For both the PKA and thermal spike simulations, the system was equilibrated at
400 K at a pressure of 0 bar in an NPT ensemble, using the Nos\'e-Hoover
barostat and thermostat. Utilizing the resulting configurations, spherical
bubbles of radius $R_{bubble}$ were created by removing U and Mo atoms and
depositing gaseous Xe atoms at a given density within the resulting void. The
system was then further equilibrated in the NPT ensemble. The simulations were
ultimately performed using an NVE ensemble with a canonical sampling thermostat
\cite{bussi2007} at the edges of the simulation box to act as a heat sink. This
enabled the resulting local heating to slowly dissipate from the simulation by
modeling a constant long-range temperature. Electronic stopping was
incorporated into the simulation by applying a frictional force to each atom.
This frictional force was calculated based on the stopping power of the related
atom species in the fuel. These values were determined using the Stopping and
Range of Ions in Matter (SRIM) software \cite{ziegler2010srim}. The frictional
force due to electronic stopping is applied to atoms whose kinetic energy
exceeds a specified threshold. This threshold is often set to 10 eV or double
the cohesive energy for metals \cite{nordlund1998, duffy2006}. In the present
work, electronic stopping was applied only when an atom's kinetic energy
exceeded 9 eV, which is double the U-U cohesive energy of 4.5 eV. The cohesive
energy was evaluated by Beeler et al. \cite{beeler2018disp} using the same ADP
potential applied in the present work. A variable timestep size was also
implemented for all simulations, such that any atom's maximum displacement
between two successive timesteps was less than or equal to 0.01 \r{A}.

For the PKA model, a $160 \times 160 \times 160$ \r{A}$^3$ supercell was
utilized, holding around 8 million atoms. Such a large supercell was needed to
prevent the self-interaction of cascades through their periodic images. The
thermostatted sink was implemented in all three directions. Figure
\ref{fig:struct} gives a cross section of such a supercell. Within a 20 \r{A}
radius gas bubble, the investigation was conducted on 200 and 400 Xe atoms,
corresponding to pressures of 124 MPa and 787 MPa, respectively. The bubbles
with 200 Xe atoms were underpressurized, whereas those with 400 Xe atoms led to
an overpressurized bubble. For reference, the equilibrium pressure of such a
bubble is 436 MPa \cite{beelerADP}. The cascade was initiated by selecting a
single PKA, located approximately 5.5 nm from the bubble's center. The PKA was
directed toward the bubble at high velocity. PKAs featuring kinetic energies of
up to 500 keV were simulated. These simulations were executed to model the
initial 120 ps of the cascade, ensuring that the total temperature of all the
examined systems fell below 600 K. For each bubble pressure and PKA energy
value, five simulations were performed---using different initial atomic
configurations and PKA directions---to gather statistical data.

\begin{figure}[ht]
	\centering
	\includegraphics[height=7cm]{images/struct.png}
	\caption{
		Initial configuration for an MD simulation of the PKA. The supercell is
		sliced through the middle to show the 20 \r{A} radius Xe gas bubble
		(black). The red and blue dots represent U and Mo atoms, respectively.
	}
	\label{fig:struct}
\end{figure}

In the thermal spike model, a $120 \times 120 \times 50$ \r{A}$^3$ supercell
was simulated, containing approximately 1.5 million atoms. The thermal spike
axis was aligned with the shortest dimension of the supercell, and the
thermostatted sink was applied only in directions perpendicular to the thermal
spike axis. Atoms within a cylindrical zone of 35 \r{A} radius along the
thermal spike axis were excited via high kinetic energy in order to simulate
local heating due to electronic stopping. This was achieved by rescaling the
atom velocities. To understand the role played independently by bubble size,
the Xe/vacancy ratio was at first kept constant. The Xe/vacancy ratio here is
defined as the ratio of the number of Xe atoms in the bubble to the number of
lattice sites occupied by the bubble. This ratio determines the bubble
pressure. A Xe/vacancy ratio of 0.2 was chosen for this purpose, since it
ensures an equilibrium bubble pressure, as per Beeler et al.
\cite{beeler2020improved}. The thermal spike's position within the supercell
was varied to account for both on- and off-centered collisions between the gas
bubble and thermal spike. For on-centered thermal spikes, gas bubbles with
radii of 5--40 \r{A} were simulated, with the thermal spike energies varying
from 5 to 30 keV/nm. For off-centered thermal spikes, gas bubbles of radii 15
\r{A}, 25 \r{A}, and 35 \r{A} were examined, with a thermal spike energy of 15
keV/nm. The off-centered distance was varied---at an interval of 10 \r{A}---up
to a specific cutoff value (discussed later). Bubbles with different Xe/vacancy
ratios were also simulated to account for the bubble pressure's effect on
re-solution. Xe/vacancy ratios of 0.1, 0.3, and 0.5 were used for bubbles of
radii 15 \r{A}, 25 \r{A}, and 35 \r{A}. Only on-centered thermal spikes were
employed in this case, and their energies were varied over a range of 5--30
keV/nm.

\begin{figure}[ht]
	\centering
	\begin{subfigure}{0.49\textwidth}
		\centering
		\caption{}
		\includegraphics[height=6cm]{images/temp_time.pdf}
	\end{subfigure}
	\begin{subfigure}{0.49\textwidth}
		\centering
		\caption{}
		\includegraphics[height=6cm]{images/resol_time.pdf}
	\end{subfigure}
	\caption{
		(a) System temperature vs. time for 40 \r{A} radius Xe gas bubble
		simulations featuring different thermal spike energies. (b) Number of
		re-solved Xe atoms vs. time. The vertical dotted lines represent 450
		ps.
	}
	\label{fig:cooldown}
\end{figure}

The thermal spike simulations were run for at least 450 ps, which is sufficient
for all systems to cool down to under 800 K. Figure \ref{fig:cooldown} shows
the system temperature and the number of re-solved Xe atoms over time in regard
to a few 40 \r{A} radius bubble simulations featuring various thermal spike
energies. As seen in the figure, the number of re-solved atoms stabilizes well
before 450 ps. For further verification, longer simulations were also conducted
to allow the systems to cool down to under 500 K. The re-solution behavior
remains unchanged throughout the cooldown from 800 to 500 K. Thus, a minimum
simulation time of 450 ps was chosen so as to optimize the use of computing
resources. Cluster analysis was performed on the Xe atoms in the system, using
OVITO \cite{ovito}. The cutoff distance among clusters was chosen as 10 \r{A}.
Initially, all the Xe atoms in the system formed a single cluster, representing
the original Xe bubble. However, after introducing the PKAs or thermal spikes,
multiple Xe clusters could be identified due to re-solution of Xe atoms. The
cluster containing the most Xe atoms was considered the original Xe gas bubble,
and all Xe atoms in the other clusters were classified as re-solved atoms. When
only a single atom remained in the original bubble, that gas bubble was
considered completely re-solved.


\FloatBarrier
\section{Results}

\FloatBarrier
\subsection{PKA simulations}

Limited re-solution was observed in the PKA simulations. Figure \ref{fig:pka}
showcases snapshots from one such simulation, with the black spheres indicating
Xe atoms and the red and blue spheres representing U and Mo atoms,
respectively. The energy introduced into the system by the PKA propagates as a
shock wave toward the supercell boundary. This shock wave generates numerous
point defects, most of which eventually recombine. The Xe gas bubble undergoes
deformation immediately after initiating the PKA, but swiftly reverts to a
stable configuration. In the PKA simulations, only one re-solved Xe atom at
most is observed, with most simulations showing no re-solution at all. Thus,
homogeneous re-solution is considered negligible within the examined energy
range of the PKAs (up to 500 keV).

The homogeneous re-solution was expected to be a few orders of magnitude
smaller than the heterogeneous re-solution, as the collision cross section of a
PKA is substantially smaller than the interaction cross section between a
bubble and a thermal spike \cite{olander2006re}. Govers el al.
\cite{govers2012} performed PKA simulations for UO$_2$ and observed, at most,
three re-solved Xe atoms out of gas bubbles containing hundreds. Thus, they
deemed homogeneous re-solution too insignificant for calculating overall
re-solution rates. The PKA simulations performed in our work on $\gamma$U-10Mo
show similar results. Also, in the case of $\gamma$U-Mo, only 5\% of the energy
from a fission event is deposited ballistically \cite{beeler2021rad}. Given
that a limited number of re-solved atoms were observed in the simulations, and
only a small fraction of the energy from fission reactions go to ballistic
collisions, it is reasonable to assume that the homogeneous re-solution is
negligible.

\begin{figure}[ht]
	\centering
	\begin{subfigure}{0.45\textwidth}
		\centering
		\caption{}
		\includegraphics[width=\textwidth]{images/pka1.png}
	\end{subfigure}
	\begin{subfigure}{0.45\textwidth}
		\centering
		\caption{}
		\includegraphics[width=\textwidth]{images/pka2.png}
	\end{subfigure}

	\begin{subfigure}{0.45\textwidth}
		\centering
		\caption{}
		\includegraphics[width=\textwidth]{images/pka3.png}
	\end{subfigure}
	\begin{subfigure}{0.45\textwidth}
		\centering
		\caption{}
		\includegraphics[width=\textwidth]{images/pka4.png}
	\end{subfigure}
	\caption{
		Snapshots of a PKA (500 keV) simulation at (a) 0, (b) 1.5, (c) 44.5,
		and (d) 114.5 ps. The Xe atoms are shown in black, along with the U
		(red) and Mo (blue) atoms.
	}
	\label{fig:pka}
\end{figure}


\FloatBarrier
\subsection{Thermal spike simulations}

% On-centered thermal spike

In contrast, the thermal spike model demonstrated significant re-solution of Xe
atoms. Figure \ref{fig:spike} showcases several snapshots of an on-centered
thermal spike simulation. The thermal spike formed a cylindrical volume of
liquid that engulfed the entire Xe gas bubble, leading to its disintegration.
However, the region cooled down significantly after a few ps. Despite the fact
that many Xe atoms coalesced into a large gas bubble upon cooling, a
substantial number remained in the fuel matrix or formed other, smaller
bubbles. The thermal spikes also generated shock waves that propagated radially
to the boundary and created point defects in the system. The number of point
defects decreased with cooling through the sink. Since substantially greater
re-solution was observed in the thermal spike model than in the PKA model, we
only considered heterogeneous re-solution of Xe atoms in subsequent
calculations.

\begin{figure}[ht]
	\centering
	\begin{subfigure}{0.45\textwidth}
		\centering
		\caption{}
		\includegraphics[width=\textwidth]{images/spike1.png}
	\end{subfigure}
	\begin{subfigure}{0.45\textwidth}
		\centering
		\caption{}
		\includegraphics[width=\textwidth]{images/spike2.png}
	\end{subfigure}

	\begin{subfigure}{0.45\textwidth}
		\centering
		\caption{}
		\includegraphics[width=\textwidth]{images/spike3.png}
	\end{subfigure}
	\begin{subfigure}{0.45\textwidth}
		\centering
		\caption{}
		\includegraphics[width=\textwidth]{images/spike4.png}
	\end{subfigure}
	\caption{
		Snapshots of a thermal spike (30 keV/nm) simulation at (a) 0, (b) 1.5,
		(c) 44.5, and (d) 114.5 ps. The Xe atoms are shown in black, along with
		the U (red) and Mo (blue) atoms.
	}
	\label{fig:spike}
\end{figure}

Re-solution data obtained from the on-centered thermal spike simulations for
different bubble sizes are plotted in Figure \ref{fig:frac}(a) as a fraction of
re-solved Xe atoms over the effective energy transferred to the lattice,
$S_{e,eff}$. Thermal spikes re-solve a greater fraction of Xe atoms in the
smaller bubbles than in the larger bubbles. Furthermore, the fraction of
re-solved Xe atoms seems to saturate with increasing deposited energy. Thus, an
exponentially saturating function was used to model the available data:
\begin{align}
	\chi_0 &= 1 - \exp[-\alpha S_{e,eff}]
\end{align}
where $\chi_0$ is the fraction of re-solved atoms due to an on-centered thermal
spike, and $\alpha$ is the saturation factor. The saturation factors for
different bubble sizes are plotted in Figure \ref{fig:frac}(b) against the
bubble radius. The graph suggests there is an inverse proportionality between
the saturation factor and the bubble size, meaning that larger bubbles are more
difficult to re-solve completely. To express this relationship, the saturation
factor was made an inverse power function of bubble radius, as follows:
\begin{align}
	\alpha &= \frac{5.1}{R_{bubble}^{2.2}}
\end{align}

\begin{figure}[ht]
	\centering
	\begin{subfigure}{0.69\textwidth}
		\centering
		\caption{}
		\includegraphics[height=6cm]{images/resolutionVsRadius.pdf}
	\end{subfigure}
	\begin{subfigure}{0.3\textwidth}
		\centering
		\caption{}
		\includegraphics[height=6cm]{images/saturationFactor.pdf}
	\end{subfigure}
	\caption{
		(a) Fraction of re-solved Xe atoms as a function of the energy
		deposited to the lattice. (b) Saturation factor as a function of bubble
		radius.
	}
	\label{fig:frac}
\end{figure}

% Off-centered thermal spike

The results discussed so far have pertained to on-centered thermal spikes.
Off-centered thermal spikes are expected to have diminishing effects on
re-solution, as a smaller portion of the Xe gas bubble would be enveloped by
the initial thermal spike region. Furthermore, all re-solution is anticipated
to cease when the initial cylindrical region of the thermal spike no longer
contacts the bubble surface. Let us define the off-centered distance $r$ as the
distance between the Xe gas bubble center and the cylindrical axis of the
thermal spike. The farthest distance at which the thermal spike contacts the
bubble can be denoted as $r_{c} \equiv R_{bubble} + R_{spike}$. Simulations
were conducted with off-centered distances over a range of 0 to $r_{c}$, at
intervals of 10 \r{A}.

To compare the results obtained from different bubble sizes, both the Xe
re-solution fraction and the off-centered distance were normalized. The
fraction of re-solved Xe atoms $\chi$ was normalized using the fraction of
re-solved Xe atoms from the on-centered thermal spike $\chi_0$, and the
off-centered distance $r$ was normalized using $r_{c}$, which varies depending
on bubble size. The normalization enabled comparison of re-solution data
stemming from different bubble sizes. Figure \ref{fig:off} depicts the
normalized fraction of re-solved atoms $\chi/\chi_0$ as a function of $r/r_c$,
highlighting how the effect of the relative off-centered distance of the
thermal spike remained similar for bubbles of different sizes. The data were
fitted to a logistic function, due to the plateaus observed at both ends of the
off-centered distance data. The fitted equation was as follows:
\begin{align}
	\label{eq:off}
	\frac{\chi}{\chi_0}
	&= \frac{1.058}{1 + \exp \big[8.168 \big(\frac{r}{r_c}\big) - 3.331 \big]}
\end{align}

\begin{figure}[ht]
	\centering
	\includegraphics[height=7cm]{images/offcentered.pdf}
	\caption{
		Re-solution caused by 15 keV/nm off-centered thermal spikes.
	}
	\label{fig:off}
\end{figure}

% Bubble pressure

To understand the effect of bubble pressure, bubbles with various Xe/vacancy
ratios were simulated. Figure \ref{fig:pres}(a) shows the number of re-solved
Xe atoms within a 25 \r{A} radius gas bubble as a function of Xe/vacancy ratio.
The number of re-solved atoms is apparently invariant with respect to
Xe/vacancy ratio, which is positively correlated with bubble pressure. Similar
trends were observed in bubbles with radii of 15 \r{A} and 35 \r{A}.
Consequently, the number of re-solved Xe atoms from identical-radius bubbles
with different pressures were averaged. The results are presented in Figure
\ref{fig:pres}(b). The number of re-solved Xe atoms also appears consistent
across bubbles of various sizes. Thus, the number of re-solved Xe atoms does
not seem to depend on either bubble pressure or size for bubbles having radii
of at least 15 \r{A}. Only the thermal spike energy seems to have a
relationship with the number of re-solved atoms, as is also apparent in Figure
\ref{fig:frac}(a). One possible explanation for this behavior is that thermal
spikes create low-density regions around the Xe gas bubble, and these regions
facilitate the separation of Xe atoms from the bubble. Since the thermal spike
energy dictates the volume of these low-density regions, it also affects the
total number of Xe atoms that can remain in a low-energy configuration outside
the original bubble.

\begin{figure}[ht]
	\centering
	\begin{subfigure}{0.49\textwidth}
		\centering
		\caption{}
		\includegraphics[height=6cm]{images/xevac.pdf}
	\end{subfigure}
	\begin{subfigure}{0.49\textwidth}
		\centering
		\caption{}
		\includegraphics[height=6cm]{images/r2dep.pdf}
	\end{subfigure}
	\caption{
		(a) Number of re-solved Xe atoms from 25 \r{A} radius bubbles as a
		function of Xe/vacancy ratio. (b) Number of re-solved Xe atoms against
		bubble radius. Data from identical-radius bubbles with different
		Xe/vacancy ratios were averaged.
	}
	\label{fig:pres}
\end{figure}


\FloatBarrier
\subsection{Re-solution rate}

The re-solution rate can be defined as the probability of a Xe atom in a gas
bubble getting re-solved into the fuel matrix, per unit time. The heterogeneous
re-solution rate can thus be calculated based on the fraction of re-solved
atoms in a bubble (due to a thermal spike) and the number of thermal spikes
that occur per second. Both bubble size and pressure are considered when
calculating the bubble re-solution rate. As stated in the previous subsection,
the number of re-solved Xe atoms remains constant regardless of Xe/vacancy
ratio. Therefore, the fraction of re-solved Xe atoms (the number of re-solved
Xe atoms divided by the total number of Xe atoms in the system) is inversely
proportional to the Xe/vacancy ratio. This observation can be used to decouple
the re-solution calculation as follows:
\begin{align}
	b_{het}(R_{bubble}, \dot{F}, \phi)
		&= f(R_{bubble}, \dot{F}) \cdot g(\phi)
\end{align}
where $b_{het}$ is the heterogeneous re-solution rate, $\dot{F}$ is the fission
rate, and $\phi$ is the Xe/vacancy ratio in the bubble. The Xe/vacancy ratio,
$\phi$, is closely related to the bubble pressure, $P$. (This relation is
discussed later.) The function $f(R_{bubble}, \dot{F})$ calculates the
re-solution rate of Xe atoms from gas bubbles with $\phi=0.2$, whereas the
function $g(\phi)$ corrects the re-solution rate for bubbles deviating from the
nominal $\phi$ value of 0.2. We also define $f(R_{bubble}, \dot{F})$ as the
nominal heterogeneous re-solution rate. The subsequent formulation of
$f(R_{bubble}, \dot{F})$ draws significant inspiration from the work of
Setyawan et al. \cite{setyawan2018}.

\begin{figure}[ht]
	\centering
	\includegraphics[height=7cm]{images/coordSystem.pdf}
	\caption{
		Cylindrical coordinate system for calculating the heterogeneous
		re-solution rate.
	}
	\label{fig:coord}
\end{figure}

To represent the overall re-solution behavior, the contributions from all the
fission products---originating at different distances from a bubble and
oriented toward a random direction---need to be summed up by means of a volume
integral. Consider a cylindrical coordinate system in which the gas bubble is
at the origin (Figure \ref{fig:coord}). For ease of calculation, fission
product origins can be rotated around the coordinate system origin so that all
fission tracks (as well as the thermal spikes they create) point in the same
direction. Given that the fission product generation is uniform and isotropic
in the material, the aforementioned rotational transformation will lead to a
uniform distribution of unidirectional fission products. The axial coordinate
$x$ is then defined as parallel to the fission tracks, and the radial
coordinate $r$ as perpendicular to the tracks. The behavior with respect to the
azimuth is constant because the fission products are unidirectional after the
transformation. If we denote the fission rate as $\dot{F}$, the number of
fission events per second in an infinitesimal volume $dV = 2 \pi dr dx$ would
be $\dot{F} dV$. Also, if a fission track produced by species $i$ interacts
with a Xe gas bubble, it re-solves $\chi_i$ fraction of the Xe atoms. Each
fission product species $i$ from fission events thus contributes $\chi_i
\dot{F} dV$ to the total re-solution of the bubble. The nominal heterogeneous
re-solution rate can then be expressed as:
\begin{align}
	f(R_{bubble}, \dot{F})
	&= \sum_{i=1}^2 \int_V \chi_i \dot{F} dV \\
	&= \sum_{i=1}^2 \int_{x=0}^{\infty} \int_{r=0}^{\infty}
		\chi_i \dot{F} 2 \pi r dr dx \\
	&= \dot{F} \sum_{i=1}^2 \int_{r=0}^{\infty}
		\bigg( \frac{\chi_i}{\chi_{0,i}} \bigg)
		2 \pi r dr \int_{x=0}^{\infty} \chi_{0,i} dx
\end{align}
where it is assumed that a single fission reaction creates two fission
products, ignoring ternary fission reactions in the process. Furthermore, the
double integral is decoupled into two 1-D integrals, since
$\Big(\frac{\chi_i}{\chi_{0,i}}\Big)$ is only dependent on $r$, and
$\chi_{0,i}$ is dependent on $x$.

We can now compute the $r$ integral by using Equation \ref{eq:off}. The upper
limit of the integral can be changed to $r=r_c$, since any possibility of
re-solution is presumed nonexistent for $r > r_c$.
\begin{align}
	\int_{r=0}^{r_c} \bigg( \frac{\chi_i}{\chi_{0,i}} \bigg) 2 \pi r dr
	&= 2 \pi r_c^2 \int_{r/r_c=0}^{1}
		\frac{1.058}{1 + \exp \big[8.168 \big(\frac{r}{r_c}\big) - 3.331 \big]}
		\bigg(\frac{r}{r_c}\bigg) d\bigg(\frac{r}{r_c}\bigg) \\
	&\approx 0.225 \pi r_c^2
\end{align}

Thus, the nominal heterogeneous re-solution rate $f(R_{bubble}, \dot{F})$ can
be written as:
\begin{align}
	f(R_{bubble}, \dot{F})
	&= \dot{F} \sum_{i=1}^2 \int_{r=0}^{\infty}
		\bigg( \frac{\chi_i}{\chi_{0,i}} \bigg)
		2 \pi r dr \int_{x=0}^{\infty} \chi_{0,i} dx \\
	&= \dot{F} \sum_{i=1}^2 0.225 \pi r_c^2 \int_{x=0}^{\infty}
		[1 - \exp(-\alpha S_{e,eff,i})] dx \\
	\label{eq:dx}
	&= 0.225 \pi r_c^2 \dot{F} \sum_{i=1}^2 \int_{x=0}^{\infty}
		[1 - \exp(-5.1 \zeta S_{e,i} / R_{bubble}^{2.2})] dx 
\end{align}
where $S_{e,i}$ is the electronic stopping power of the fission product $i$,
and $\zeta = S_{e,eff} / S_{e}$ is the fraction of fission product energy
imparted to the lattice. In this work, $\zeta$ is assumed to be an unknown
parameter ranging somewhere between 0.55 and 0.95. Introducing this parameter
is necessary to account for the fraction of the fission product energy that is
not transferred to the thermal spike.

% Electronic stopping of fission products

We chose to evaluate the $S_e$ values for Xe-140 and Sr-94 in $\gamma$U-10Mo,
based on the fission reaction $_0^1n + _{92}^{235}U \rightarrow _{54}^{140}Xe +
_{38}^{94}Sr + Q$. Despite this being one of many fission reactions that may
occur, Xe-140 and Sr-94 can still be considered representative of heavy and
light fission products. This is because the two aforementioned species appear
close to the two peaks in a typical fission product yield graph
\cite{setyawan2018, mills1995}. Figure \ref{fig:elec} shows $S_{e,Xe}$ and
$S_{e,Sr}$ as a function of the distance traversed by the fission product.
These data were generated using the SRIM software. The following equations,
similar to the ones used in \cite{setyawan2018}, were utilized to represent the
$S_e$ data:
\begin{align}
	\label{eq:xe}
	S_{e,Xe} &= 21.3 \exp(-0.239 x^{1.78})
		+ 5.23 \exp(-4.67 \times 10^{-8} x^{11}) \\
	\label{eq:sr}
	S_{e,Sr} &= 19.7 \exp(-0.00273 x^{3.71})
		+ 6.8 \exp(-0.424 x^{1.45})
\end{align}

\begin{figure}[ht]
	\centering
	\includegraphics[height=7cm]{images/elec_stopping.pdf}
	\caption{
		Total electronic stopping power ($S_e$) of Xe-140 and Sr-94 in
		$\gamma$U-10Mo, as calculated by the SRIM software as a function of
		distance traversed by the fission product from the location of the
		fission reaction.
	}
	\label{fig:elec}
\end{figure}

Finally, Equations \ref{eq:dx}, \ref{eq:xe}, and \ref{eq:sr} can be used to
calculate the re-solution rate for different values of $\zeta$ and bubble
radius $R_{bubble}$. Figure \ref{fig:res} displays the results for a fission
rate of $10^{14}$ fiss/cm$^3$/s since usual fission rates in the $\gamma$U-10Mo
fuel are of this order \cite{annualreport2021}. The re-solution rates were
computed using numerical integration, and the data can be extracted from the
associated graphs or by recalculating the integrals. However, an approximate
analytical function might better serve the higher-length-scale models. To that
end, we propose the following function for fitting the computed re-solution
rate:
\begin{align}
	\label{eq:param}
	f(R_{bubble}, \dot{F})
	&= \bigg[ \frac{a}{1 + (R_{bubble} / c)^d} \bigg]
		(10^{-14} \: \dot{F})
\end{align}
where $R_{bubble}$ is in \r{A}, $\dot{F}$ is in fissions per cubic centimeter
per second (fiss/cm$^3$/s), and $a$, $c$, and $d$ are adjustable parameters.
The parameter values for different $\zeta$ are listed in Table 1. For all
$\zeta$ values, the parameter $d$ remains consistent at up to three decimal
places. To determine the re-solution rate of any arbitrary $\zeta$ of between
0.55 and 0.95, linear interpolation would be sufficient.

\begin{figure}[ht]
	\centering
	\includegraphics[height=7cm]{images/resRate.pdf}
	\caption{
		Xe gas bubble re-solution rate in $\gamma$U-10Mo, as a function of
		bubble radius, at a fission rate of $10^{14}$ fiss/cm$^3$/s and for a
		Xe/vacancy ratio of 0.2.
	}
	\label{fig:res}
\end{figure}

\begin{table}[ht]
\centering
\caption{Equation \ref{eq:param} parameter fits for different $\zeta$ values.}
\label{tab:param}
\begin{tabular}{llll}
\toprule
$\zeta$     & $a$        & $c$     & $d$      \\
\midrule
0.55        & 0.0167     & 2.717   & 1.225    \\
0.65        & 0.0163     & 3.184   & 1.225    \\
0.75        & 0.0161     & 3.635   & 1.225    \\
0.85        & 0.0159     & 4.073   & 1.225    \\
0.95        & 0.0158     & 4.502   & 1.225    \\
\bottomrule
\end{tabular}
\end{table}

% Multiply the pressure factor

Now consider the effect of $\phi$ (and $P$) on re-solution. Since the number of
re-solved atoms is constant with respect to $\phi$ for a specific bubble size
(Figure \ref{fig:pres}(a)), bubbles with more Xe atoms (and thus more pressure)
will have a lower fraction of re-solved atoms. This leads to the following
simple expression:
\begin{align}
	g(\phi) &= \frac{0.2}{\phi}
\end{align}

The Xe/vacancy ratio, $\phi$, can be used to obtain the molar volume of Xe,
$v$. At 400 K, the equilibrium volume of $\gamma$U-Mo is $19.7$ \r{A}$^3$/atom.
Thus, $v$ can be expressed as:
\begin{align}
	v
	&= \frac{N_A \cdot 19.7}{\phi} \text{ \r{A}}^3/\text{mol} \\
	\label{eq:v}
	&= \frac{11.86}{\phi} \text{ cm}^3/\text{mol}
\end{align}
where $N_A$ is the Avogadro constant.

With the molar volume now known, the pressure, P, can be computed via the Xe
bubble equation of state (EOS), as provided by Beeler et al. \cite{beelerADP}.
The Virial EOS is expanded to the third order with respect to volume, and to
the second order with respect to temperature. The EOS is as follows:
\begin{align}
	\label{eq:eos}
	P &= \frac{RT}{v}
		\bigg( A + \frac{B}{v} + \frac{C}{v^2} + \frac{D}{v^3} \bigg)
\end{align}
where $R$ is the gas constant ($8.3145$ J/mol-K), and $A=1$, $B=151.12 \text{
cm}^3/\text{mol}$, $C=2976 \text{ cm}^6/\text{mol}^2$, and $D=705527 \text{
cm}^9/\text{mol}^3$ at temperature $T=400$ K. Combining Equations \ref{eq:v}
and \ref{eq:eos} provides a direct relationship between $\phi$ and $P$:
\begin{align}
	\label{eq:pres}
	P &= (280.4 \: \phi + 3573 \: \phi^2 + 5933 \: \phi^3
		+ 118600 \: \phi^4) \: \text{ MPa} \\
	  &\approx (5894.417 \: \phi^2 + 123341.14 \: \phi^4) \: \text{ MPa},
		\quad 0.05 \leq \phi \leq 0.50
\end{align}
where $P$ is approximated as a quadratic function of $\phi^2$ (by assuming an
equation of the form $P = c_1 \phi^2 + c_2 \phi^4$ and fitting it to a set of
data generated using Equation \ref{eq:pres}) so as to make algebraic
manipulation easier. Consequently, $g(\phi)$ can be expressed as follows:
\begin{align}
	g(\phi)
	= \frac{0.2}{\phi}
	= \frac{k}{(l+\sqrt{l^2+mP})^{1/2}}
	\coloneq h(P)
\end{align}
where $P$ is in MPa, $k=99.334$ (MPa)$^{1/2}$, $l=-5894.417$ MPa, and
$m=493364.56$ MPa. Both $g(\phi)$ and $h(P)$ are unitless.

Now that we have two equivalent functions of $\phi$ and $P$, the heterogeneous
re-solution rate takes the following forms:
\begin{align}
	b_{het}(R_{bubble}, \dot{F}, \phi)
		&= f(R_{bubble}, \dot{F}) \cdot g(\phi)
		= f(R_{bubble}, \dot{F}) \cdot h(P) \\
		&= \bigg[ \frac{a}{1+(R_{bubble}/c)^d} 10^{-14} \dot{F} \bigg]
			\bigg( \frac{0.2}{\phi} \bigg) \label{eq:g} \\
		&= \bigg[ \frac{a}{1+(R_{bubble}/c)^d} 10^{-14} \dot{F} \bigg]
			\bigg[ \frac{k}{(l + \sqrt{m+nP})^{1/2}} \bigg] \label{eq:h}
\end{align}
Equations \ref{eq:g} and \ref{eq:h} entirely define the model developed in this
work to predict the Xe gas bubble re-solution rate. Figure \ref{fig:3d} depicts
the re-solution rate as a function of both bubble radius and Xe/vacancy ratio
calculated using equation \ref{eq:g}. It is evident from the figure that the
re-solution rate decreases with both these variables.

\begin{figure}[ht]
	\centering
	\includegraphics[height=7cm]{images/3d.pdf}
	\caption{
		Xe gas bubble re-solution rate in $\gamma$U-10Mo as a function of
		bubble radius and Xe/vacancy ratio at a fission rate of $10^{14}$
		fiss/cm$^3$/s for $\zeta=0.75$.
	}
	\label{fig:3d}
\end{figure}


\FloatBarrier
\section{Discussion}

Re-solution rate is a crucial parameter for calculating fission gas swelling in
$\gamma$U-10Mo. The mesoscale software program DART incorporates this parameter
for such calculations \cite{ye2023}. The currently used re-solution rate
$b_{dart}$ in DART is defined as follows:
\begin{align}
	b_{dart} &= b_0 \cdot \dot{F} \cdot G \\
	b_0 &= R_{spike}^2 \cdot \mu_{ff} \\
	G &=
	\begin{cases}
		1 & ,R_{bubble} \leq \lambda \\
		1 - (\frac{R_{bubble}-R_{resol}}{R_{bubble}})^3
		  & ,R_{bubble} > \lambda
	\end{cases}
\end{align}
where $b_0$ is the bubble destruction probability, $\dot{F}$ is the fission
rate, and $G$ is a piecewise function representing different re-solution modes
for small and large gas bubbles. The parameter $b_0$ can be estimated based on
the interaction volume of a thermal spike with bubbles, per the formula $b_o =
R_{spike}^2 \times \mu_{ff}$, where $R_{spike}$ is the radius of a thermal
spike and $\mu_{ff}$ is the recoil length of fission fragments. In the
piecewise function G, $R_{bubble}$ is the bubble radius, $\lambda$ is the
gas-atom knock out distance, and $R_{resol}$ is the thickness of the annulus
within which all gas-atoms are knocked out. The parameters $b_0$, $\lambda$,
and $R_{resol}$ are considered adjustable, and the optimized values are $b_0 =
2 \times 10^{-18}$ cm$^3$, $\lambda = 5 \times 10^{-7}$ cm, and $R_{resol} = 3
\times 10^{-9}$ cm. Since the parameters $\lambda$ and $R_{resol}$ are not
coupled, the re-solution rate is discontinuous at a bubble radius $\lambda$.
The rationale behind the step function-like implementation of the re-solution
rate is to delineate intragranular bubbles from intergranular bubbles.
According to Ye et al. \cite{ye2023}, this is to account for the strong
trapping effects of grain boundaries. However, they also asserted that a
smoother transition between the two different bubble regimes is warranted.

The re-solution rate calculated in this work is based solely on the simulation
of intragranular bubbles. We assumed the trapping effect of the grain
boundaries to be insignificant over the timescale of the thermal spikes, thus
it was not considered. Also, accounting for trapping in the re-solution rate
contradicts the definition of re-solution. To model gas bubble evolution, the
re-solution rate and trapping rate should be implemented separately in
higher-length-scale programs, thereby affording greater freedom in modeling
complex behavior of gas bubbles in the fuel, under different contexts. For
example, the grain boundary diffusion coefficient of Xe in $\gamma$U-Mo can be
15 orders of magnitude higher than the intrinsic diffusion coefficient of Xe in
$\gamma$U-Mo at around 600 K \cite{hasan2024gb}. Thus, the Xe trapping rate for
intergranular bubbles can be set 15 times higher than that for intragranular
bubbles. Comparison between the re-solution rate computed in the present work
and the rate used in DART can elucidate their differences. Figure
\ref{fig:dart} shows the re-solution rates for bubble radii up to 100 \r{A}.
Unlike the DART model prediction, the re-solution rate computed in the present
work is a smooth decaying function of bubble radius. For very small bubbles,
the calculated re-solution rate is about two orders of magnitude higher than
the DART prediction. For bubbles with a radius of around 50 \r{A}, the
difference is less than one order of magnitude. This difference suddenly
increases to about three orders of magnitude for bubbles having greater than 50
\r{A} radii due to the discontinuous nature of the rate in DART. The $\zeta$
value does not impact this comparison, as the re-solution rate at $\zeta=0.95$
is, at most, double that at $\zeta=0.55$. The large bubble size regime
($R_{bubble} > 50$ \r{A}) is where DART adds a strong trapping effect to the
re-solution rate, and thus the rate does not represent a pure re-solution rate.

\begin{figure}[ht]
	\centering
	\includegraphics[height=7cm]{images/resRate_withDart.pdf}
	\caption{
		Comparison of the calculated nominal re-solution rate against the DART
		model prediction \cite{ye2023}.
	}
	\label{fig:dart}
\end{figure}


\FloatBarrier
\section{Conclusion}

This study utilized MD simulations to determine the re-solution rate of Xe gas
bubbles in $\gamma$U-10Mo fuel. Both the homogeneous and heterogeneous
re-solution mechanisms were simulated using the PKA method and thermal spike
method, respectively. However, homogeneous re-solution was found negligible in
describing the overall re-solution behavior. On- and off-centered thermal
spikes were both simulated to obtain the spatial correlation between
re-solution and thermal spike distance. The Xe/vacancy ratio in the bubble was
also varied to better understand the role of bubble pressure. Additionally, the
electronic stopping power of fission products (Xe and Sr) in $\gamma$U-10Mo was
simulated to determine the distance-dependent electronic stopping power. All
this information was then used to calculate the overall re-solution rate of Xe
gas bubbles in the fuel. The re-solution rate of intragranular Xe gas bubbles
under equilibrium pressure in $\gamma$U-10Mo was found to be on the order of
$10^{-4}$--$10^{-2}$ s$^{-1}$ for a fission rate of $10^{14}$
fissions/cm$^3$/s. An analytical form of re-solution rate---dependent on bubble
radius, bubble pressure, and fission rate---was also provided. The outcomes
generated by this work will inform higher-length-scale models of $\gamma$U-10Mo
with a physics-based description of the Xe gas bubble re-solution rate.


\bibliographystyle{unsrt}
\bibliography{ref.bib}

\end{document}

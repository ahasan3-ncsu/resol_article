\documentclass[12pt]{article}
\usepackage[margin=1in]{geometry}
\usepackage{graphicx, subcaption, booktabs}
\usepackage{amsmath, amssymb, siunitx}
\usepackage{placeins, hyperref}

\title{Xe gas bubble re-solution in U-10Mo nuclear fuel}
\author{ATM Jahid Hasan}

\begin{document}

\maketitle

\begin{abstract}
	The U.S. High-Performance Research Reactor (USHPRR) program
	aims to convert high enriched fuel in high-power research reactors
	to low enriched fuel.
	The choice for this conversion is a fuel type
	based on the U-10Mo alloy.
	The behavior of fission product gases, such as Xe, in the fuel
	needs to be well understood
	for evaluating the performance of the U-10Mo fuel.
	Xe gas bubbles play a pivotal role in this fuel type
	by trapping more fission product gas atoms and growing in size.
	Also, parts of the bubbles can disintegrate under irradiation
	by a process called re-solution.
	The interplay between the trapping rate and the re-solution rate
	governs the evolution of these gas bubbles.
	In this study,
	binary collision approximation (BCA) and molecular dynamics (MD) simulations
	are performed to quantify the re-solution rate of Xe gas bubbles
	in U-10Mo fuel.
	First, the energy loss of fission fragments (FFs)
	due to electronic and nuclear stopping is evaluated.
	Electronic stopping can initiate thermal spikes along FF tracks.
	To quantify re-solution due to thermal spikes,
	MD simulations coupled with the two-temperature model (TTM) are performed.
	It is found that thermal spikes cannot bring about re-solution in U-10Mo.
	For re-solution due to nuclear stopping,
	FF behavior in U-10Mo is simulated using BCA
	to get average FF incidence probability, energy and angle
	as a function of distance from the FF origin.
	Afterward, interactions between
	FFs of different energies and Xe gas bubbles in U-10Mo
	are assessed.
	Based on all the information gathered from BCA simulations,
	an overall Xe gas bubble re-solution rate $b$ is computed.
	$b/\dot{F}$
	ranges from \num{4.4e-26} m$^3$/fission for intergranular bubbles
	to \num{8.7e-25} m$^3$/fission for intragranular bubbles,
	where $\dot{F}$ represents the fission rate in the fuel.
	The effect of bubble pressure on re-solution rate is also evaluated.
\end{abstract}


\section{Introduction}

A U-10Mo alloy-based monolithic fuel design was identified
as the fuel type for converting U.S. High-Performance Research Reactors (HPRRs)
\cite{meyer2014} from high enriched fuel to low enriched fuel.
To understand the fuel's behavior under irradiation,
mesoscale and engineering-level fuel performance models require
knowledge of the fundamental mechanistic behavior of fission products
within the fuel to describe key phenomena,
such as swelling \cite{beeler2018gb, annualreport2021}.
Specifically, understanding the progression of Xe gas bubbles in the fuel
is crucial for optimizing reactor performance and safety.
These Xe gas bubbles act as a sink for individual Xe atoms,
trapping them and causing the bubbles to grow after absorption.
Under irradiation, the Xe atoms in the gas bubble are reintroduced
into the fuel matrix through fission-product-induced cascades
and thermal spikes---a process known as re-solution.
The relative rates of the re-solution affect the overall size and density
of the bubbles \cite{ye2023, olander2006re, parfitt2008},
in turn impacting bubble evolution and subsequent fuel swelling.
Re-solution of fission gas in nuclear fuels involves
two commonly accepted mechanisms:
homogeneous re-solution and heterogeneous re-solution.
In homogeneous re-solution, atoms from the gas bubbles are ejected individually
through collisions with fission products
or the recoil atoms that traverse the bubbles.
These atomic collision cascades are primarily
governed by the nuclear stopping power of the material.
In the heterogeneous model, a portion of gas bubbles are dissolved
by a passing fission fragment in the vicinity.
The driving mechanism is
the local heating of the material containing the gas bubbles,
through the electronic stopping of the fission fragments \cite{setyawan2018}.
The fact that both these mechanisms occur on a short timescale
makes it challenging to conduct experiments for determining re-solution rates
that contribute to the fission gas release models.
Thus, atomistic-scale modeling is necessary
for determining the re-solution rate
and for elucidating the fundamental mechanism
behind re-solution in U-10Mo.

In the literature up to this point,
atomistic simulations have been widely used
to evaluate the re-solution rate in various nuclear materials.
For instance, in 2008, Parfitt et al. \cite{parfitt2008}
used simulations of primary knock-on atoms (PKAs) in uranium dioxide (UO$_2$)
to assess the re-solution of helium gas bubbles.
In 2009, Schwen et al. \cite{schwen2009md} investigated
the homogeneous re-solution of Xe gas bubbles in UO$_2$,
using binary collision approximation (BCA) and molecular dynamics (MD).
The following year, Huang et al. \cite{huang2010md}
examined the impact of thermal spikes on Xe re-solution in UO$_2$.
In 2012, Govers et al. \cite{govers2012}
performed PKA and thermal spike simulations of Xe gas bubbles in UO$_2$
and proposed a mathematical model for the re-solution rate.
However, the most comprehensive work on Xe gas bubble re-solution in UO$_2$
was conducted  in 2018 by Setyawan et al. \cite{setyawan2018}.
They reconciled the inconsistencies found in the conclusions of previous works
on Xe bubble re-solution in UO$_2$
and evaluated the re-solution rate as a function of bubble radius.
Their findings suggest that heterogeneous re-solution of gas bubbles
is the dominant method of re-solution in UO$_2$.
In addition to UO$_2$,
the re-solution rate of fission gas bubbles was also evaluated
in uranium carbide (UC) by Matthews et al. \cite{matthews2015diss},
using BCA.
The thermal spike model was not employed in UC
because it was assumed that the local heating does not exceed
the melting temperature \cite{matthews2015diss, ronchi1986}.
In summary, BCA and MD simulations were both used
to determine the re-solution rate in nuclear fuels.

In MD simulations of homogeneous re-solution,
a regular lattice atom is typically endowed with high kinetic energy
to emulate a PKA.
The PKA then interacts ballistically with other atoms,
initiating a collision cascade near the gas bubble
and inducing disorder \cite{parfitt2008, govers2012}.
One alternative approach in MD is to simulate
only a portion of the cascade (i.e., a subcascade)
by imparting energy to a random gas atom within the bubble.
In doing so, the simulation avoids unnecessary cascade events
that may not significantly influence the re-solution process.
However, BCA must be utilized in this approach
to first obtain an energy spectrum of the gas atom PKAs \cite{schwen2009md}.
One challenge in using MD simulations to model homogeneous re-solution
is the channeling of PKAs or their recoils over long distances,
without any collisions \cite{jarrin2021}.
This can make collecting statistics
on the interactions between PKAs and gas bubble atoms computationally demanding,
especially when the PKA direction is random.
A potential solution is to direct the PKAs
toward a high index lattice direction \cite{stoller2000}.
Moreover, collision cascades end up in heat spikes due to nuclear stopping.
In that regard, the simulation of PKAs
also encompasses the heterogeneous re-solution.
To identify atoms that are re-solved ballistically, a threshold atomic speed,
above which it is improbable to find atoms in thermal equilibrium,
can be utilized \cite{parfitt2008}.
For MD simulations of heterogeneous re-solution due to swift heavy ions,
the thermal spike model is normally employed.
This model is useful for describing
the interaction between the fission fragments and the fuel.
These interactions occur primarily via
electronic stopping of the energetic particles
that initially raise the electronic subsystem temperature.
The energy deposited in the electronic subsystem can then transfer
to the lattice as thermal energy via electron-phonon coupling.
Finally, the energy is transferred among the atoms,
leading to a rapid increase in lattice temperature
within a cylindrical zone of typically a few nm in radius.
This increase is known as a thermal spike
\cite{wang1994, toulemonde2002, patra2019}.
In MD simulations, electronic interactions cannot be treated directly.
However, the final step described above can be emulated
by raising the temperature of atoms within a cylindrical region.

For qualification of U-10Mo fuel,
the ability to accurately predict the fission gas atom evolution
under various operational and transient conditions is crucial.
The Dispersion Analysis Research Tool (DART),
developed by Argonne National Laboratory \cite{ye2023},
is a mesoscale code that can calculate fission gas swelling
in U-10Mo under different operational situations.
One of the many parameters required to model swelling behavior
is the re-solution rate of fission gas bubbles.
DART employs a re-solution model
that includes a piecewise function to account for the bubble radius.
The parameters in this function are calibrated
by fitting the computed swelling value to experimental data,
meaning that we can only roughly estimate the re-solution rate.
A physics-based re-solution rate for fission gas bubbles would make
the swelling calculations of higher-length-scale models more rigorous.
In the present study, we utilize BCA and MD simulations
to investigate the re-solution of Xe gas bubbles in U-10Mo fuel,
considering both the homogeneous and heterogeneous re-solution mechanisms.


\section{Computational methods}

MD: LAMMPS, ADP

BCA: RustBCA, K-C


\section{Energy loss of fission fragments in U-10Mo}

Representative fission reaction:
\begin{align}
_0^1n + _{92}^{235}U
\rightarrow
_{39}^{97}Y + _{53}^{136}I + Q
\end{align}

Y and I have a fission product yield of about 0.12 (cite here).
Nuclear and electronic stopping powers of these fission fragments in U-10Mo
are shown in Figure \ref{fig:stopping}.

\begin{figure}[!ht]
\begin{subfigure}{0.49\textwidth}
	\centering
	\includegraphics[width=8cm]{images/Y_stopping.pdf}
\end{subfigure}
\begin{subfigure}{0.49\textwidth}
	\centering
	\includegraphics[width=8cm]{images/I_stopping.pdf}
\end{subfigure}
\caption{
	Nuclear and electronic stopping power
	of (a) $_{39}^{97}Y$ and (b) $_{53}^{136}I$.
}
\label{fig:stopping}
\end{figure}


\FloatBarrier
\section{Re-solution due to electronic stopping}

MD simulations incorporating TTM have been performed
to quantify re-solution due to electonic stopping.
No re-solution has been observed even at 30 keV/nm.
This is consistent with the findings of Kolotova et al.


\section{Re-solution due to nuclear stopping}

\subsection{Model for re-solution calculation}

Rotated fission events \ref{fig:rotation}

\begin{figure}[ht]
	\centering
	\includegraphics[width=14cm]{images/rotation.pdf}
	\caption{
		Rotation of fission event positions and velocities around the origin
		such that all velocities point to the $-x$ direction.
		Red dots represent positions and gray arrows represent velocities.
	}
	\label{fig:rotation}
\end{figure}

Volume element \ref{fig:volel}

\begin{figure}[ht]
	\centering
	\includegraphics[width=6cm]{oldimg/coordSystem.pdf}
	\caption{
		placeholder for new illustration
	}
	\label{fig:volel}
\end{figure}

\begin{align}
	dV &= 2 \pi w dw dx \\
	b &= \sum_{k = Y, I} \int_V \xi_k \dot{F} dV \\
	b &= \dot{F} \sum_{k = Y, I} \int_V \xi_k dV
		= \dot{F} \left( \int_V \xi_Y dV + \int_V \xi_I dV \right) \\
	b &= \dot{F} \sum_{k = Y, I} \int_x \int_w \xi_k(x, w) 2 \pi w dw dx
\end{align}

\subsection{Master simulations}

% simulation visuals using vispy here

Master simulations \ref{fig:master}

% the colorbars need labels
\begin{figure}[ht]
	\centering
	\begin{subfigure}{0.49\textwidth}
		\centering
		\caption{}
		\includegraphics[width=8cm]{images/Y_p.pdf}
	\end{subfigure}
	\begin{subfigure}{0.49\textwidth}
		\centering
		\caption{}
		\includegraphics[width=8cm]{images/I_p.pdf}
	\end{subfigure}
	\begin{subfigure}{0.49\textwidth}
		\centering
		\caption{}
		\includegraphics[width=8cm]{images/Y_e.pdf}
	\end{subfigure}
	\begin{subfigure}{0.49\textwidth}
		\centering
		\caption{}
		\includegraphics[width=8cm]{images/I_e.pdf}
	\end{subfigure}
	\begin{subfigure}{0.49\textwidth}
		\centering
		\caption{}
		\includegraphics[width=8cm]{images/Y_a.pdf}
	\end{subfigure}
	\begin{subfigure}{0.49\textwidth}
		\centering
		\caption{}
		\includegraphics[width=8cm]{images/I_a.pdf}
	\end{subfigure}
	\caption{
		Ion incidence probability per surface area of (a) Y and (b) I.
		Average incidence energy of (c) Y and (d) I.
		Average incidence angle of (e) Y and (f) I.
	}
	\label{fig:master}
\end{figure}

\FloatBarrier
\subsection{Fission fragment interactions with Xe gas bubbles}

van der Waals EOS \ref{fig:vdw}

\begin{align}
	n &= \bigg( B + \frac{kT}{p} \bigg)^{-1} \\
	p_{eq} &= \frac{2 \gamma}{R_b} \\
	n_{eq} &= \bigg( B + \frac{kT R_b}{2 \gamma} \bigg)^{-1}
\end{align}

\begin{figure}[ht]
	\centering
	\includegraphics[width=8cm]{images/n_vdw.pdf}
	\caption{
		Equilibrium Xe number density in gas bubbles
		calculated from the van der Waals equation of state.
	}
	\label{fig:vdw}
\end{figure}

\begin{align}
	D &= R_b + \delta \\
	y' &= \mathcal{I}_X (x', X, Y) \\
	\chi(E', \ell)
	   &= \mathcal{I}_{\mathcal{E}}
	   (E', \mathcal{E}, [\chi(E, \ell)]_{E \in \mathcal{E}}) \\
	\chi(E', \ell')
	   &= \mathcal{I}_{\mathcal{L}}
	   (\ell', \mathcal{L}, [\chi(E', \ell)]_{\ell \in \mathcal{L}})
\end{align}

% simulation visuals here using vispy

$\chi$ data \ref{fig:chi}

% use scientific notational for tick labels and make them consistent
\begin{figure}[ht]
	\centering
	\begin{subfigure}{0.49\textwidth}
		\centering
		\caption{}
		\includegraphics[width=8cm]{images/chi_2nm_Y.pdf}
	\end{subfigure}
	\begin{subfigure}{0.49\textwidth}
		\centering
		\caption{}
		\includegraphics[width=8cm]{images/chi_2nm_I.pdf}
	\end{subfigure}
	\begin{subfigure}{0.49\textwidth}
		\centering
		\caption{}
		\includegraphics[width=8cm]{images/chi_64nm_Y.pdf}
	\end{subfigure}
	\begin{subfigure}{0.49\textwidth}
		\centering
		\caption{}
		\includegraphics[width=8cm]{images/chi_64nm_I.pdf}
	\end{subfigure}
	\caption{
		$\chi(E, \ell)$ for bubble radius of 2 nm with incident (a) Y and (b) I.
		$\chi(E, \ell)$ for bubble radius of 64 nm with incident (c) Y and (d) I.
	}
	\label{fig:chi}
\end{figure}

\FloatBarrier
\subsection{Calculation of \texorpdfstring{$\xi$}{xi}}

\begin{align}
	\xi(x, w) &= \sum_{m \in S}
		p(r_m) \frac{A_m}{\cos \alpha(x, w)}
		\chi(E(r_m), ||r_m - r_c||) \\
	b / \dot{F} &= \sum \xi \Delta V \\
		&= \sum_{i,j \in V} \xi(c_{i,j})
		\pi (w_{i,j+1}^2 - w_{i,j}^2) (x_{i+1,j} - x_{i,j})
\end{align}

$\xi$ and $\xi \Delta V$ \ref{fig:xi}

\begin{figure}[ht]
	\centering
	\begin{subfigure}{0.49\textwidth}
		\centering
		\caption{}
		\includegraphics[width=8cm]{images/2nm_Y_xi.pdf}
	\end{subfigure}
	\begin{subfigure}{0.49\textwidth}
		\centering
		\caption{}
		\includegraphics[width=8cm]{images/2nm_Y_db.pdf}
	\end{subfigure}
	\begin{subfigure}{0.49\textwidth}
		\centering
		\caption{}
		\includegraphics[width=8cm]{images/64nm_Y_xi.pdf}
	\end{subfigure}
	\begin{subfigure}{0.49\textwidth}
		\centering
		\caption{}
		\includegraphics[width=8cm]{images/64nm_Y_db.pdf}
	\end{subfigure}
	\caption{
		(a) $\xi$ and (b) $\xi \Delta V$
		for bubble radius of 2 nm with incident Y.
		(c) $\xi$ and (d) $\xi \Delta V$
		for bubble radius of 64 nm with incident Y.
	}
	\label{fig:xi}
\end{figure}

\subsection{Homogeneous re-solution rate}

Re-solution \ref{fig:res}

\begin{figure}[ht]
	\centering
	\includegraphics[width=8cm]{images/bhom.pdf}
	\caption{
		Homogeneous re-solution rate as a function of bubble radius $R_b$
		at equilibrium Xe number density $n_{eq}$ in U-10Mo.
	}
	\label{fig:res}
\end{figure}

\subsection{Effect of bubble pressure}

Pressure effects \ref{fig:pres}

\begin{figure}[ht]
	\centering
	\includegraphics[width=8cm]{images/pressure.pdf}
	\caption{
		Effect of Xe number density on homogeneous re-solution rate.
	}
	\label{fig:pres}
\end{figure}

\subsection{Re-solution rate as a function of bubble size and pressure}


\section{Discussion}

Comparison with DART \ref{fig:comp}

% also need to compare with uo2, uc and uzr

\begin{figure}[ht]
	\centering
	\includegraphics[width=8cm]{images/comp.pdf}
	\caption{
		Comparison of the DART model prediction for re-solution
		with the results from this work.
	}
	\label{fig:comp}
\end{figure}


\FloatBarrier
\section{Conclusion}

This study utilizes BCA and MD simulations
to determine the re-solution rate of Xe gas bubbles in U-10Mo nuclear fuel.
Both the homogeneous and heterogeneous re-solution mechanisms are investigated.
Thermal spikes initiated by electronic stopping cannot cause re-solution.
Thus, the occurrence of heterogeneous re-solution in U-10Mo is not probable.
Homogeneous re-solution, which is brought about by nuclear stopping,
is found to be the only mechanism of re-solution in U-10Mo.
Therefore, the re-solution rate is calculated
by first obtaining FF behavior in the fuel,
then evaluating the FF interactions with Xe gas bubbles,
and finally putting all the information together in a physical model.
The computed re-solution rate $b$ for intergranular bubbles
is $\num{4.4e-26} \dot{F}$ s$^{-1}$
and for intragranular bubbles $\num{8.7e-25} \dot{F}$ s$^{-1}$,
where the unit of $\dot{F}$ is fission/m$^3$/s.
Furthermore, BCA simulations with varying Xe number density in the bubble
revealed that
the re-solution rate is inversely proportional to the Xe number density.
Thus, higher bubble pressure leads to a lower re-solution rate.
The results of this study will inform higher-length-scale models of U-10Mo
with a physics-based description of the Xe gas bubble re-solution rate.


\bibliographystyle{unsrt}
\bibliography{ref.bib}

\end{document}

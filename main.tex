\documentclass[12pt]{article}
\usepackage[margin=1in]{geometry}
\usepackage{authblk}
\usepackage{graphicx, subcaption, booktabs}
\usepackage{placeins, xspace, hyperref}
\usepackage{amsmath, amssymb, siunitx}
\usepackage[version=4]{mhchem}

\newcommand{\Y}{\ce{_{39}^{97}Y}\xspace}
\newcommand{\I}{\ce{_{53}^{136}I}\xspace}

\title{Xe gas bubble re-solution in U-10Mo nuclear fuel}

\author[1]{ATM Jahid Hasan}
\author[2]{Linu Malakkal}
\author[2]{Mathew Swisher}
\author[1,2]{Benjamin~Beeler}
\affil[1]{North Carolina State University}
\affil[2]{Idaho National Laboratory}

\begin{document}

\maketitle

% TODO: grammar check at the end

\begin{abstract}
	The U.S. High-Performance Research Reactor (USHPRR) program
	aims to convert high-power research reactors
	from higly enriched uranium (HEU) fuel to low-enriched uranium (LEU) fuel.
	The chosen candidate for this conversion is a monolithic fuel type
	based on the U-10Mo alloy.
	A critical aspect of U-10Mo fuel performance is the behavior and evolution
	of fission product gases, particularly Xe,
	which forms gas bubbles within the fuel matrix.
	These bubbles grow by trapping more fission gas atoms over time.
	Also, parts of the bubbles can disintegrate under irradiation
	through a process called ``re-solution''.
	The interplay between the trapping and re-solution rates
	governs bubble evolution, impacting fuel performance and safety.
	In this study,
	binary collision approximation (BCA) and molecular dynamics (MD) simulations
	were performed to quantify the Xe gas bubble re-solution rate in U-10Mo fuel.
	First, the energy loss of fission fragments (FFs)
	through electronic and nuclear stopping was evaluated.
	The effect of electronic stopping on re-solution was then analyzed
	using MD simulations coupled with the two-temperature model (TTM).
	The results show that thermal spikes generated by electronic stopping
	do not contribute to gas bubble re-solution in U-10Mo.
	To quantify re-solution due to nuclear stopping,
	BCA simulations of FFs in U-10Mo were performed
	to obtain average FF incidence probability, energy and angle
	as a function of distance from the FF origin.
	Subsequent simulations assessed FF--bubble interactions in U-10Mo
	for different FF energies and bubble radii.
	From these analyses,
	an overall re-solution rate $b$ was calculated
	at equilibrium bubble pressure per unit fission rate density,
	yielding values ranging
	from \num{4.4e-26} m$^3$/fission for the largest intergranular bubbles
	to \num{8.8e-25} m$^3$/fission for the smallest intragranular bubbles.
	The effect of bubble pressure on the re-solution rate was also evaluated,
	revealing an inverse relationship between the two.
\end{abstract}


\section{Introduction}

A U-10Mo alloy-based monolithic fuel design was selected
as the fuel type for converting U.S. High-Performance Research Reactors (HPRRs)
\cite{meyer2014} from highly enriched uranium (HEU) fuel
to low-enriched (LEU) fuel.
To reliably predict the fuel's behavior under irradiation,
mesoscale and engineering-level fuel performance models require
knowledge of the fundamental mechanistic behavior of fission products
within the fuel to describe key phenomena,
such as swelling \cite{beeler2018gb, annualreport2021}.
Specifically, understanding the progression of Xe gas bubbles in the fuel
is crucial for optimizing reactor performance and safety.
These gas bubbles act as a sink for diffusing Xe atoms in the fuel,
whose subsequent entrapment leads to progressive bubble growth.
Conversely, under irradiation, the Xe atoms in the gas bubbles are reintroduced
into the fuel matrix through collision cascades and thermal spikes
produced by fission fragments (FFs)---a process referred to as ``re-solution''.
The relative rates of Xe trapping and re-solution
dictate the size and density of the gas bubbles
\cite{ye2023, olander2006re, parfitt2008},
thereby influencing bubble evolution
and the overall swelling behavior of the fuel.

There are two widely accepted mechanisms of fission gas bubble re-solution:
homogeneous re-solution and heterogeneous re-solution \cite{olander2006re}.
In the homogeneous model proposed by Nelson \cite{nelson1969},
individual atoms are ejected from gas bubbles via collisions
with energetic FFs or recoil atoms traversing the bubbles.
These atomic collision cascades are primarily
governed by the nuclear stopping power of the material.
In contrast,
in the heterogeneous model proposed by Turnbull \cite{turnbull1971},
a portion of gas bubbles is dissolved by passing FFs in their vicinity.
The driving mechanism is
the local heating of the material containing the gas bubbles,
through the electronic stopping of the FFs \cite{setyawan2018}.
Irrespective of the mechanism,
the re-solution rate $b$ is defined as
the fraction of gas atoms returned to the solid solution
from bubbles per unit time,
or equivalently, the probability per unit time of
a single gas atom being ejected from a bubble back into the lattice
\cite{olander2006re, setyawan2018}.
Since both mechanisms occur on very short timescales,
experimentally determining the re-solution rates
required for fission gas release models becomes challenging.
As a result, atomic-scale simulations are essential
to elucidate the underlying mechanisms
and provide a quantitative description of the re-solution process in U-10Mo.

In the literature up to this point,
atomistic simulations have been widely used
to evaluate the re-solution rate in various nuclear materials.
For instance, in 2008, Parfitt et al. \cite{parfitt2008}
simulated primary knock-on atoms (PKAs) in uranium dioxide (UO$_2$)
using molecular dynamics (MD) to assess the re-solution of helium gas bubbles.
In 2009, Schwen et al. \cite{schwen2009md} investigated
the homogeneous re-solution of Xe gas bubbles in UO$_2$,
using binary collision approximation (BCA) and MD.
The following year, Huang et al. \cite{huang2010md}
examined the impact of thermal spikes on Xe re-solution in UO$_2$
using MD coupled with the two-temperature model (TTM).
In 2012, Govers et al. \cite{govers2012} performed MD simulations to study
how PKA and thermal spikes interact with Xe gas bubbles in UO$_2$
and proposed a mathematical model to describe the observed re-solution.
The most comprehensive work on Xe gas bubble re-solution in UO$_2$
was conducted in 2018 by Setyawan et al. \cite{setyawan2018}.
They reconciled the inconsistencies found in the conclusions of previous works
on Xe bubble re-solution in UO$_2$
and evaluated the re-solution rate as a function of bubble radius
with the help of extensive MD simulations.
Their results suggest that heterogeneous re-solution of gas bubbles
is the dominant method of re-solution in UO$_2$.
In addition to UO$_2$,
the re-solution rate of fission gas bubbles has also been evaluated
in other nuclear materials.
Matthews et al. \cite{matthews2015} evaluated re-solution
in uranium carbide (UC) in 2015,
while Mao et al. \cite{mao2025} studied re-solution
in uranium zirconium (U-Zr) alloys in 2025,
both using BCA simulations.
Unlike with UO$_2$,
thermal spikes are not expected to occur in UC and U-Zr systems,
due to their higher electronic conductivities and thermal diffusivities
\cite{ronchi1986, matthews2015, mao2025}.
Thus, only homogeneous re-solution has been studied in UC and U-Zr.
In summary, both BCA and MD simulations have been utilized
to determine re-solution rates in nuclear fuels.

MD simulations of homogeneous re-solution typically involve
assigning high kinetic energy to a regular lattice atom
in order to emulate a PKA.
The PKA then interacts ballistically with other atoms,
initiating a collision cascade near the gas bubble
and inducing localized atomic disorder \cite{parfitt2008, govers2012}.
One alternative MD approach focuses on simulating a subcascade
by imparting energy directly to a random gas atom within the bubble.
This method reduces computational costs
by avoiding unnecessary cascade events
that may not significantly influence the re-solution process.
However, to implement this approach accurately,
BCA simulations are required in order to first obtain
an energy spectrum of gas atom PKAs \cite{schwen2009md}.
One challenge in modeling homogeneous re-solution with MD
is the channeling of PKAs or their recoils over long distances,
without undergoing significant collisions \cite{jarrin2021}.
This phenomenon can make it computationally demanding
to gather statistics on interactions between PKAs and gas bubble atoms,
particularly when PKA directions are assigned randomly.
A potential solution is to direct the PKAs
along high-index lattice directions \cite{stoller2000},
thus increasing the collision probability.
Additionally,
collision cascades often produce heat spikes due to nuclear stopping,
which induce defect formation and facilitate damage-assisted re-solution.
Ballistically re-solved atoms can then be differentiated
by employing a threshold atomic speed
that is highly improbable occur in normal thermal equilibrium of the bubble
at the lattice temperature prior to the cascade initiation
\cite{parfitt2008}.

For heterogeneous re-solution,
MD simulations of thermal spikes are commonly employed.
FFs lose a significant portion of their energy via electronic stopping,
with the deposited energy initially raising the temperature
of the electronic subsystem.
The energy is subsequently transferred to the lattice as thermal energy
via electron-phonon coupling,
leading to a rapid rise of lattice temperature within a cylindrical zone
of typically a few nanometers in radius.
This localized heating, known as a thermal spike
\cite{wang1994, toulemonde2002, patra2019},
can induce re-solution if the spike intersects a gas bubble.
Although MD simulations cannot model electronic interactions directly,
the thermal spike process can be approximated by either
instantaneously increasing the temperature of atoms in a cylindrical region
\cite{govers2012, setyawan2018}
or by coupling MD with the TTM \cite{duffy2006, huang2010md}.

Accurate prediction of fission gas behavior
under various operational and transient conditions is critical
for the qualification of U-10Mo fuel.
To this end, the Dispersion Analysis Research Tool (DART),
a mesoscale code developed by Argonne National Laboratory \cite{ye2023},
has been equipped with the ability to calculate
fission gas swelling in U-10Mo under a range of operating conditions.
DART employs a re-solution model
that includes a piecewise function to account for
both intergranular and intragranular gas bubbles.
However, the parameters of this model are calibrated
by fitting the computed swelling values to experimental data,
providing only a rough estimation of the re-solution rate.
A physics-based model of the re-solution rate would enhance
the predictive capability of the higher-length-scale swelling models.
In this study, we utilized BCA and MD simulations
to investigate the re-solution of Xe gas bubbles in U-10Mo fuel,
addressing both the homogeneous and heterogeneous re-solution mechanisms.
This work intends to provide
a mechanistic understanding of the re-solution process in U-10Mo,
enabling more rigorous modeling of fission gas behavior and swelling
in the fuel under various reactor conditions.


\section{Computational methods}

In this section, we outline general computational methods used in this study.
For better readability,
specific simulation parameters and implementation details
are discussed along within the results.

\subsection{Binary collision approximation}

RustBCA, an open-source software for simulating ion-material interactions
\cite{drobny2021},
was used for all BCA simulations in this work.
RustBCA supports a variety of ion-material interactions,
including sputtering, implantation, and reflection.
Out of the box, RustBCA supports infinite homogeneous 0D targets,
finite-depth layered inhomogeneous 1D targets,
inhomogeneous 2D targets through a triangular mesh,
and homogeneous 3D triangular mesh geometry.
However, it does not support inhomogeneous 3D geometry,
which is necessary to emulate gas bubbles embedded in solids.
For this reason, we implemented a custom 3D geometry in RustBCA
termed \texttt{SPHEREINCUBOID}.
This configuration enables the user to specify
a spherical material inside a distinct cuboid material.
The source code for this implementation is available
at \url{https://github.com/ATM-Jahid/RustBCA}.
Like most BCA codes, RustBCA assumes an amorphous, static material,
neglecting crystal structures and accumulation of irradiation damage.
It also cannot take into account temperature effects.

The electronic stopping in the BCA simulations performed in this work
was described by the Biersack-Varelas interpolation \cite{varelas1970},
and the nuclear interactions were described by the universal Kr-C potential
\cite{moller1984, eckstein2013}.
An exponentially distributed mean-free-path model was used for gaseous regions
and a constant mean-free-path model for solid regions.
A threshold number density of $\num{1.5e28}$ m$^{-3}$ was used to distinguish
between gaseous and solid regions.

\subsection{Molecular dynamics}

The LAMMPS software package \cite{lammps} was utilized
to perform MD simulations,
using a U-Mo-Xe angular-dependent potential
\cite{smirnova2013, starikov2018, beelerADP}.
This angular-dependent potential can accurately describe
the body-centered cubic (bcc) phase of $\gamma$U-Mo alloys,
effectively reproducing their stable structure, elastic modulus,
room temperature density, and melting point.
To emulate the electronic stopping of the FFs,
all MD simulations were coupled with the TTM
using the ttm/mod command in LAMMPS \cite{norman2013, pisarev2014}.
This approach treats the electronic subsystem as a continuum
while describing the ionic subsystem through standard MD.
Energy transfer within the electronic subsystem
is governed by the heat diffusion equation,
which includes source terms to model the heat transfer
between the electronic and ionic subsystems:
\begin{align}
	C_e (T_e) \rho_e \frac{\partial T_e}{\partial t}
		&= \nabla (\kappa_e \nabla T_e) + g_p (T_e - T_a)
\end{align}
where $C_e$ represents the electronic specific heat
as a function of electronic temperature $T_e$,
$\rho_e$ is the electronic density,
$\kappa_e$ is the electronic thermal conductivity,
$g_p$ is the coupling constant for electron-ion interactions,
and $T_a$ is the ionic temperature \cite{duffy2006, rutherford2007}.
The electronic specific heat was expressed as $C_e = \gamma T_e$,
where $\gamma = \num{4e-9}$ eV/(K$^2$e).
The electronic density was set to $\rho_e = 625$ e/nm$^3$,
and the thermal conductivity was calculated using $\kappa_e = D_e \rho_e C_e$,
where $D_e = 100$ nm$^2$/ps is the thermal diffusion coefficient
\cite{li2017, kolotova2017}.
Even though these TTM parameter values have only been used
for pure U or U-5at.\%Mo,
it is assumed that reasonable accuracy can be obtained with these values
for the U-10Mo system as well.

Electronic pressure effects were included in the model
to account for the blast force acting on ions
due to the electronic pressure gradient \cite{chen2006, norman2013}.
Thus, the total force acting on an ion is:
\begin{align}
	\vec{F}_i
		&= -\frac{\partial U}{\partial \vec{r}_i}
		+ \vec{F}_{langevin} - \nabla P_e / n_{ion}
\end{align}
where $\vec{F}_{langevin}$ is the force from the Langevin thermostat
simulating electron-phonon coupling,
$U$ is the potential energy of the system,
$\nabla P_e / n_{ion}$ is the electron blast force,
and $n_{ion}$ is the ion concentration.
The electronic pressure was modeled as $P_e = 0.5 \rho_e C_e T_e$
\cite{norman2013, pisarev2014, kolotova2017}.


\section{Energy loss of fission fragments in U-10Mo}

To understand and quantify re-solution in U-10Mo,
FF behavior must first be investigated.
Fission of \ce{_{92}^{235}U} produces a wide range of isotopes.
To keep computational complexity manageable,
two isotopes, \Y and \I,
were selected as representative light and heavy FFs, respectively.
These isotopes are produced in the fuel via the following nuclear reaction:
\begin{align}
	\ce{
		_0^1n + _{92}^{235}U -> _{39}^{97}Y + _{53}^{136}I + Q
	}
	\label{eq:iso}
\end{align}
While \Y has an initial kinetic energy of approximately $101.3$ MeV,
\I starts with roughly $74.6$ MeV.
These two isotopes were chosen because they correspond
to the two peaks commonly observed in fission product yield distributions,
and they each have a yield of about $0.12$
\cite{setyawan2018, mills1995}.

\begin{figure}[!ht]
	\begin{subfigure}{0.49\textwidth}
		\centering
		\caption{}
		\includegraphics[width=8cm]{images/Y_stopping.pdf}
	\end{subfigure}
	\begin{subfigure}{0.49\textwidth}
		\centering
		\caption{}
		\includegraphics[width=8cm]{images/I_stopping.pdf}
	\end{subfigure}
	\caption{
		Nuclear and electronic stopping power of (a) \Y and (b) \I.
	}
	\label{fig:stopping}
\end{figure}

To evaluate the energy loss of these FFs in U-10Mo,
BCA simulations were conducted using RustBCA's \texttt{0D} geometry option.
In these simulations, U-10Mo was modeled as an infinite medium
extending from $0$ to $\infty$ in the $x$ direction,
and from $-\infty$ to $\infty $ in the $y$ and $z$ directions.
The FFs were injected at the origin
with an initial velocity oriented along the $x$ axis.
For each FF, $2,000$ independent ion-material simulations were performed.
From these simulations,
FF positions and velocities from each binary collision
were recorded, then processed and binned
to obtain electronic and nuclear stopping power profiles,
as illustrated in Figure \ref{fig:stopping}.
At the beginning of the FF trajectories,
the electronic stopping power is just shy of $20$ keV/nm for both FFs
and decreases almost linearly with distance.
In contrast, the nuclear stopping power remains low
for the majority of the FF trajectory,
peaking only at the very end of the path
as the FF reaches energies where nuclear collisions become more probable.
The nuclear energy loss for \Y accounts for
approximately $5\%$ of its total initial energy,
whereas for \I, the nuclear energy loss is around $10\%$.

Using insights derived from the stopping power profiles,
two complementary methods were employed
to investigate gas bubble re-solution in U-10Mo.
To analyze the effect of electronic stopping,
the two-temperature model (TTM) was utilized to simulate the energy transfer
between the electronic subsystem and the ionic subsystem
(discussed further in Section \ref{sec:elec}).
To capture the effect of nuclear stopping on re-solution,
the interactions among FFs, U-10Mo, and Xe gas bubbles were simulated
using the newly implemented \texttt{SPHEREINCUBOID} geometry in RustBCA,
as detailed in Section \ref{sec:nuke}.


\section{Re-solution due to electronic stopping}
\label{sec:elec}

To simulate thermal spikes, a simulation cell with dimensions
$120 \alpha_0 \times 120 \alpha_0 \times 50 \alpha_0$
($\alpha_0 = 0.343$ nm is the lattice parameter of U-10Mo)
was created with periodic boundary conditions in all directions.
A random distribution of U and Mo atoms in a bcc lattice
was first generated to achieve a Mo concentration of $22$ at.\%.
Using this configuration,
a spherical gas bubble with a radius of $2$ nm was created
by removing U and Mo atoms and depositing Xe atoms inside the void.
The Xe gas atoms were introduced at an Xe/vacancy ratio of $0.2$,
resulting in approximately 330 Xe atoms within the bubble.
The system was equilibrated at $400$ K and $0$ bar
by using an NPT ensemble for 10 ps, with a timestep size of 1 fs.
The Nos\'e-Hoover thermostat and barostat
regulated the temperature and pressure during equilibration.
The bubble size and Xe/vacancy ratio at the prescribed temperature and pressure
follow the results from the equation of state of Beeler et al.
\cite{beeler2020improved}.

To solve heat transfer related to the electronic subsystem,
electronic cells with dimensions
$2 \alpha_0 \times 2 \alpha_0 \times 50 \alpha_0$
were defined across the whole simulation cell,
with periodic boundary conditions applied in all directions.
A thermal spike was introduced
by initializing the electronic temperature profile in these cells according to:
\begin{align}
	T_e &= T_{init} + T_{spike} \exp\left( -r / R \right)
\end{align}
where $r$ is the radial distance from the thermal spike axis,
$R = 10 \alpha_0$, and $T_{init} = 400$ K.
The thermal spike axis was aligned with the shortest dimension of the supercell.
The values of $T_{spike}$ were chosen
according to the desired electronic stopping power.
The simulations were then performed using an NVE ensemble,
with a canonical sampling thermostat \cite{bussi2007}
applied at the edges of the simulation box to serve as a heat sink.
Only those edges that did not intersect the thermal spike axis
were included in the heat sink region.
A variable timestep size was implemented
such that the maximum displacement of any atom
between two successive timesteps was less than or equal to $0.001$ nm.
$100,000$ timesteps were performed for each simulation,
yielding $80$--$90$ ps of total simulation time.

\begin{figure}[!ht]
	\centering
	\begin{subfigure}{0.49\textwidth}
		\centering
		\caption{}
		\includegraphics[width=8cm]{images/ttm1.png}
	\end{subfigure}
	\begin{subfigure}{0.49\textwidth}
		\centering
		\caption{}
		\includegraphics[width=8cm]{images/ttm2.png}
	\end{subfigure}
	\begin{subfigure}{0.49\textwidth}
		\centering
		\caption{}
		\includegraphics[width=8cm]{images/ttm3.png}
	\end{subfigure}
	\begin{subfigure}{0.49\textwidth}
		\centering
		\caption{}
		\includegraphics[width=8cm]{images/ttm4.png}
	\end{subfigure}
	\caption{
		Snapshots of a thermal spike (30 keV/nm) simulation
		at (a) 0, (b) 2.4, (c) 10.1, and (d) 29.1 ps.
		The Xe atoms are shown in black, along with U (red) and Mo (blue) atoms.
	}
	\label{fig:ttm}
\end{figure}

Simulations were performed with three values of $T_{spike}$
($28,000$ K, $31,500$ K, and $34,500$ K),
corresponding to electronic stopping powers ranging from $20$ to $30$ keV/nm.
No re-solution events were observed in any of these simulations.
Figure \ref{fig:ttm} presents a few snapshots
of a $30$ keV/nm thermal spike simulation as visualized in OVITO \cite{ovito}.
The local ionic subsystem temperature rises rapidly for about $1.5$ ps
following the initiation of the thermal spike,
reaching its peak before beginning to cool.
After $30$ ps, only a few defects are visible within the system.
The local temperature gradually returns to the initial equilibrium temperature
over approximately $60$ ps.

Kolotova et al. reported threshold electronic stopping powers
for defect formation and melting in U-5at.\%Mo at various temperatures
\cite{kolotova2017}.
At $400$ K, the threshold stopping powers for defect formation and melting
were found to be approximately $22$ keV/nm and $26$ keV/nm, respectively.
Given that no gas bubble re-solution was observed
in the MD simulations of $30$ keV/nm thermal spikes,
peak stopping power of FFs in U-10Mo is approximately $20$ keV/nm
(Figure \ref{fig:stopping}),
and threshold stopping power for defect formation exceeds $20$ keV/nm,
it is highly unlikely that gas bubble re-solution can occur in U-10Mo
through the heterogeneous mechanism.
Since U-10Mo is a metallic system
and displays a relatively high thermal conductivity,
this behavior is expected.


\section{Re-solution due to nuclear stopping}
\label{sec:nuke}

\subsection{Model for re-solution calculation}

The re-solution rate can be defined as the probability
of a Xe atom escaping a gas bubble and entering the surrounding fuel matrix,
per unit time.
To capture the overall re-solution behavior in the material,
the contributions from all the FFs,
originating at various distances from the bubble
and oriented toward random directions,
must be integrated over the entire volume of interest.
Consider a 3D coordinate system
in which the gas bubble is located at the origin.
To simplify the integration,
FF points of origin can be rotated around the coordinate system origin
such that their initial velocities align in the same direction.
Since FF generation in the material is uniform and isotropic,
this rotational transformation results in
a uniform distribution of unidirectional FFs,
as shown in Figure \ref{fig:coord}a.
In this transformed system,
the FF velocities are oriented along the $-x$ direction,
and the radial coordinate $w := \sqrt{y^2+z^2}$
represents the impact parameter relative to the bubble center
(Figure \ref{fig:coord}b).
If the fission rate is denoted as $\dot{F}$,
the number of fission events per second
in an infinitesimal volume $dV = 2 \pi w \: dw  \: dx$ would be $\dot{F} dV$.
Also, we assume that when a FF isotope $k$ originating at $(x, w)$
interacts with a Xe gas bubble at the origin,
it re-solves $\xi_k \equiv \xi_k(x, w)$ fraction of the Xe atoms.
All the $k$ isotopes from fission events in $dV$ thus contribute
$\xi_k(x, w) \dot{F} dV$ to the total re-solution of the bubble.
The re-solution rate $b$ can then be expressed as:
\begin{align}
	b &= \sum_{k = Y, I} \int_V \xi_k(x, w) \dot{F} dV \label{eq:b} \\
	  &= \dot{F} \sum_{k = Y, I} \int_V \xi_k(x, w) dV
	   = \dot{F} \left( \int_V \xi_Y(x, w) dV + \int_V \xi_I(x, w) dV \right) \\
	  &= \dot{F} \sum_{k = Y, I} \int_{x=0}^{\infty} \int_{w=0}^{\infty}
		\xi_k(x, w) 2 \pi w dw dx
\end{align}
where it is assumed that all fission events produce \Y and \I,
as per Equation \ref{eq:iso}.

\begin{figure}[!ht]
	\begin{subfigure}{\textwidth}
		\centering
		\caption{}
		\includegraphics[width=12cm]{images/rotation.pdf}
	\end{subfigure}
	\begin{subfigure}{\textwidth}
		\centering
		\caption{}
		\includegraphics[width=8cm]{images/coord.pdf}
	\end{subfigure}
	\caption{
		(a) Rotation of fission event positions and velocities around the origin
		such that all velocities point to the $-x$ direction.
		Red dots represent positions and gray arrows represent velocities.
		(b) A coordinate system where a Xe gas bubble is at the origin,
		and fission fragments are all pointing toward the $-x$ direction.
	}
	\label{fig:coord}
\end{figure}

\subsection{Reference fission fragment simulations}

The most straightforward method to calculate $\xi_k(x, w)$ involves
simulating a FF and a Xe gas bubble for a specific $(x, w)$ value.
However, performing these simulations for all required $(x, w)$ values
is computationally expensive and often yields statistically unreliable results.
As the distance between the origin of a given FF and a Xe gas bubble increases,
the probability of interaction between them decreases significantly.
The interaction probability is also influenced by the size of the gas bubble:
smaller bubbles exhibit a lower probability of interaction.
Therefore, for certain bubble sizes and $(x, w)$ values,
even conducting hundreds of thousands of BCA simulations
may result in only a few interactions,
making most of these simulations an inefficient use of computational resources.

\begin{figure}[!ht]
	\centering
	\begin{subfigure}{0.49\textwidth}
		\centering
		\caption{}
		\includegraphics[height=5cm]{images/ff_track.png}
	\end{subfigure}
	\begin{subfigure}{0.49\textwidth}
		\centering
		\caption{}
		\includegraphics[height=5cm]{images/surf_grid.pdf}
	\end{subfigure}
	\caption{
		(a) Trajectories of 100 simulated \Y ions in U-10Mo.
		(b) Surface discretization of fission fragment information across volume.
	}
	\label{fig:fftrack}
\end{figure}

\begin{figure}[!ht]
	\centering
	\begin{subfigure}{0.49\textwidth}
		\centering
		\caption{}
		\includegraphics
			[width=8cm, trim={0.8cm 0 1.5cm 0.4cm}, clip]
			{images/Y_p.pdf}
	\end{subfigure}
	\begin{subfigure}{0.49\textwidth}
		\centering
		\caption{}
		\includegraphics
			[width=8cm, trim={0.8cm 0 1.5cm 0.4cm}, clip]
			{images/I_p.pdf}
	\end{subfigure}
	\begin{subfigure}{0.49\textwidth}
		\centering
		\caption{}
		\includegraphics
			[width=8cm, trim={0.8cm 0 1.5cm 0.4cm}, clip]
			{images/Y_e.pdf}
	\end{subfigure} \begin{subfigure}{0.49\textwidth}
		\centering
		\caption{}
		\includegraphics
			[width=8cm, trim={0.8cm 0 1.5cm 0.4cm}, clip]
			{images/I_e.pdf}
	\end{subfigure}
	\begin{subfigure}{0.49\textwidth}
		\centering
		\caption{}
		\includegraphics
			[width=8cm, trim={0.8cm 0 1.5cm 0.4cm}, clip]
			{images/Y_a.pdf}
	\end{subfigure}
	\begin{subfigure}{0.49\textwidth}
		\centering
		\caption{}
		\includegraphics
			[width=8cm, trim={0.8cm 0 1.5cm 0.4cm}, clip]
			{images/I_a.pdf}
	\end{subfigure}
	\caption{
		Ion incidence probability per unit surface area of (a) \Y and (b) \I.
		Average incidence energy of (c) \Y and (d) \I.
		Average incidence angle of (e) \Y and (f) \I.
	}
	\label{fig:ref}
\end{figure}

An alternative to the brute-force approach involved first analyzing
the behavior of FFs in the fuel matrix without the presence of any gas bubbles.
If the probability of a FF
passing through a specific point in the fuel with a given energy is known,
assessing re-solution behavior from local FF--bubble interactions
at that exact point then becomes straightforward.
Thus, FF simulations in U-10Mo were performed to obtain three key properties:
the probability of a FF passing through a particular point per unit area,
its average incidence energy,
and its average incidence angle.
In these reference simulations, the FF started from the origin
and was directed toward the $x$ axis within a U-10Mo matrix,
as displayed in Figure \ref{fig:fftrack}a.
It is important to note that this setup differs
from the configuration shown in Figure \ref{fig:coord}b,
where the Xe gas bubble is positioned at the origin.

To collect data from the simulations,
an annular surface discretization scheme,
as illustrated in Figure \ref{fig:fftrack}b, was used.
The discretized surface elements were associated with specific $(x, w)$ values,
with the spacing between successive surface elements
in the $x$ or $w$ direction being set to $\Delta x = \Delta w = 50$ nm.
This level of discretization was sufficient
to produce smooth profiles for the FF properties.
To verify the convergence of these FF profiles,
six points were selected for both \Y and \I.
Simulations were conducted in batches of $1000$ ions.
The simulations were terminated
when the relative changes in probability, energy, and angle
between two consecutive batches of simulations fell below $0.001$
at each of the six selected points.
The following $(x, w)$ coordinates, measured in $\mu$m,
were used for convergence:
$(3, 0)$, $(5, 0)$, $(7, 0)$,
$(4.5, 0.5)$, $(6.5, 0.5)$, and $(6, 1)$ for \Y,
and $(2, 0)$, $(3.5, 0)$, $(5, 0)$,
$(3, 0.5)$, $(4.5, 0.5)$, and $(4, 1)$ for \I.
The \Y profiles converged after $30,000$ simulations,
whereas the \I profiles required $40,000$ simulations to achieve convergence.

Figure \ref{fig:ref} presents the results
from the simulations and subsequent discretization.
The incidence probability per unit area
is displayed in Figures \ref{fig:ref}a and \ref{fig:ref}b.
The probability profiles broaden as $x$ increases,
producing a plume-like pattern.
Figures \ref{fig:ref}c and \ref{fig:ref}d
illustrate the incidence ion energies.
The circular pattern clearly demonstrates how the ions lose energy
as they travel farther from the origin,
with the energy loss as a function of distance being predominantly linear.
Lastly, Figures \ref{fig:ref}e and \ref{fig:ref}f display
the incidence angle of ions with respect to the $x$ axis.
As expected, ions closer to $w = 0$ have a low incidence angle,
while ions farther away show higher incidence angles.
In both the incidence energy and incidence angle plots,
a few discrete FF paths are visible in the top-left region.
These are highly unlikely occurrences,
as is evident from the probability figures.
The ion profiles also reveal the ranges of \Y and \I ions in U-10Mo
to be approximately $8.5$ and $6.5$ $\mu$m, respectively.
The observed ranges, shown in Figure \ref{fig:ref},
are consistent with the data presented in Figure \ref{fig:stopping}.

\subsection{Fission fragment interactions with Xe gas bubbles}

\begin{figure}[!ht]
	\centering
	\includegraphics[width=8cm]{images/n_vdw.pdf}
	\caption{
		Equilibrium Xe number density in gas bubbles
		calculated from the van der Waals equation of state.
	}
	\label{fig:vdw}
\end{figure}

With the reference FF properties available,
the next step involves simulating local FF--bubble interactions
to quantify the gas bubble re-solution rate.
For these simulations, equilibrium Xe number densities were used
for bubbles of all sizes.
The equilibrium number densities were calculated
using The van der Waals equation of state (EOS),
as described in \cite{olander1975}:
\begin{align}
	n &= \left( B + \frac{kT}{p} \right)^{-1} \\
	n_{eq} &= \left( B + \frac{kT}{p_{eq}} \right)^{-1}
	        = \left( B + \frac{kT R_b}{2 \gamma} \right)^{-1}
\end{align}
The equilibrium bubble pressure $p_{eq}$ utilized in the EOS
was derived from the Young-Laplace equation:
\begin{align}
	p_{eq} &= \frac{2 \gamma}{R_b}
\end{align}
where $\gamma = 1.55$ J/m$^2$ is the surface energy in U-10Mo \cite{beelerADP}.
The resulting equilibrium number densities
are plotted as a function of bubble radius in Figure \ref{fig:vdw}.
Notably, the van der Waals EOS predicts
a plateau in the equilibrium number density for smaller bubbles.

\begin{figure}[!ht]
	\centering
	\includegraphics[width=8cm]{images/recoil_dr.pdf}
	\caption{
		Recoil displacement against recoil energy
		in $5$ simulations of \Y in U-10Mo.
	}
	\label{fig:recoil}
\end{figure}

FF--bubble interactions should be simulated in a manner
that accounts for all relevant recoils.
To this end, five BCA simulations of \Y in U-10Mo
were analyzed to evaluate recoil behavior.
Figure \ref{fig:recoil} presents a scatter plot
of the recoil displacements vs. their energies.
The maximum energy transferred to the a recoil was about 1 MeV,
with the recoil displacement being around 100 nm.
While it is theoretically possible to transfer more than 1 MeV
to a recoil in a head-on collision,
such events are extremely rate.
Therefore, it is reasonable to assume that
recoils generated by FFs more than $\delta = 100$ nm away from a gas bubble
will not interact with that bubble.
In the FF--bubble BCA simulations, the FFs were positioned
at a distance $D = R_b + \delta$ from the bubble center.
Recoil trajectories from one such simulation are visualized
in Figure \ref{fig:ffbub} using VisPy \cite{vispy}.

% TODO: go through changes in captions as well
\begin{figure}[!ht]
	\centering
	\includegraphics[width=8cm]{images/ff_bubble.png}
	\caption{
		Recoil trajectories in a simulation of
		a 5 MeV \Y incident on a 64 nm radius Xe bubble.
		Red, blue, black, and cyan represent U, Mo, Xe, and \Y, respectively.
		Total Xe recoils: 3823.
		Xe outside the bubble: 155.
		Re-solved Xe atoms: 9.
	}
	\label{fig:ffbub}
\end{figure}

% NOTE: i'm here

To determine which Xe recoils are re-solved,
we consider Xe atoms that are displaced
at least $\lambda = 1$ nm away from the bubble surface as re-solved Xe atoms.
In reality, the bubble surface is not a static structure
due to the thermal fluctuations of the Xe atoms that make up the surface.
Thus, a Xe atom that is just outside the surface is highly likely
to end up in the bubble in a short time.
Therefore, there exists a finite annular region outside the bubble surface
that needs to be cleared by a Xe atom to be considered fully re-solved.
Our choice of $\lambda$ was informed by the literature
\cite{schwen2009md, govers2012, setyawan2018}.
Figure \ref{fig:xeres} depicts the displacements of Xe recoils
from a FF--bubble simulation.
From a graph of the recoil displacement as a function of energy,
Figure \ref{fig:xeres}a, it is also possible to find
a minimum threshold energy $E_{min}$ required for re-solution,
since $\lambda$ and $E_{min}$ are interconnected.
While some previous re-solution studies made a subjective choice for $\lambda$,
some studies based it on $E_{min}$ \cite{ronchi1986, matthews2015}.
From our simulations, we find $E_{min}$ to be about $25$ eV for $\lambda=1$ nm.
On the other hand, Figure \ref{fig:xeres}b depicts
how most Xe recoils originate close to the bubble surface
and end up just outside the surface.

\begin{figure}[!ht]
	\begin{subfigure}{0.49\textwidth}
		\centering
		\caption{}
		\includegraphics[width=8cm]{images/xe_dr.pdf}
	\end{subfigure}
	\begin{subfigure}{0.49\textwidth}
		\centering
		\caption{}
		\includegraphics[width=8cm]{images/xe_hist.pdf}
	\end{subfigure}
	\caption{
		% FIX: terms in the figure are not explained
		Xe recoils from the simulation of
		a 5 MeV \Y incident on a 64 nm radius Xe bubble.
		(a) Xe recoil final position against recoil energy.
		(b) Xe recoil final position histogram.
	}
	\label{fig:xeres}
\end{figure}

For FF--bubble simulations, both energy $E$ and off-centered distance $\ell$
were discretized.
Here, $\ell$ was measured as the minimum distance between
the bubble center and the initial FF velocity vector.
The energy discretization scheme depended on which FF is simulated,
and the $\ell$ discretization depended on the bubble radius.
The $\ell$ values were chosen in such a way that
the region around $R_b$ was sampled at a high spatial resolution.
Subsequently, $5,000$ BCA simulations were performed for each configuration.
From these simulations, the number of re-solved Xe atoms was determined,
and this number was then divided by the initial number of Xe atoms in the bubble
to obtain the re-solved bubble fraction $\chi$.
In other words, if a Xe gas bubble was at the origin $(x, w) = (0, 0)$,
$\chi(E', \ell')$ would be the re-solved bubble fraction
due to the interaction of the bubble with a FF
originating at $(x, w) = (-D, \ell')$ with an energy $E'$
and moving along the $+x$ axis.
The results for bubbles of two different sizes
are shown in Figure \ref{fig:chi}.
The $2$ nm and $64$ nm bubbles are representative
of intragranular and intergranular bubbles, respectively.
The error bars indicate $2\sigma$ deviations from the mean.

\begin{figure}[!ht]
	\centering
	\begin{subfigure}{0.49\textwidth}
		\centering
		\caption{}
		\includegraphics[width=8cm]{images/chi_2nm_Y.pdf}
	\end{subfigure}
	\begin{subfigure}{0.49\textwidth}
		\centering
		\caption{}
		\includegraphics[width=8cm]{images/chi_2nm_I.pdf}
	\end{subfigure}
	\begin{subfigure}{0.49\textwidth}
		\centering
		\caption{}
		\includegraphics[width=8cm]{images/chi_64nm_Y.pdf}
	\end{subfigure}
	\begin{subfigure}{0.49\textwidth}
		\centering
		\caption{}
		\includegraphics[width=8cm]{images/chi_64nm_I.pdf}
	\end{subfigure}
	\caption{
		$\chi(E, \ell)$ for bubble radius of 2 nm
		with incident (a) \Y and (b) \I.
		$\chi(E, \ell)$ for bubble radius of 64 nm
		with incident (c) \Y and (d) \I.
	}
	\label{fig:chi}
\end{figure}

Using the simulation results and interpolation,
it is now possible to obtain any reasonable $\chi(E', \ell')$ value.
If $\mathcal{I}$ is an interpolator that uses the mapping $X \rightarrow Y$,
we can use the notation $y' = \mathcal{I}_X (x', X, Y)$ to mean that
the interpolator returns $y'$ when $x'$ is provided as an input.
Any arbitrary $\chi(E', \ell')$ is then defined the following way:
\begin{align}
	\chi(E', \ell)
	   &= \mathcal{I}_{\mathcal{E}}
	   (E', \mathcal{E}, [\chi(E, \ell)]_{E \in \mathcal{E}}) \\
	\chi(E', \ell')
	   &= \mathcal{I}_{\mathcal{L}}
	   (\ell', \mathcal{L}, [\chi(E', \ell)]_{\ell \in \mathcal{L}})
\end{align}
where $\mathcal{E}$ and $\mathcal{L}$ are the sets of discrete energies
and off-centered distances that are used in the simulations,
and $E$ and $\ell$ are elements of those sets.
In this work,
$\mathcal{I}_{\mathcal{E}}$ is a Pchip interpolator \cite{fritsch1984},
and $\mathcal{I}_{\mathcal{L}}$ is a linear interpolator.

\subsection{Calculation of \texorpdfstring{$\xi$}{xi}}

\begin{figure}[!ht]
	\centering
	\includegraphics[width=13cm]{images/surf_mesh.pdf}
	\caption{
		Illustration of a fission fragment going through a surface mesh element
		in the vicinity of a Xe gas bubble.
	}
	\label{fig:surf_mesh}
\end{figure}

Equipped with $\chi$ values,
we proceed to calculate $\xi$, the re-solved bubble fraction.
Imagine a FF at the origin directed toward $x$, and a bubble at $(x, w)$.
At $(x, w)$, the FF has an average incidence angle $\alpha(x, w)$.
The incidence angle changes minimally with small changes in $x$ and $w$,
and thus it is reasonable to assume the incidence angle
in the vicinity of $(x, w)$ is simply $\alpha(x, w)$.
Next, we find a point $(x', w')$ such that
it is $D$ distance away from $(x, w)$
and the line connecting these two points has a slope $\tan \alpha(x, w)$.
A surface $S$ perpendicular to that connecting line can then be constructed,
where $S$ is square in shape and has a side length $2D = 2(R_b + \delta)$.
This surface can then be subdivided into small meshed elements
as displayed in Figure \ref{fig:surf_mesh}.
The FF has a certain probability of going through each mesh element,
as defined by the ion probability profiles % NOTE: beeler cut this off
(Figures \ref{fig:ref}a, \ref{fig:ref}b).
If the FF goes through a mesh element $m$
with probability per unit area $p(r_m)$ and incidence energy $E(r_m)$,
$\xi(x, w)$ can be calculated using the following equation:
\begin{align}
	\xi(x, w) &= \sum_{m \in S}
		p(r_m) \frac{A_m}{\cos \alpha(x, w)}
		\chi(E(r_m), ||r_m - r_c||)
	\label{eq:mesh}
\end{align}
where $r_m$ denotes the coordinate of the mesh element center
and $r_c$ denotes the center of the surface $S$.
The probability of the FF going through the mesh element $m$
is calculated by the product of $p(r_m)$
and the area of the mesh element $A_m$
projected to the direction perpendicular of $x$.
Notice that all FF trajectories that do not go through $S$ are ignored
because of the extremely low probability of such a FF or its recoils
reaching the bubble.

Even though the mesh elements displayed in Figure \ref{fig:surf_mesh}
are of constant size,
such a scheme is not efficient for the calculation of $\xi$.
This is due to the fact that $R_b$ can be a thousand times smaller than $D$.
Thus, we implemented an adaptive meshing scheme
that ensured a maximum mesh size of $R_b / 2$ for $\ell \in [0, 2 R_b]$
and $35$ nm for $\ell \in (2 R_b, D]$.
The upper bounds of the mesh size were chosen
in such a way that any finer resolution would not lead to
a relative change of $\xi$ by more than $0.001$.

\begin{figure}[!ht]
	\centering
	\begin{subfigure}{0.49\textwidth}
		\centering
		\caption{}
		\includegraphics
			[width=8cm, trim={0.8cm 0 1.5cm 0.7cm}, clip]
			{images/2nm_Y_xi.pdf}
	\end{subfigure}
	\begin{subfigure}{0.49\textwidth}
		\centering
		\caption{}
		\includegraphics
			[width=8cm, trim={0.8cm 0 1.4cm 0.7cm}, clip]
			{images/2nm_Y_db.pdf}
	\end{subfigure}
	\begin{subfigure}{0.49\textwidth}
		\centering
		\caption{}
		\includegraphics
			[width=8cm, trim={0.8cm 0 1.4cm 0.7cm}, clip]
			{images/64nm_Y_xi.pdf}
	\end{subfigure}
	\begin{subfigure}{0.49\textwidth}
		\centering
		\caption{}
		\includegraphics
			[width=8cm, trim={0.8cm 0 1.4cm 0.7cm}, clip]
			{images/64nm_Y_db.pdf}
	\end{subfigure}
	\caption{
		% FIX: caption
		(a) $\xi$ and (b) $\xi \Delta V$
		for a bubble radius of 2 nm and incident \Y.
		(c) $\xi$ and (d) $\xi \Delta V$
		for a bubble radius of 64 nm and incident \Y.
	}
	\label{fig:xi}
\end{figure}

$\xi$ was calculated for all combinations of FF isotopes and bubble radii.
$\xi_Y$ for $2$ nm and $64$ nm bubbles
are shown in Figures \ref{fig:xi}a and \ref{fig:xi}b.
These $\xi$ profiles can also be interpreted in another way
that makes the calculation of the overall re-solution rate straightforward.
If a bubble is at the origin and a FF is at $(x, w)$ pointing toward $-x$,
the $\xi$ profiles we would get are exactly the same.
A discretized version of Eq. \ref{eq:b} can then be formulated as follows:
\begin{align}
	b / \dot{F}
		&= \sum_{k=Y,I} \sum \xi_k \Delta V \\
		&= \sum_{k=Y,I} \sum_{i,j \in V} \xi_k(x_{i,j}, w_{i,j})
			\left[\pi (w_{i,j+1}^2 - w_{i,j}^2) (x_{i+1,j} - x_{i,j})\right]
	\label{eq:xitob}
\end{align}
where $i, j$ denote the indices of the discrete grid points
where $\xi$ is evaluated.
Since the $\xi$ profiles are already discretized,
this leads to an easy computation.
$\xi \Delta V$ profiles are also depicted in Figure \ref{fig:xi}
to show the significant effect of $\Delta V$ on re-solution.

\subsection{Homogeneous re-solution rate}

The total re-solution rate of a given bubble can be determined by
summing up all the values in $\xi_Y \Delta V$ profiles
due to both \Y and \I.
Figure \ref{fig:res} displays the re-solution rates
for bubbles with radius ranging from $1$ nm to $128$ nm.
The error bars here describe the uncertainty in $b$
only due to the uncertainty in $\chi$.
Since the ion profiles and $\xi$ were computed
using stringent convergence criteria,
it is reasonable to assume almost all of the uncertainty propagates from $\chi$.
Thus, the error bars for $b$ also roughly denote $2\sigma$ deviations.
We also observe vanishingly small deviations for larger bubbles.
This is simply because the probability of an interaction
between a FF and larger bubbles is higher.

\begin{figure}[!ht]
	\centering
	\includegraphics[width=8cm]{images/bhom.pdf}
	\caption{
		Homogeneous re-solution rate as a function of bubble radius $R_b$
		at equilibrium Xe number density $n_{eq}$ in U-10Mo.
	}
	\label{fig:res}
\end{figure}

Even though simple linear interpolations are sufficient
to obtain re-solution rates for arbitrary bubble radii,
an approximate analytical function may better serve higher-length-scale models.
To that end, we propose the following functional form:
\begin{align}
	b / \dot{F} = a R_b^k + c
\end{align}
where $a = \num{8.43e-25}, k = -0.926$ and $c = \num{3.46e-26}$.
$R_b$ is in nm, and $b / \dot{F}$ is in m$^3$/fsn.
The functional fit has a root mean squared error of $\num{3.17e-27}$ m$^3$/fsn
and a $R^2$ score of $0.99986$.

\subsection{Effect of bubble pressure}

\begin{figure}[!ht]
	\centering
	\includegraphics[width=8cm]{images/pressure.pdf}
	\caption{
		Effect of Xe number density on homogeneous re-solution rate.
	}
	\label{fig:pres}
\end{figure}

The effect of pressure on re-solution was investigated
by varying the Xe number density of bubbles of radii $8$ nm and $64$ nm.
FF--bubble interaction simulations were performed
for \Y and \I ions of energies 1 MeV and 20 MeV.
The limited scope of simulations investigating the effect of bubble pressure
is intended to show general trends in the re-solution behavior
of under- and over-pressurized bubbles.
$\chi$ values calculated from these simulations
are plotted in Figure \ref{fig:pres}.
An inversely proportional relationship between $\chi$ and $n$ is observed.
The equation $\chi / \chi_{eq} = n_{eq} / n$,
which is used to make the lines in Figure \ref{fig:pres},
fits the data with an R$^2$ score of 0.97.
To understand the origin of this inverse relationship,
the simulation data was analyzed further.
It was found that the number of re-solved Xe atoms is almost an invariant
with respect to the initial Xe number density in the bubble.
Since $\chi$ is the ratio of the number of re-solved atoms
and the total number of Xe atoms in the bubble,
an inverse relation between $\chi$ and $n$ naturally arise.
One possible explanation for the invariance of the number of re-solved atoms
is that the FF--bubble interactions primarily affecting re-solution behavior
occurs close to the bubble surface,
and the gas bubble surface area is independent of $n$.

% NOTE: beeler used "throughout" instead of "so far"
So far, the symbol $b$ has been used to denote the re-solution rate
at equilibrium Xe number density $n_{eq}$.
Here, $b_{eq}$ will be used to describe re-solution
at equilibrium Xe density,
whereas $b$ will be reserved for the general re-solution rate.
The re-solved bubble fraction $\chi$ is related to $\xi$
through Equation \ref{eq:mesh},
and $\xi$ is related to $b$ through Equation \ref{eq:xitob}.
These relations can thus be used
to relate the re-solution rate directly to the Xe number density:
\begin{align}
	b &\propto \xi \propto \chi \\
	b / b_{eq} &= n_{eq} / n \\
	b &= \left( a R_b^k + c \right) \left( \frac{n_{eq}}{n} \right) \dot{F}
	\label{eq:model}
\end{align}
where $R_b$ is in nm, $n$ is in m$^{-3}$,
$\dot{F}$ is in fsn m$^{-3}$ s$^{-1}$, and $b$ is in s$^{-1}$.
Equation \ref{eq:model} now describes the fission gas bubble re-solution rate
as a function of bubble size, bubble pressure, and fission rate.


\section{Discussion}

The re-solution rate computed in this work is compared
against the literature values for UO$_2$, UC, and U-10Mo
in Figure \ref{fig:comp}.
Setyawan et al. performed MD simulations of gas bubbles
ranging from $0.6$ nm in radius to $3$ nm \cite{setyawan2018}.
The slope of $\ln b$ against $\ln R_b$ in UO$_2$
is similar to that in our work,
albeit the re-solution rate in UO$_2$ is almost
one order of magnitude lower than in U-10Mo.
Matthews et al. conducted BCA simulations to evaluate the re-solution rate
for a wide range of bubble radii in UC \cite{matthews2015}.
The slope of $\ln b$ against $\ln R_b$ in UC is flatter than that in U-10Mo.
An intersection between the two rates is also observed,
showing the re-solution rate is higher in U-10Mo for smaller bubbles,
while it is higher in UC for larger bubbles.
Even with all the differences,
the re-solution rates in UO$_2$, UC, and U-10Mo (this work)
are very close to each other,
and the difference is often less than one order of magnitude.
We could not compare our results against U-Zr
because Mao et al. simulated all interactions at a fixed distance of $2$ $\mu$m
between the FFs and the bubbles \cite{mao2025}.
Thus, their data do not readily lead to an overall re-solution rate.

% TODO: change "fsn" to "fission"
\begin{figure}[!ht]
	\centering
	\includegraphics[width=8cm]{images/comp.pdf}
	\caption{
		% FIX: caption
		Comparison of the re-solution rates in UO$_2$, UC, and U-10Mo
		as a function of bubble radius $R_b$.
	}
	\label{fig:comp}
\end{figure}

Finally, we compare the computed re-solution rate
against the existing re-solution model of U-10Mo as implemented in DART.
The model is as follows:
\begin{align}
	b_{dart} &= b_0 \cdot \dot{F} \cdot G \\
	G &=
	\begin{cases}
		1 & ,R_{bubble} \leq \lambda \\
		1 - (\frac{R_{bubble}-R_{resol}}{R_{bubble}})^3
		  & ,R_{bubble} > \lambda
	\end{cases}
\end{align}
where $b_0$ is the bubble destruction probability,
and $G$ is a piecewise function representing
different re-solution modes for small and large gas bubbles.
In the piecewise function G, $R_{bubble}$ is the bubble radius,
$\lambda$ is the gas-atom knock-out distance,
and $R_{resol}$ is the thickness of the annulus
within which all gas-atoms are knocked out.
The parameters $b_0$, $\lambda$, and $R_{resol}$ are considered adjustable,
and the optimized values are $b_0 = \num{2e-18}$ cm$^3$,
$\lambda = \num{5e-7}$ cm, and $R_{resol} = \num{3e-9}$ cm.
Since the parameters $\lambda$ and $R_{resol}$ are not coupled,
the re-solution rate is discontinuous at a bubble radius $\lambda$.
According to Ye et al. \cite{ye2023},
this is to account for the strong trapping effects of grain boundaries
on intergranular bubbles.

The re-solution rate computed in this work is slightly lower than
the DART model prediction for intragranular bubbles
but significantly higher than the prediction for intergranular bubbles.
In fact, the difference for intergranular bubbles can be
as large as two orders of magnitude.
This large difference most likely stemmed from
the inclusion of trapping effects in the re-solution rate within DART.
However, the trapping effect of the grain boundaries
is insignificant in the time frame of collision cascades.
Therefore, the re-solution rate and trapping rate
should be implemented separately
in higher-length-scale models of gas bubble evolution,
thereby affording greater freedom in modeling complex behavior
of gas bubbles in the fuel under different contexts.
For example, the grain boundary diffusion coefficient of Xe in U-10Mo can be
15 orders of magnitude higher than the intrinsic diffusion coefficient
at around 600 K \cite{hasan2024gb}.
Thus, in a gas bubble evolution model,
the Xe trapping rate for intergranular bubbles can be set
15 times higher than that for intragranular bubbles,
and the re-solution rate can be set independently
using the analytical model prescribed in this work.
This way the re-solution rate will not be confounded with the trapping rate.


\section{Conclusions}

This study utilized BCA and MD simulations
to determine the re-solution rate of Xe gas bubbles in U-10Mo nuclear fuel.
Both the homogeneous and heterogeneous re-solution mechanisms were investigated.
Thermal spikes initiated by electronic stopping cannot cause re-solution.
Thus, the occurrence of heterogeneous re-solution in U-10Mo is not probable.
Homogeneous re-solution, which is brought about by nuclear stopping,
was found to be the only mechanism of re-solution in U-10Mo.
Therefore, the re-solution rate was calculated
by first profiling FF behavior in the fuel,
then evaluating the FF interactions with Xe gas bubbles,
and finally putting all the information together in a physical model.
The computed re-solution rate $b$ is
$\num{4.4e-26} \dot{F}$ s$^{-1}$ for the largest intergranular bubble
and $\num{8.8e-25} \dot{F}$ s$^{-1}$ for the smallest intragranular bubble,
where the unit of $\dot{F}$ is fission/m$^3$/s.
Furthermore, BCA simulations with varying Xe number density in the bubble
revealed that
the re-solution rate is inversely proportional to the Xe number density.
Thus, higher bubble pressure leads to a lower re-solution rate.
The results of this study will inform higher-length-scale models of U-10Mo
with a physics-based description of the Xe gas bubble re-solution rate.
% TODO: ^ maybe omit "rate"


\bibliographystyle{unsrt}
\bibliography{ref.bib}

\end{document}

\documentclass{elsarticle}
\usepackage[margin=1in]{geometry}
\usepackage{graphicx, xcolor, placeins}
\usepackage{amsmath, booktabs, subcaption}
\usepackage{lineno}

\begin{document}


\begin{frontmatter}

\title{Re-solution of Xe gas bubbles in the $\gamma$U-10Mo fuel}
\author[inl,ncsu]{ATM Jahid Hasan}
\author[inl]{Linu Malakkal}
\author[inl]{Mathew Swisher}
\author[inl,ncsu]{Benjamin Beeler}
\address[inl]{Idaho National Laboratory,
Idaho Falls, ID 83415, United States}
\address[ncsu]{North Carolina State University,
Raleigh, NC 27695, United States}

\begin{abstract}
	Molecular dynamics simulations were performed to quantify the re-solution
	rate of Xe gas bubble in $\gamma$U-10Mo fuel. Two models, the primary
	knock-on atom (PKA) and thermal spike models, were employed to evaluate
	homogeneous and heterogeneous re-solution, respectively. Notably,
	homogeneous re-solution was found to be negligible compared to the
	heterogeneous re-solution. In the thermal spike simulations, Xe gas bubbles
	of varying radii ($R_{bubble}$) were considered. The thermal spike axis was
	also adjusted to account for both on-centered and off-centered collisions
	with the gas bubble. Subsequently, the probability of Xe gas atoms being
	re-solved into the $\gamma$U-10Mo matrix was evaluated as a function of
	bubble radius ($R_{bubble}$), off-centered distance ($r$), and thermal
	spike energy ($S_{e,eff}$). Additionally, the electronic stopping power
	($S_e$) of fission products was simulated to determine the energy imparted
	by the fission products at specific distances. To further refine the
	analysis, a fraction ($\zeta$) of the energy imparted by fission products
	was equated to the thermal spike energy ($\zeta S_e = S_{e,eff}$). Finally,
	using all this information, an overall Xe gas bubble re-solution rate in
	the $\gamma$U-10Mo fuel was computed as a function of bubble radius and
	fission rate, considering reasonable $\zeta$ values.
\end{abstract}

\end{frontmatter}


\linenumbers
\section{Introduction}

% 1st paragraph: necessity of re-solution rate, why U-Mo, why only Xe
% 2nd paragraph: accepted mechanisms of re-solution, their connections
% Paragraphs joined
A $\gamma$U-10Mo alloy-based monolithic fuel design has been identified as the
fuel type for the conversion of the United States High-Performance Research
Reactors (HPRRs) \cite{meyer2014} from high enriched fuel to the low enriched
fuel. However, the $\gamma$U-10Mo monolithic fuel exhibits excessive swelling
during operation \cite{beeler2018gb}. To understand the fuel's behavior under
irradiation, mesoscale and engineering-level fuel performance models require
knowledge of fundamental mechanistic behavior of fission products within the
fuel. Specifically, understanding the progression of Xe gas bubbles in the fuel
is crucial for optimizing the reactor performance and safety. These Xe gas
bubbles act as a sink for individual Xe atoms, trapping them and causing the
bubbles to grow after absorption. Under irradiation, the Xe atoms in the gas
bubble are reintroduced into the fuel matrix through fission product-induced
cascades and thermal spikes, a process known as the radiation enhanced
% enhanced? I'd prefer induced
re-solution. The relative rates of the re-solution determines the overall size
and density of the bubbles \cite{ye2023, olander2006re, parfitt2008}, which
in turn delays the onset of fission gas release and impacts the final gas
release rate.
% 2nd paragraph
The re-solution of fission gas in nuclear fuels involves two
commonly accepted mechanisms: homogeneous re-solution and heterogeneous
re-solution. In the homogeneous re-solution mechanism, atoms from the gas
bubbles are ejected individually through collisions with fission products or
the recoil of U atoms that traverse the bubbles. These atomic collision
cascades are primarily governed by the nuclear stopping power of the material.
% is this right? obliterate is a strong word here! insinuates total destruction.
In the heterogeneous model, entire gas bubbles are obliterated by a passing
fission fragment in the vicinity. The driving mechanism is either the trapping
of the gas atoms in material displaced from one side of the bubble to the
other, or to the vaporization of the dense gas due to the passing pressure wave
\cite{olander2006re}.
% the above sentence is sketchy.
Since both these mechanisms occurs on a short timescale,
it is challenging to determine re-solution rates that contribute to the fission
gas release models through experiments. Therefore, atomistic scale modelling is
not only necessary to determine the re-solution rate but also to elucidate the
fundamental mechanism of re-solution in $\gamma$U-10Mo.

%the electronic stopping of the energetic particles initially raises the
%electronic subsystem temperature. The energy  deposited in the electronic
%subsystem can then transfer to the lattice as thermal energy via
%electron-phonon coupling. Consequently, the local temperature rise above the
%lattice's melting temperature destroys the gas bubble structure, leading to the
%re-solution of Xe atoms \cite{olander2006re, huang2010md, govers2012,
%matthews2015diss, setyawan2018}.

% 5th paragraph: what MD work has been done in other materials, UO2/UC, Xe/He
So far in literature, atomistic simulations have been widely used to evaluate
the re-solution rate in various nuclear materials. For instance, in 2008,
Parfitt et al. \cite{parfitt2008} used simulations of primary knock-on atoms
(PKAs) in uranium dioxide (UO$_2$) to assess the re-solution of helium gas
bubbles. In 2009, Schwen et al. \cite{schwen2009md} investigated the homogeneous
re-solution of Xenon (Xe) gas bubbles in UO$_2$ using a binary collision model
and molecular dynamics (MD) simulations. The following year, Huang et
al. \cite{huang2010md} endeavored to examine the impact of thermal spikes on Xe
re-solution in UO$_2$. In 2012, Govers et al. \cite{govers2012} evaluated both
the PKA and thermal spike models for Xe gas bubbles in UO$_2$ and also proposed
a mathematical model for the re-solution rate. However, the most comprehensive
work on Xe gas bubble re-solution in UO$_2$ was conducted by Setyawan et
al. \cite{setyawan2018} in 2018. They reconciled the inconsistencies in the
conclusions of previous works on Xe bubble re-solution in UO$_2$ and evaluated
the re-solution rate as a function of bubble radius. Their findings suggest
that heterogeneous re-solution of gas bubbles is the dominant method of
re-solution in UO$_2$. In addition to UO$_2$, re-solution rate of fission gas
bubbles has also been evaluated in uranium carbide (UC) by Matthews et
al. \cite{matthews2015diss} using binary collision methods. The thermal spike
model was not employed in UC because the local heating does not exceed the
melting temperature \cite{matthews2015diss, ronchi1986}. In summary, the binary
collision model and MD simulations were used to determine the re-solution rate
in nuclear fuels. Therefore, we employ MD simulations to model the homogeneous
and heterogeneous mechanism of re-solution and  determine the re-solution rates
in $\gamma$U-10Mo.

% 3rd paragraph: PKA, collision cascades, subcascades, channeling
In the MD simulations of homogeneous re-solution, a regular lattice atom is
typically endowed with high kinetic energy to emulate a PKA. The PKA then
interacts ballistically with other atoms, initiating a collision cascade near
the gas bubble to induce maximum disorder \cite{parfitt2008, govers2012}. An
alternative approach in MD is to simulate only a portion of the cascade, known
as a subcascade. This is achieved by imparting energy to a random gas atom
within the bubble. In doing so, the simulation avoids unnecessary cascade
events that may not significantly influence the re-solution process
\cite{schwen2009md}. However, one challenge of modeling homogeneous re-solution
with MD simulations is the channeling of PKAs or their recoils over long
distances without any collisions \cite{jarrin2021}. As a result, collecting
statistics on PKA interactions with gas bubbles can be computationally
demanding, especially when the PKA directions are random. A potential solution
is to direct the PKAs toward a high index lattice direction \cite{stoller2000}.
Therefore, in this work, we direct the PKAs in a particular direction to induce
maximum disorder and to avoid channelling. Whereas, in the MD simulations of
the heterogeneous re-solution, the thermal spike model is a useful tool for
describing the interaction between the fission fragments and fuel. This
interactions primarily occurs through the electronic stopping of the energetic
particles initially raising the electronic subsystem temperature. The energy
deposited in the electronic subsystem can then transfer to the lattice as
thermal energy via electron-phonon coupling. Finally, the energy is transferred
among the atoms, leading to rapid increase in lattice temperature within a
cylindrical zone of typically few nm in radius, known as the thermal spike
\cite{wang1994, toulemonde2002, patra2019}. In MD, electronic interactions
cannot be treated directly. However, the final step described above can be
emulated by raising the temperature of atoms within a cylindrical region, and
this approach is utilized in this work.

%% \cite{olander2006re, huang2010md, govers2012, matthews2015diss, setyawan2018}.

% This should go to the computational section Additionally, MD simulations of
% PKAs necessitates large simulation supercells to prevent the self-interaction
% of cascades through their periodic images.

% 4th paragraph: thermal spike, electron-phonon coupling

% 6th paragraph: what we are doing in this study, relation to DART
For the qualification of $\gamma$U-Mo fuel, it's crucial to accurately predict
the evolution of the fission gas atoms under various operational and transient
conditions. The Dispersion Analysis Research Tool (DART), developed by Argonne
National Laboratory \cite{ye2023}, is a mesoscale code capable of calculating
fission gas swelling in $\gamma$U-Mo under various operational conditions. One
of the many parameters required to model swelling behavior is the re-solution
rate of fission gas bubbles. DART employs a simple re-solution model, which
includes a piecewise function to account for the bubble radius. The parameters
in this function are calibrated by fitting the computed swelling value to
experimental data. As a result, we only have a rough estimate of the
re-solution rate. A physics-based re-solution rate for fission gas bubbles
would make the swelling calculations of higher length scale models more
rigorous and predictive. Therefore, in this study, we investigate the
re-solution of Xe gas bubble in $\gamma$U-Mo fuel using MD simulations by
considering both homogeneous and heterogeneous re-solution mechanisms by
simulating PKAs and thermal spikes, respectively. We also calculate the
electronic stopping power of representative fission products to determine the
energy imparted on gas bubbles by fission events at a certain distance.
Finally, we evaluate an overall re-solution rate of Xe gas bubbles as a
function of both bubble radius and fission rate.


\FloatBarrier
\section{Computational details}

The large-scale atomic/molecular massively parallel simulator (LAMMPS) software
package \cite{lammps} has been utilized for the MD work in conjunction with a
U-Mo-Xe angular-dependent potential (ADP) \cite{beelerADP}. This ADP can
describe the body-centered cubic (BCC) phase of $\gamma$U-Mo alloys accurately,
reproducing their stable structure, modulus of elasticity, room temperature
density, and melting point. Given that this work focuses on high-energy
interactions of PKAs and thermal spike events, the distance between the ions
and atoms can become extremely small. The classical interatomic potential alone
does not provide an accurate description of these interactions. To address
this, the Ziegler, Biersack, and Littmark (ZBL) \cite{ziegler1985} repulsive
potential, which provides a realistic depiction of ion-ion repulsive
interactions, is combined with the ADP for all the simulations.

For both PKA and thermal spike simulations, the system is equilibrated at 400 K
at a pressure of 0 bar in an NPT ensemble using the Nos\'e-Hoover barostat and
thermostat. Using the resulting configurations, spherical bubbles of radius
$R_{bubble}$ are created by removing a stoichiometric number of U and Mo atoms
and depositing gaseous Xe atoms at a given density within the resulting void.
The system is then further equilibrated in the NPT ensemble. Simulations are
finally performed using an NVE ensemble with a canonical sampling thermostat
\cite{bussi2007} at the edges of the simulation box to act as a heat sink. This
allows the resulting local heating to slowly dissipate from the simulation by
modeling a constant long-range temperature. Electronic stopping is incorporated
into the simulation by applying a frictional force to each atom. The frictional
force is calculated using the stopping power of the related atom species in the
fuel. These values are determined using the stopping and range of ions in
matter (SRIM) software \cite{ziegler2010srim}. The frictional force due to
electronic stopping is applied to atoms having more kinetic energy than a
specified threshold. This threshold is often set to 10 eV or double the
cohesive energy for metals \cite{nordlund1998, duffy2006}. In our work,
electronic stopping is applied only when an atom has a kinetic energy of more
than 9 eV, which is twice the U-U cohesive energy of 4.5 eV. The cohesive
energy is evaluated by Beeler et al. \cite{beeler2018disp} using the same ADP
potential used in this work. A variable timestep size is also implemented for
all simulations such that the maximum displacement of any atom between two
successive timesteps is less than or equal to 0.01 \r{A}.

For the PKA model, a supercell of dimensions  $160 \times 160 \times 160$
\r{A}$^3$ is utilized, which holds around 8 millions atoms. Such a large
supercell is needed to prevent the self-interaction of cascades through their
periodic images. The thermostatted sink is implemented in all three directions.
Figure \ref{fig:struct} illustrates the cross-section of such a supercell.
Within a 20 \r{A} gas bubble, the investigation is conducted on 200 and 400 Xe
atoms, which correspond to pressures of 124 MPa and 787 MPa, respectively. The
bubbles having 200 Xe atoms are under-pressurized, whereas 400 Xe atoms lead to
an over-pressurized bubble. For reference, the equilibrium pressure of such a
bubble is 436 MPa \cite{beelerADP}. The cascade is initiated by selecting a
single PKA, located approximately 5.5 nm from the bubble's center. The PKA is
given a high velocity toward the bubble. PKAs of kinetic energies up to 500 keV
are simulated. These simulations are executed to model the initial 120
picoseconds of the cascade, ensuring that the total temperature of all the
examined systems fall below 600 K. For each bubble pressure and PKA energy,
five simulations are performed with different initial atomic configuration and
PKA directions to gather statistical data.

\begin{figure}[ht]
	\centering
	\includegraphics[height=7cm]{images/struct.png}
	\caption{
		Initial configuration for an MD simulation of the PKA. The supercell is
		sliced through the middle to show the 20 \r{A} radius Xe gas bubble
		(black). Red and blue dots represent U and Mo atoms, respectively.
	}
	\label{fig:struct}
\end{figure}

In the thermal spike model, a supercell of dimensions $120 \times 120 \times
50$ \r{A}$^3$ is simulated, containing approximately 1.5 million atoms. The
thermal spike axis is aligned with the shortest dimension of the supercell and
the thermostatted sink is applied only in directions perpendicular to the
thermal spike axis. Atoms within a cylindrical zone of 35 \r{A} radius along
the thermal spike axis are excited with high kinetic energy to simulate local
heating due to electronic stopping. This is achieved by rescaling the
velocities of the atoms. To understand the role of only the bubble size, the
Xe/vacancy ratio is kept constant at first. A Xe/vacancy ratio of 0.2 is chosen
for this purpose since it ensures an equilibrium bubble pressure as per Beeler
et al. \cite{beeler2020improved}. The thermal spike's position within the
supercell is varied to account for both on-centered and off-centered collisions
between the gas bubble and thermal spike. For on-centered thermal spikes, gas
bubbles with radii ranging from 0 \r{A} to 40 \r{A} are simulated , with
thermal spike energies varying from 5 keV/nm to 30 keV/nm. For off-centered
thermal spikes, gas bubbles of radii 15 \r{A}, 25 \r{A}, and 35 \r{A} are
examined with thermal spike energy of only 15 keV/nm. The off-centered distance
is varied at an interval of 10 \r{A} up to a specific cutoff value, which is
discussed later. Bubbles of different Xe/vacancy ratio are also simulated to
consider the effect of bubble pressure on re-solution. Xe/vacancy ratios of
0.1, 0.3, and 0.5 are used for bubbles of radii 15 \r{A}, 25 \r{A}, and 35
\r{A}. Only on-centered thermal spikes are employed in this case and their
energy is varied from 5 keV/nm to 30 keV/nm.

\begin{figure}[ht]
	\centering
	\begin{subfigure}{0.49\textwidth}
		\centering
		\caption{}
		\includegraphics[height=6cm]{images/temp_time.pdf}
	\end{subfigure}
	\begin{subfigure}{0.49\textwidth}
		\centering
		\caption{}
		\includegraphics[height=6cm]{images/resol_time.pdf}
	\end{subfigure}
	\caption{
		(a) System temperature versus time for 40 \r{A} radius Xe gas bubble
		simulations with different thermal spike energies. (b) Number of
		re-solved Xe atoms versus time. The vertical dotted lines represent 450
		ps.
	}
	\label{fig:cooldown}
\end{figure}

The thermal spike simulations are run for a minimum of 450 picoseconds,
sufficient for all systems to cool down below 800 K. Figure \ref{fig:cooldown}
shows the system temperature and the number of re-solved Xe atoms over time for
a few 40 \r{A} radius bubble simulations with various thermal spike energies.
The number of resolved atoms stabilizes well before 450 picoseconds, as
observed from the figure \ref{fig:cooldown}. Additionally, longer simulations
are conducted to allow the systems to cool down below 500 K for further
verification. The re-solution behavior remains unchanged during the cooldown
from 800 K to 500 K. Thus, a minimum simulation time of 450 picoseconds is
chosen to optimize the use of computing resources. Cluster analysis is
performed on the Xe atoms in the system using OVITO \cite{ovito}. The cutoff
distance among clusters is chosen to be 10 \r{A}. Initially, all the Xe atoms
in the system form a single cluster, representing the original Xe bubble.
However, following the introduction of PKAs or thermal spikes, multiple Xe
clusters may be identified due to the re-solution of Xe atoms. The cluster
containing the most Xe atoms is considered the original Xe gas bubble and all
Xe atoms in the other clusters are classified as the re-solved atoms. If only a
single atom remains in the original bubble, the gas bubble is considered to be
completely re-solved.


\FloatBarrier
\section{Results}

\FloatBarrier
\subsection{PKA simulations}

In the PKA simulations, limited re-solution is observed. Figure 2 showcases
snapshots from one such simulation, where the black spheres indicate Xe atoms,
and the red and blue spheres correspond to U and Mo atoms, respectively. The
energy introduced into the system by the PKA propagates as a shock wave toward
the supercell boundary. This shock wave generates numerous point defects, the
majority of which eventually annihilate. The Xe gas bubble undergoes
deformation immediately after the PKA initiation. However, the gas bubble
swiftly reverts to a stable configuration. In the PKA simulations, at most,
only one re-solved Xe atom is observed, with most simulations showing no
re-solution at all. Therefore, homogeneous re-solution is considered negligible
within the examined energy range of the PKAs (up to 500 keV).

Homogeneous re-solution is expected to be a few orders of magnitude smaller
than the heterogeneous re-solution since the collision cross section of a PKA
is way smaller than the interaction cross section between a bubble and a
thermal spike \cite{olander2006re}. Govers el al. \cite{govers2012} performed
PKA simulations for UO$_2$ and observed at most 3 re-solved Xe atoms out of gas
bubbles containing hundreds. Therefore, they deemed the homogeneous re-solution
not significant enough for the calculation of overall re-solution rate. The PKA
simulations performed in this work in $\gamma$U-10Mo show similar results.
Also, in the case of $\gamma$U-Mo, only 5\% of the energy from a fission event
is deposited ballistically \cite{beeler2021rad}. Given that limited number
re-solved atoms are observed in the simulations and only a small fraction of
the energy from fission reactions go to ballistic collisions, it is reasonable
to assume that the homogeneous re-solution is negligible.

\begin{figure}[ht]
	\centering
	\begin{subfigure}{0.45\textwidth}
		\centering
		\caption{}
		\includegraphics[width=\textwidth]{images/pka1.png}
	\end{subfigure}
	\begin{subfigure}{0.45\textwidth}
		\centering
		\caption{}
		\includegraphics[width=\textwidth]{images/pka2.png}
	\end{subfigure}

	\begin{subfigure}{0.45\textwidth}
		\centering
		\caption{}
		\includegraphics[width=\textwidth]{images/pka3.png}
	\end{subfigure}
	\begin{subfigure}{0.45\textwidth}
		\centering
		\caption{}
		\includegraphics[width=\textwidth]{images/pka4.png}
	\end{subfigure}
	\caption{
		Snapshots of a PKA (500 keV) simulation at (a) 0 ps, (b) 1.5 ps, (c)
		44.5 ps, and (d) 114.5 ps. Xe atoms are shown in black along with U
		(red) and Mo (blue) atoms.
	}
	\label{fig:pka}
\end{figure}


\FloatBarrier
\subsection{Thermal spike simulations}

% On-centered thermal spike

Contrarily, the thermal spike model demonstrated significant re-solution of Xe
atoms. Figure \ref{fig:spike} showcases several snapshots of an on-centered
thermal spike simulation. The thermal spike forms a cylindrical volume of
liquid that engulfs the entire Xe gas bubble, leading to its disintegration.
However, the region cools down significantly after a few picoseconds. While
many Xe atoms coalesce into a large gas bubble upon cooling, a substantial
number of Xe atoms remain in the fuel matrix or form other smaller bubbles. The
thermal spikes also generate shock waves that propagate radially to the
boundary and create point defects in the system. The number of point defects
decreases with cooling through the sink. Since the re-solution observed in the
thermal spike model is substantially greater than in the PKA model, we only
consider the heterogeneous re-solution of Xe atoms in subsequent calculations.

\begin{figure}[ht]
	\centering
	\begin{subfigure}{0.45\textwidth}
		\centering
		\caption{}
		\includegraphics[width=\textwidth]{images/spike1.png}
	\end{subfigure}
	\begin{subfigure}{0.45\textwidth}
		\centering
		\caption{}
		\includegraphics[width=\textwidth]{images/spike2.png}
	\end{subfigure}

	\begin{subfigure}{0.45\textwidth}
		\centering
		\caption{}
		\includegraphics[width=\textwidth]{images/spike3.png}
	\end{subfigure}
	\begin{subfigure}{0.45\textwidth}
		\centering
		\caption{}
		\includegraphics[width=\textwidth]{images/spike4.png}
	\end{subfigure}
	\caption{
		Snapshots of a thermal spike (30 keV/nm) simulation at (a) 0 ps, (b)
		1.5 ps, (c) 44.5 ps, and (d) 114.5 ps. Xe atoms are shown in black
		along with U (red) and Mo (blue) atoms.
	}
	\label{fig:spike}
\end{figure}

From the on-centered thermal spike simulations for different bubble sizes,
re-solution data is gathered and plotted in Figure \ref{fig:resol}(a) as a
fraction of re-solved Xe atoms against the effective energy transferred to the
lattice, $S_{e,eff}$. Thermal spikes re-solve a greater fraction of Xe atoms in
the smaller bubbles compared to the larger ones. Also, the fraction of
re-solved Xe atoms seems to saturate with increasing deposited energy. Thus, an
exponentially saturating function like the following is used to model the
available data.
\begin{align}
	\chi_0 &= 1 - \exp[-\alpha S_{e,eff}]
\end{align}
where $\chi_0$ is the fraction of re-solved atoms due to an on-centered thermal
spike, and $\alpha$ is the saturation factor. The saturation factors for
different bubble sizes are plotted in \ref{fig:resol}(b) against bubble radius.
The graph suggests there is an inverse proportionality between the saturation
factor and the bubble size, which means the larger bubbles are more difficult
to re-solve completely. To express this relationship, the saturation factor is
made an inverse power function of bubble radius as follows.
\begin{align}
	\alpha &= \frac{5.1}{R_{bubble}^{2.2}}
\end{align}

\begin{figure}[ht]
	\centering
	\begin{subfigure}{0.69\textwidth}
		\centering
		\caption{}
		\includegraphics[height=6cm]{images/resolutionVsRadius.pdf}
	\end{subfigure}
	\begin{subfigure}{0.3\textwidth}
		\centering
		\caption{}
		\includegraphics[height=6cm]{images/saturationFactor.pdf}
	\end{subfigure}
	\caption{
		(a) Fraction of re-solved Xe atoms as a function of the energy
		deposited to the lattice. (b) Saturation factor as a function of bubble
		radius.
	}
	\label{fig:resol}
\end{figure}

% Off-centered thermal spike

% What is the thermal spike energy?
% Discuss bubble movement for off-centered thermal spikes
% maybe include some data as well

The results discussed so far pertain to on-centered thermal spikes. It is
expected that off-centered thermal spikes will have diminishing effects on
re-solution, as smaller portion of the Xe gas bubble would be enveloped by the
initial thermal spike region. Furthermore, all re-solution is anticipated to
cease when the initial cylindrical region of the thermal spike no longer
contacts the bubble surface. Let's define the off-centered distance $r$ as the
distance between the Xe gas bubble center and the cylindrical axis of the
thermal spike. The farthest distance at which the thermal spike contacts the
bubble can be denoted as $r_{c} \equiv R_{bubble} + R_{spike}$. Simulations are
conducted with off-centered distances ranging from 0 \r{A} to $r_{c}$ \r{A} at
an interval of 10 \r{A}.

To compare the results from different bubble sizes, both the Xe re-solution
fraction and the off-centered distance are normalized. The fraction of
re-solved Xe atoms $\chi$ is normalized using the fraction of re-solved Xe
atoms for on-centered thermal spike $\chi_0$, and the off-centered distance $r$
is normalized using the $r_{c}$, which varies for different bubble sizes. The
normalization allows for the comparison of re-solution data from different
bubble sizes. Figure \ref{fig:off} depicts the normalized fraction of re-solved
atoms $\chi/\chi_0$ as a function of $r/r_c$, which highlights how the effect
of the relative off-centered distance of the thermal spike is similar for
bubbles of different sizes. The data was fitted to a logistic function due to
the plateaus observed at both ends of the off-centered distance data. The
fitted equation is as follows.
\begin{align}
	\label{eq:off}
	\frac{\chi}{\chi_0}
	&= \frac{1.058}{1 + \exp \big[8.168 \big(\frac{r}{r_c}\big) - 3.331 \big]}
\end{align}

\begin{figure}[ht]
	\centering
	\includegraphics[width=8cm]{images/offcentered.pdf}
	\caption{
		Re-solution due to 15 keV/nm off-centered thermal spikes.
	}
	\label{fig:off}
\end{figure}

% Bubble pressure

To understand the effect of bubble pressure, bubbles with various Xe/vacancy
ratios are simulated. Subfigure \ref{fig:pres}(a) shows the number of re-solved
Xe atoms as a function of Xe/vacancy ratio within a gas bubble of 25 \r{A}
radius. The number of re-solved atoms apears to be invariant with respect to
the Xe/vacancy ratio, which is positively correlated with bubble pressure.
Similar trends are observed in bubbles with radii of 15 \r{A} and 35 \r{A}.
Consequently, the number of re-solved Xe atoms from bubbles of the same radius
but different pressures are averaged and presented in subfigure
\ref{fig:pres}(b). The number of re-solved Xe atoms also appears to be
consistent across bubbles of various sizes. Therefore, the number of re-solved
Xe atoms does not seem to depend on either bubble pressure or size. Only the
thermal spike energy seems to have a relationship with the number of re-solved
atoms, which is also apparent in subfigure \ref{fig:resol}(a). One possible
explanation for this behavior is that thermal spikes create low density regions
around the Xe gas bubble and these low density regions facilitate the
separation of Xe atoms from the bubble.

\begin{figure}[ht]
	\centering
	\begin{subfigure}{0.49\textwidth}
		\centering
		\caption{}
		\includegraphics[height=6cm]{images/xevac.pdf}
	\end{subfigure}
	\begin{subfigure}{0.49\textwidth}
		\centering
		\caption{}
		\includegraphics[height=6cm]{images/r2dep.pdf}
	\end{subfigure}
	\caption{
		(a) Number of re-solved Xe atoms against Xe/vac ratio of 25 \r{A}
		radius bubbles. (b) Number of re-solved Xe atoms against bubble radius.
		Data from the bubbles of same radius but different Xe/vac ratios are
		averaged.
	}
	\label{fig:pres}
\end{figure}


\FloatBarrier
\subsection{Re-solution rate}

{\color{teal}
--------------------
done till this point
--------------------
}

% Definition of re-solution rate?
% Explain how the pressure factor comes in.

Both bubble size and pressure are going to be considered in the calculation of
the bubble re-solution rate. As we have stated in the previous subsection, the
number of re-solved Xe atoms remains constant with pressure. This observation
can be used to decouple the re-solution calculation as follows.
\begin{align}
	b_{het}(R_{bubble}, \dot{F}, \phi)
		&= f(R_{bubble}, \dot{F}) \cdot g(\phi)
\end{align}
where $b_{het}$ is the heterogeneous re-solution rate, $\dot{F}$ is the fission
rate, and $\phi$ is the Xe/vacancy ratio in the bubble. Xe/vacancy ratio,
$\phi$ is closely related to the bubble pressure, $P$ and this relation is
going to be discussed later. The function $f(R_{bubble}, \dot{F})$ calculates
the re-solution rate of Xe atoms from gas bubbles having $\phi=0.2$, whereas
the function $g(\phi)$ corrects the re-solution rate for bubbles deviating from
the nominal $\phi$ value of 0.2.

\begin{figure}[ht]
	\centering
	\includegraphics[height=6cm]{images/coordSystem.pdf}
	\caption{
		Cylindrical coordinate system for calculating the heterogeneous
		re-solution rate.
	}
	\label{fig:coord}
\end{figure}

To represent the overall re-solution behavior, the contributions from all the
fission products, originating at different distances from a bubble and oriented
toward a random direction, need to be summed up by means of a volume integral.
Consider a cylindrical coordinate system where the gas bubble is at the origin.
To make calculations easier, fission product origins can be rotated around the
coordinate system origin so that all fission tracks (and the thermal spikes
they create) point in the same direction. Given the fission product generation
is uniform and isotropic in the material, the aforementioned rotational
transformation will lead to a uniform distribution of the fission products that
are unidirectional. The axial coordinate $x$ is then defined to be parallel to
the fission tracks and the radial coordinate $r$ perpendicular to the tracks.
The behavior with respect to the azimuth is constant because the fission
products are unidirectional after the transformation. If we denote the fission
rate as $\dot{F}$, the number of fission events per second in an infinitesimal
volume $dV = 2 \pi dr dx$ would be $\dot{F} dV$. Each fission product $i$ from
a fission event contributes $\chi_i \dot{F} dV$ to the total re-solution of the
bubble. The heterogeneous re-solution rate of bubbles having $\phi=0.2$ can
then be expressed as follows.
\begin{align}
	f(R_{bubble}, \dot{F})
	&= \sum_{i=1}^2 \int_V \chi_i \dot{F} dV \\
	&= \sum_{i=1}^2 \int_{x=0}^{\infty} \int_{r=0}^{\infty}
		\chi_i \dot{F} 2 \pi r dr dx \\
	&= \dot{F} \sum_{i=1}^2 \int_{r=0}^{\infty}
		\bigg( \frac{\chi_i}{\chi_{0,i}} \bigg)
		2 \pi r dr \int_{x=0}^{\infty} \chi_{0,i} dx
\end{align}
where it is assumed that a single fission reaction creates two fission
products, thus ignoring ternary fission reactions. Also, the double integral is
decoupled into two one-dimensional integrals since
$\Big(\frac{\chi_i}{\chi_{0,i}}\Big)$ is dependent on $r$ only and $\chi_{0,i}$
on $x$.

Now, we can compute the $r$ integral using equation \ref{eq:off}. The upper
limit of the integral can be changed to $r=r_c$ since any re-solution is
presumed to be nonexistent for $r > r_c$.
\begin{align}
	\int_{r=0}^{r_c} \bigg( \frac{\chi_i}{\chi_{0,i}} \bigg) 2 \pi r dr
	&= 2 \pi r_c^2 \int_{r/r_c=0}^{1}
		\frac{1.058}{1 + \exp \big[8.168 \big(\frac{r}{r_c}\big) - 3.331 \big]}
		\bigg(\frac{r}{r_c}\bigg) d\bigg(\frac{r}{r_c}\bigg) \\
	&\approx 0.225 \pi r_c^2
\end{align}

Thus, the nominal heterogeneous re-solution rate $f(R_{bubble}, \dot{F})$ can
be written as
\begin{align}
	f(R_{bubble}, \dot{F})
	&= \dot{F} \sum_{i=1}^2 \int_{r=0}^{\infty}
		\bigg( \frac{\chi_i}{\chi_{0,i}} \bigg)
		2 \pi r dr \int_{x=0}^{\infty} \chi_{0,i} dx \\
	&= \dot{F} \sum_{i=1}^2 0.225 \pi r_c^2 \int_{x=0}^{\infty}
		[1 - \exp(-\alpha S_{e,eff,i})] dx \\
	\label{eq:dx}
	&= 0.225 \pi r_c^2 \dot{F} \sum_{i=1}^2 \int_{x=0}^{\infty}
		[1 - \exp(-5.1 \zeta S_{e,i} / R_{bubble}^{2.2})] dx 
\end{align}
where $S_{e,i}$ is the electronic stopping power of the fission product $i$,
and $\zeta = S_{e,eff} / S_{e}$ is the fraction of fission product energy that
is imparted to the lattice. In this work, $\zeta$ is assumed to be an unknown
parameter ranging somewhere between 0.55 and 0.95. The introduction of this
parameter is necessary to account for the fraction of the fission product
energy that is not transferred to the thermal spike.

We choose to evaluate the $S_e$ values for Xe-140 and Sr-94 in $\gamma$U-10Mo
based on the fission reaction $_0^1n + _{92}^{235}U \rightarrow _{54}^{140}Xe +
_{38}^{94}Sr + Q$. Even though it is one of many possible fission reactions
that may take place, Xe-140 and Sr-94 can still be considered as
representatives of heavy and light fission products. The rationale behind this
is that the two aforementioned species appear close to the two peaks of a
typical fission product yield graphs (citation needed).

$S_{e,Xe}$ and $S_{e,Sr}$ as a function of the distance traversed by the
fission product is shown in Figure \ref{fig:elec}. This data is generated using
the SRIM software. The following model is fit to represent the $S_e$ data.
\begin{align}
	\label{eq:xe}
	S_{e,Xe} &= 21.3 \exp(-0.239 x^{1.78})
		+ 5.23 \exp(-4.67 \times 10^{-8} x^{11}) \\
	\label{eq:sr}
	S_{e,Sr} &= 19.7 \exp(-0.00273 x^{3.71})
		+ 6.8 \exp(-0.424 x^{1.45})
\end{align}

\begin{figure}[ht]
	\centering
	\includegraphics[width=8cm]{images/elec_stopping.pdf}
	\caption{
		Total electronic stopping power ($S_e$) of Xe-140 and Sr-94 in
		$\gamma$U-10Mo calculated with the SRIM software as a function of
		distance traversed by the fission product from the location of the
		fission reaction.
	}
	\label{fig:elec}
\end{figure}

Finally, the re-solution rate can be calculated for different values of $\zeta$
and bubble radius $R_{bubble}$ using equations \ref{eq:dx}, \ref{eq:xe}, and
\ref{eq:sr}. Figure \ref{fig:res} displays the result from this work. The
re-solution rates are computed using numerical integration, and the data can be
extracted from the associated graphs or by recalculating the integrals.
However, an approximate analytical function might serve the higher length scale
models better. To that end, we propose the following function for fitting the
computed re-solution rate.
\begin{align}
	\label{eq:param}
	f(R_{bubble}, \dot{F})
	&= \bigg[ \frac{a}{1 + (R_{bubble} / c)^d} \bigg]
		(10^{-14} \: \dot{F})
\end{align}
where $R_{bubble}$ is in \r{A}, $\dot{F}$ is in fissions per cubic centimeter
per second (fiss/cm$^3$/s), and $a$, $c$ and $d$ are adjustable parameters. The
parameter values for different $\zeta$ are listed in Table 1. The parameter $d$
remains consistent up to three decimal places for all $\zeta$. To determine the
re-solution rate of any arbitrary $\zeta$ between 0.55 and 0.95, linear
interpolation would be sufficient.

\begin{figure}[ht]
	\centering
	\includegraphics[height=7cm]{images/resRate.pdf}
	\caption{
		Xe gas bubble re-solution rate in $\gamma$U-10Mo as a function of
		bubble radius at a fission rate of $10^{14}$ fiss/cm$^3$/s.
	}
	\label{fig:res}
\end{figure}

\begin{table}[ht]
\centering
\caption{Equation \ref{eq:param} parameter fits for different $\zeta$ values.}
\label{tab:param}
\begin{tabular}{llll}
\toprule
$\zeta$     & $a$        & $c$     & $d$      \\
\midrule
0.55        & 0.0167     & 2.717   & 1.225    \\
0.65        & 0.0163     & 3.184   & 1.225    \\
0.75        & 0.0161     & 3.635   & 1.225    \\
0.85        & 0.0159     & 4.073   & 1.225    \\
0.95        & 0.0158     & 4.502   & 1.225    \\
\bottomrule
\end{tabular}
\end{table}

% Multiply the pressure factor

Now, let's consider the effect of $\phi$ and $P$ on re-solution. Since the
number of re-solved atoms is constant with respect to $\phi$ for a specific
bubble size, bubbles having more Xe atoms (and thus more pressure) will have
less re-solution. This leads to the following simple expression.
\begin{align}
	g(\phi) &= \frac{0.2}{\phi}
\end{align}

The Xe/vac ratio, $\phi$ can be used to get the molar volume of Xe, $v$. At 400
K, the equilibrium volume of $\gamma$U-Mo is $19.7$ \r{A}$^3$/atom. Therefore,
$v$ can be expressed the following way.
\begin{align}
	v
	&= \frac{N_A \cdot 19.7}{\phi} \text{ \r{A}}^3/\text{mol} \\
	\label{eq:v}
	&= \frac{11.86}{\phi} \text{ cm}^3/\text{mol}
\end{align}
where $N_A$ is the Avogadro constant.

With the molar volume available, the pressure, P can be computed using the Xe
bubble equation of state (EOS) provided by Beeler et al. in \cite{beelerADP}.
The Virial EOS is expanded to the third order with respect to volume and to
second order with respect to temperature. The EOS is as follows.
\begin{align}
	\label{eq:eos}
	P &= \frac{RT}{v}
		\bigg( A + \frac{B}{v} + \frac{C}{v^2} + \frac{D}{v^3} \bigg)
\end{align}
where $R$ is the gas constant ($8.3145$ J/mol-K), and $A=1$, $B=151.12 \text{
cm}^3/\text{mol}$, $C=2976 \text{ cm}^6/\text{mol}^2$, and $D=705527 \text{
cm}^9/\text{mol}^3$ at temperature $T=400$ K. Combining equations \ref{eq:v}
and \ref{eq:eos} provide a direct relationship between $\phi$ and $P$.
\begin{align}
	P
	&= (280.4 \: \phi + 3573 \: \phi^2 + 5933 \: \phi^3
		+ 118600 \: \phi^4) \: \text{ MPa}\\
	&\approx (5894.417 \: \phi^2 + 123341.14 \: \phi^4) \: \text{ MPa},
		\quad 0.05 \leq \phi \leq 0.50
\end{align}
where $P$ is approximatd as a quadratic function of $\phi^2$ to make algebraic
manipulation easier. With that, we can express $g(\phi)$ as follows.
\begin{align}
	g(\phi)
	= \frac{0.2}{\phi}
	= \frac{k}{(l+\sqrt{m+nP})^{1/2}}
	= h(P)
\end{align}
where $P$ is in MPa, and $k=99.334$ (MPa)$^{1/2}$, $l=-5894.417$ MPa,
$m=34744147$ (MPa)$^2$, and $n=493364.55$ MPa. Both $g(\phi)$ and $h(P)$ are
unitless.

Now that we have two equivalent functions of $\phi$ and $P$, the heterogeneous
re-solution rate take the following forms.
\begin{align}
	b_{het}(R_{bubble}, \dot{F}, \phi)
		&= f(R_{bubble}, \dot{F}) \cdot g(\phi)
		= f(R_{bubble}, \dot{F}) \cdot h(P) \\
		&= \frac{a}{1+(R_{bubble}/c)^d} \cdot \frac{0.2}{\phi} \cdot
			10^{-14} \dot{F} \\
		\label{eq:fin}
		&= \frac{a}{1+(R_{bubble}/c)^d} \cdot \frac{k}{(l + \sqrt{m+nP})^{1/2}}
			\cdot 10^{-14} \dot{F}
\end{align}

The re-solution rate is a crucial parameter for calculating fission gas
swelling in $\gamma$U-10Mo. The mesoscale program Dispersion Analysis Research
Tool (DART) incorporates this parameter for such calculations \cite{ye2023}.
The currently used re-solution rate $b_{dart}$ in DART is defined as follows.
\begin{align}
	b_{dart} &= b_0 \cdot \dot{F} \cdot G \\
	b_0 &= R_{spike}^2 \cdot \mu_{ff} \\
	G &=
	\begin{cases}
		1 & ,R_{bubble} \leq \lambda \\
		1 - (\frac{R_{bubble}-R_{resol}}{R_{bubble}})^3
		  & ,R_{bubble} > \lambda
	\end{cases}
\end{align}
where $b_0$ is the bubble destruction probability, $\dot{F}$ is the fission
rate, and $G$ is a piecewise function representing different re-solution modes
for small and large gas bubbles. The parameter $b_0$ can be estimated using the
interaction volume of a thermal spike with bubbles, given by the formula, $b_o
= R_{spike}^2 \times \mu_{ff}$, where $R_{spike}$ is the radius of a thermal
spike and $\mu_{ff}$ is the recoil length of fission fragments. In the
piecewise function G, $R_{bubble}$ is the bubble radius, $\lambda$ is the
gas-atoms knock out distance, and $R_{resol}$ is the thickness of the annulus
within which all gas-atoms are knocked out. The parameters $b_0$, $\lambda$,
and $R_{resol}$ are considered adjustable, and the optimized values are $b_0 =
2 \times 10^{-18}$ cm$^3$, $\lambda = 5 \times 10^{-7}$ cm, and $R_{resol} = 3
\times 10^{-9}$ cm as reported in \cite{ye2023}. Since the parameters $\lambda$
and $R_{resol}$ are not coupled, the re-solution rate is discontinuous at
bubble radius $\lambda$, which is unphysical. Figure \ref{fig:dart} shows the
re-solution rate for bubble radius up to 100 \r{A} as used in DART along with
the re-solution rate computed in this work. Unlike the DART model prediction,
the re-solution rate computed in this work is a smooth decaying function of
bubble radius. The calculated re-solution rate is about two orders of magnitude
higher than the DART prediction for very small bubbles. This difference
increases to about three orders of magnitude for large bubbles. Only for
bubbles of radius around 50 \r{A}, the difference is less than one order of
magnitude. The value of $\zeta$ does not impact this comparison, since the
re-solution rate at $\zeta=0.95$ is at most double of the rate at $\zeta=0.55$.

\begin{figure}[ht]
	\centering
	\includegraphics[height=7cm]{images/resRate_withDart.pdf}
	\caption{
		Comparison of the calculated re-solution rate against DART model
		prediction \cite{ye2023}.
	}
	\label{fig:dart}
\end{figure}


\FloatBarrier
\section{Conclusion}

In this study, MD simulations are utilized to determine the re-solution rate of
Xe gas bubbles in $\gamma$U-10Mo fuel. Both homogeneous and heterogeneous
re-solution processes are simulated using the primary knock-on atom and thermal
spike methods. However, the homogeneous re-solution is found to be negligible
in describing the overall re-solution behavior. The fraction of re-solved Xe
gas bubble atoms appears to be invariant with respect to bubble pressure and Xe
density. This fraction is subsequently used to model the re-solution fraction
as a function of effective energy imparted into the lattice for different
bubble sizes. Simulations of both on-centered and off-centered thermal spikes
are conducted to get the spatial correlation between re-solution and thermal
spike distance. Additionally, the electronic stopping power of fission products
(Xe and Sr) in $\gamma$U-10Mo is simulated to provide distance-dependent
electronic stopping power. All this information is then used to calculate the
overall re-solution rate of Xe gas bubbles in the fuel. The re-solution rate of
Xe gas bubble atoms in $\gamma$U-10Mo is found to be on the order of $10^{-4}$
to $10^{-2}$ s$^{-1}$ for a fission rate of $10^{14}$ fissions/cm$^3$/s. An
analytical form of re-solution rate dependent on bubble radius, fission rate,
and $\zeta$ is also provided. The result from this work will inform higher
length scale models of $\gamma$U-10Mo with a physics-based description of Xe
gas bubble re-solution rate.


\bibliographystyle{unsrt}
\bibliography{ref.bib}

\end{document}

%\begin{align}
%	b_{het} &= f(R_{bubble}) \cdot g(\phi) \\
%	b_{het}(R_{bubble}, \dot{F}, \zeta, \phi)
%		&= f(R_{bubble}, \dot{F}, \zeta) \cdot g(\phi) \\
%	b_{het}(R_{bubble}, \dot{F}, \phi)
%		&= f(R_{bubble}, \dot{F}) \cdot g(\phi) \\
%		&= \frac{a}{1+(R_{bubble}/c)^d} 10^{-14} \dot{F} \cdot g(\phi) \\
%		&= \bigg[ \frac{a}{1+(R_{bubble}/c)^d} 10^{-14} \dot{F} \bigg]
%			\bigg( \frac{0.2}{\phi} \bigg) \\
%		&= \bigg[ \frac{a}{1+(R_{bubble}/c)^d} 10^{-14} \dot{F} \bigg]
%			\bigg[ \frac{k}{(l + \sqrt{m+nP})^{1/2}} \bigg]
%\end{align}
%
%\begin{align}
%	P
%	&= \frac{RT}{v}
%		\bigg( A + \frac{B}{v} + \frac{C}{v^2} + \frac{D}{v^3} \bigg) \\
%	P
%	&= 280.4 \: \phi + 3573 \: \phi^2 + 5933 \: \phi^3
%		+ 118600 \: \phi^4 \\
%	P
%	&\approx 6306.489 \: \phi^2 + 122200.78 \: \phi^4 \\
%	\phi^2
%	&= \frac{-6306.489 + \sqrt{(6306.489)^2 + 4 (122200.78) P}}{2 \times
%		122200.78} \\
%	&= \frac{-6306.489 + \sqrt{39771801 + 488803.14 P}}{244401.57} \\
%	&= \frac{-b + \sqrt{b^2 - 4ac}}{2a} \\
%	g(\phi)
%	&= \frac{98.874}{(-6306.489 + \sqrt{39771801 + 488803.14 P})^{1/2}} \\
%	&= \frac{a}{(l+\sqrt{m+nP})^{1/2}}
%\end{align}

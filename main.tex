\documentclass{article}
\usepackage[margin=1in]{geometry}
\usepackage{authblk}
\usepackage{graphicx, xcolor, placeins}
\usepackage{amsmath, booktabs, subcaption}

\title{Re-solution of Xe gas bubbles in the $\gamma$U-10Mo fuel}
\author[1]{ATM Jahid Hasan}
\author[2]{Linu Malakkal}
\author[2]{Mathew Swisher}
\author[1,2]{Benjamin Beeler}
\affil[1]{North Carolina State University, Raleigh, NC}
\affil[2]{Idaho National Laboratory, Idaho Falls, Idaho}

\begin{document}
\maketitle


\begin{abstract}
	Molecular dynamics simulations are performed to quantify the Xe gas bubble
	re-solution rate in $\gamma$U-10Mo fuel. Primary knock-on atom (PKA) and
	thermal spike models are employed to evaluate homogeneous and heterogeneous
	re-solution, respectively. It is found that the homogeneous re-solution is
	negligible compared to the heterogeneous re-solution. Thermal spike
	simulations are performed for Xe gas bubbles of different radii
	($R_{bubble}$). The thermal spike axis is also varied so that both
	on-centered and off-centered collisions with the gas bubble are accounted
	for. Subsequently, the probability of Xe gas atoms being re-solved into the
	$\gamma$U-10Mo matrix is evaluated as a function of bubble radius
	($R_{bubble}$), off-centered distance ($r$), and thermal spike energy
	($S_{e,eff}$). The electronic stopping power ($S_e$) of fission products
	are also simulated to find the energy imparted by the fission products at a
	certain distance. A fraction ($\zeta$) of the energy imparted by fission
	products is equated to the thermal spike energy ($\zeta S_e = S_{e,eff}$).
	Eventually, all the information is used to compute an overall Xe gas bubble
	re-solution rate in the $\gamma$U-10Mo fuel as a function of bubble radius
	and fission rate for reasonable $\zeta$ values.
\end{abstract}


\section{Introduction}

% 1st paragraph: necessity of re-solution rate, why U-Mo, why only Xe
A $\gamma$U-Mo alloy-based monolithic fuel design has been identified as the
fuel type for the conversion of the United States High-Performance Research
Reactors (HPRRs) \cite{meyer2014}. However, the $\gamma$U-Mo monolithic fuel
exhibits excessive swelling during operation \cite{beeler2018gb}. To understand
the fuel evolution under irradiation, mesoscale and engineering-level fuel
performance models require knowledge of fundamental mechanistic behavior of
fission products in the fuel. As such, understanding the progression of Xe gas
bubbles in the fuel is crucial for optimizing the performance and safety of the
nuclear reactors. The Xe gas bubbles act as a sink for individual Xe atoms
(trapping of Xe) and subsequently grow in size after absorption. Under
irradiation, the Xe atoms in the gas bubble are reintroduced into the fuel
matrix by fission product-induced cascades and thermal spikes (re-solution of
Xe), and this process reduces the bubble size \cite{ye2023, olander2006re}. The
irradiation-induced re-solution delays the onset of fission gas release and
impacts the final gas release rate.

% 2nd paragraph: accepted mechanisms of re-solution, their connections
Re-solution of fission gas has two commonly accepted mechanisms: homogeneous
re-solution and heterogeneous re-solution. In the homogeneous re-solution
mechanism, the atoms from the gas bubbles are ejected one at a time via
collisions with fission products and their recoils. These atomic collision
cascades are mainly controlled by the nuclear stopping power of the material.
In the heterogeneous mechanism, the electronic stopping of the energetic
particles first raises the electronic subsystem temperature. The deposited
energy in the electronic subsystem can then be transferred to the lattice as
thermal energy via electron-phonon coupling. Subsequently, the rise in local
temperature above the melting temperature of the lattice destroys the gas
bubble structure and re-solves Xe atoms in the process \cite{olander2006re,
huang2010md, govers2012, matthews2015diss, setyawan2018}.

% 3rd paragraph: PKA, collision cascades, subcascades, channeling
In molecular dynamics (MD) simulations of homogeneous re-solution, a regular
lattice atom is typically given high kinetic energy to emulate a primary
knock-on atom (PKA). The PKA then interacts ballistically with other atoms to
initiate a collision cascade. The objective is to create collision cascades
near the gas bubble to cause maximum disorder \cite{parfitt2008, govers2012}.
Another MD approach is to simulate only part of the cascade (also termed a
subcascade) by imparting energy to a random gas atom residing in the bubble.
This avoids simulating parts of the cascade that may not influence the
re-solution process \cite{schwen2009md}. One issue with MD simulations is the
channeling of PKAs or their recoils over long distances without any collisions
\cite{jarrin2021}. Thus, gathering statistics of the interactions of the PKAs
with the gas bubbles can be computationally challenging if the PKA direction is
random. A possible workaround is to direct the PKAs toward a high index lattice
direction \cite{stoller2000}. MD simulations of PKAs also require large
simulation supercells to avoid the self-interaction of cascades through their
periodic images.

% 4th paragraph: thermal spike, electron-phonon coupling
The thermal spike is a useful model used to describe the interaction between
the fission fragments and the fuel. These interactions primarily occur through
the electronic excitation of the material via a four-step process. First, the
swift heavy ions interact with the electrons of the target materials, and the
deposited energy is accumulated in the electronic subsystem. Second, this
energy is locally redistributed in the electronic subsystem through
electron-electron collisions. Third, the energy from the electrons is
transferred to the lattice via the electron-phonon coupling. Fourth, the energy
is transferred among the atoms, resulting in a rapid rise of lattice
temperature within a cylindrical zone of typically few nm radius, known as the
thermal spike \cite{wang1994, toulemonde2002, patra2019}. In MD, electronic
interactions cannot be treated directly. However, the fourth step described
above can be emulated by raising the temperature of atoms in a cylindrical
region.

% 5th paragraph: what MD work has been done in other materials, UO2/UC, Xe/He
Since the experimental determination of the re-solution rate of the fission gas
is difficult, atomistic modeling provides a feasible approach for determining
the re-solution rate. Atomistic simulations have been used to evaluate the
re-solution rate in other nuclear materials before. In 2008, Parfitt et al.
\cite{parfitt2008} simulated PKAs in UO$_2$ to assess the re-solution of He gas
bubbles. Schwen et al. \cite{schwen2009md} studied homogenous re-solution of Xe
gas bubbles in UO$_2$ using binary collision model and MD simulations in 2009.
One year later, Huang et al. \cite{huang2010md} tried to examine the effect of
thermal spikes on Xe re-solution in UO$_2$. In 2012, Govers et al.
\cite{govers2012} evaluated both PKA and thermal spike models for Xe gas
bubbles in UO$_2$ and also provided a mathematical model for the re-solution
rate. However, the most comprehensive work on Xe gas bubble re-solution in
UO$_2$ came from Setyawan et al. \cite{setyawan2018} in 2018. They reconciled
the inconsistencies in conclusions of the previous works on Xe bubble
re-solution in UO$_2$ and evaluated the re-solution rate as a function of
bubble radius. The heterogeneous re-solution of gas bubbles is found to be the
dominant method of re-solution in UO$_2$. Aside from UO$_2$, re-solution rate
of fission gas bubbles has also been evaluated in UC by Matthews et al.
\cite{matthews2015diss} using binary collision methods. The rationale for not
employing the thermal spike model in UC is that the local heating does not
exceed the melting temperature \cite{matthews2015diss, ronchi1986}.

% 6th paragraph: what we are doing in this study, relation to DART
The Dispersion Analysis Research Tool (DART) developed by Argonne National
Laboratory \cite{ye2023} is a mesoscale code that can calculate fission gas
swelling in $\gamma$U-Mo for various operational conditions. Fission gas bubble
re-solution rate is one of many parameters it needs to model the swelling
behavior. DART employs a simple re-solution model containing a piecewise
function to account for the bubble radius. The parameters in that piecewise
function are a few of many parameters in DART that are calibrated by fitting
the computed swelling value to the experimental data. Therefore, we only have a
crude estimate of the true re-solution rate. A physics-based re-solution rate
of fission gas bubbles will make the swelling calculations of higher length
scale models more rigorous. In this work, we investigate the re-solution of Xe
gas bubble in $\gamma$U-Mo fuel using MD simulations. Both homogeneous and
heterogeneous re-solution mechanisms are investigated by simulating PKAs and
thermal spikes, respectively. The electronic stopping power of representative
fission products is also computed to determine the energy imparted on gas
bubbles by fission events at a certain distance. Finally, an overall
re-solution rate of Xe gas bubbles is evaluated as a function of bubble radius
and fission rate.


\FloatBarrier
\section{Computational details}

The large-scale atomic/molecular massively parallel simulator (LAMMPS) software
package \cite{lammps} has been utilized for the MD work in conjunction with a
U-Mo-Xe angular-dependent potential (ADP) \cite{beelerADP}. This ADP is capable
of describing the body-centered cubic (BCC) phase of $\gamma$U-Mo alloys, and
it accurately reproduces their stable structure, modulus of elasticity, room
temperature density, and melting point. Since, the focus of this work is on
high-energy interactions of PKAs and thermal spike events, the distance between
the ions and atoms can become extremely small. The classical interatomic
potential alone does not provide an accurate description of these interactions.
To address this, the Ziegler, Biersack, and Littmark (ZBL) \cite{ziegler1985}
repulsive potential, which offers a realistic depiction of ion-ion repulsive
interactions, is combined with the ADP for all the simulations.

For both PKA and thermal spike simulations, the system is equilibrated at 400 K
at a pressure of 0 bar in an NPT ensemble using the Nos\'e-Hoover barostat and
thermostat. Using the resulting configurations, spherical bubbles of radius
$R_{bubble}$ are created by removing a stoichiometric number of uranium and
molybdenum atoms and depositing gaseous Xe atoms at a given density within the
resulting void. The system is then further equilibrated in the NPT ensemble.
Finally, simulations are performed using an NVE ensemble with a canonical
sampling thermostat \cite{bussi2007} at the edges of the simulation box to act
as a heat sink. This allows the resulting local heating to slowly dissipate
from the simulation by modeling a constant long-range temperature. Electronic
stopping is incorporated into the simulation by applying a frictional force to
each atom. The frictional force is calculated using the stopping power of the
atom in the fuel. These values are determined using the stopping and range of
ions in matter (SRIM) software \cite{ziegler2010srim}. Often the frictional
force due to electronic stopping is applied to atoms having more than a kinetic
energy threshold. This threshold is often set to 10 eV or double the cohesive
energy for metals \cite{nordlund1998, duffy2006}. In our work, electronic
stopping is applied only when an atom has a kinetic energy of more than 9 eV,
which is twice the U-U cohesive energy of 4.5 eV. The cohesive energy is
evaluated by Beeler et al. \cite{beeler2018disp} using the same ADP potential
used in this work. A variable timestep size is implemented for all simulations
such that the maximum displacement of any atom between two successive timesteps
is less than or equal to 0.01 \r{A}.

A $160 \times 160 \times 160$ \r{A}$^3$ supercell is used for the PKA model.
This amounts to $\sim$8 million atoms in the system. The thermostatted sink is
implemented in all three directions. Figure \ref{fig:struct} shows the
cross-section of such a supercell. Using this method, both the over- and
under-pressurized bubbles are created. Within a 20 \r{A} gas bubble, 200 and
400 Xe atoms are investigated, which correspond to pressures of 120 MPa and 551
MPa, respectively. It is to be noted that the equilibrium pressure of such a
bubble is 333 MPa \cite{beeler2020improved}. The cascade is initiated by
selecting a single PKA, located approximately 5.5 nm from the center of the
bubble, and rescaling the PKA's velocity to a kinetic energy of successively
higher energies (up to 500 keV). These simulations are carried out to model the
first 120 picoseconds of the cascade so that the total temperature of all the
examined systems fall below 600 K. For each bubble pressure and PKA energy, 5
simulations are performed with different initial atomic configuration and PKA
direction in order to gather statistics.

\begin{figure}[ht]
	\centering
	\includegraphics[height=7cm]{images/struct.png}
	\caption{
		Initial configuration for an MD simulation of the PKA. The supercell is
		sliced through the middle to show the 20 \r{A} radius Xe gas bubble
		(black). Red and blue dots represent U and Mo atoms, respectively.
	}
	\label{fig:struct}
\end{figure}

% pressure dependence is currently in progress
% cooldown behavior might be shown in a figure showing re-solved fraction

\begin{figure}[ht]
	\centering
	\begin{subfigure}{0.49\textwidth}
		\centering
		\caption{}
		\includegraphics[height=6cm]{images/temp_time.pdf}
	\end{subfigure}
	
	\begin{subfigure}{0.49\textwidth}
		\centering
		\caption{}
		\includegraphics[height=6cm]{images/resol_time.pdf}
	\end{subfigure}
	\caption{
		(a) System temperature versus time for 40 \r{A} radius Xe gas bubble
		simulations with different thermal spike energies. (b) Number of
		re-solved Xe atoms versus time. The vertical dotted lines represent 450
		ps.
	}
	\label{fig:cooldown}
\end{figure}

For the thermal spike model, the simulated supercell is $120 \times 120 \times
50$ \r{A}$^3$, which holds approximately 1.5 million atoms. The thermal spike
axis is parallel to the side with the shortest dimension. The thermostatted
sink is implemented only in the directions perpendicular to the thermal spike
axis. A Xe gas bubble has been inserted in the middle of the supercell with a
Xe/vacancy ratio of 0.2. According to Beeler et al. \cite{beeler2020improved},
this ratio provides an equilibrium bubble pressure. The atoms present in a
cylindrical zone of 35 \r{A} radius along the thermal spike axis are excited
with high kinetic energy to simulate local heating due to electronic stopping,
via rescaling of the atom velocities. The position of the thermal spike in the
supercell is also varied to account for both on-centered and off-centered
collisions between the gas bubble and thermal spike. For on-centered thermal
spikes, gas bubbles of radii ranging from 0 \r{A} to 40 \r{A} are simulated
with thermal spike energies ranging from 5 keV/nm to 30 keV/nm. For
off-centered thermal spikes, gas bubbles of radii 15 \r{A}, 25 \r{A}, and 35
\r{A} are examined with thermal spike energy of only 15 keV/nm. The thermal
spike simulations are run for at least 450 picoseconds, which is enough for all
systems to cool down below 800 K. Figure \ref{fig:cooldown} shows the system
temperature and the number of re-solved Xe atoms as a function of time for a
few 40 \r{A} radius bubble simulations with various thermal spike energies.
From the figure, it can be observed that the number of re-solved atoms
converges to a stable value well before 450 picoseconds. Also, a few longer
simulations are performed to allow the systems to cool down below 500 K for
further verification. The re-solution behavior does not change during the
cooldown from 800 K to 500 K. Thus, a minimum simulation time of 450
picoseconds is chosen to optimize the use of computing resources.

Cluster analysis is performed on the Xe atoms in the system using OVITO
\cite{ovito}. The cutoff distance among clusters is chosen to be 10 \r{A}. In
the beginning, all the Xe atoms in the system form a single cluster, which
corresponds to the initially created Xe bubble. After the introduction of PKAs
or thermal spikes, more than one Xe cluster can be identified due to the
re-solved Xe atoms. The cluster with the largest number of Xe atoms is
considered the original Xe gas bubble and all Xe atoms in the other clusters
are defined as the re-solved atoms. In case the original bubble has a single
remaining atom, the gas bubble is considered to be re-solved completely.


\FloatBarrier
\section{Results}

\FloatBarrier
\subsection{PKA}

In the PKA simulations, re-solution is very limited. Figure 2 presents
snapshots of one such simulation. The black spheres represent Xe atoms, while
red and blue spheres represent U and Mo atoms, respectively. The energy
imparted into the system by the introduction of the PKA propagates toward the
supercell boundary as a shock wave. This shock wave creates many point defects,
the majority of which eventually annihilate. The Xe gas bubble deforms
immediately following the PKA initiation. However, the gas bubble quickly
returns to a stable configuration. In the PKA simulations, at most one
re-solved Xe atom is observed, with most showing no re-solution at all.
Therefore, homogeneous re-solution is considered negligible in the examined
energy range of the PKAs (up to 500 keV).

% Explain why PKA model doesn't show any re-solution.
% Need to talk about subcascades
% How much of the FP energy goes to ballistic collisions? 5%. Cite that paper.
% We are not simulating swift heavy ions because we'd need ttm.
% There can be channeling. There are boundary issues. Also shock wave issues.
% Hard to tell what the imparted energy into the bubble is.
% We need a huge simulation box and that's expensive. We'd need randomized
% bubble and fission product location. The boundary constraint is also
% problematic. At that point, it'd be a holistic representation of the real
% phenomenon. Address this later.

\begin{figure}[ht]
	\centering
	\begin{subfigure}{0.45\textwidth}
		\centering
		\caption{}
		\includegraphics[width=\textwidth]{images/pka1.png}
	\end{subfigure}
	\begin{subfigure}{0.45\textwidth}
		\centering
		\caption{}
		\includegraphics[width=\textwidth]{images/pka2.png}
	\end{subfigure}

	\begin{subfigure}{0.45\textwidth}
		\centering
		\caption{}
		\includegraphics[width=\textwidth]{images/pka3.png}
	\end{subfigure}
	\begin{subfigure}{0.45\textwidth}
		\centering
		\caption{}
		\includegraphics[width=\textwidth]{images/pka4.png}
	\end{subfigure}
	\caption{
		Snapshots of a PKA (500 keV) simulation at (a) 0 ps, (b) 1.5 ps, (c)
		44.5 ps, and (d) 114.5 ps. Xe atoms are shown in black along with U
		(red) and Mo (blue) atoms.
	}
	\label{fig:pka}
\end{figure}


\FloatBarrier
\subsection{On-centered thermal spike}

Contrarily, the thermal spike model demonstrated significant re-solution of Xe
atoms. Figure \ref{fig:spike} showcases a few snapshots of an on-centered
thermal spike simulation. The thermal spike forms a cylindrical volume of
liquid that engulfs the entire Xe gas bubble, causing it to disintegrate.
However, the region cools down significantly after a few picoseconds. While
many Xe atoms gather into a large gas bubble upon cooling, a considerable
number of Xe atoms remain in the lattice or form other smaller bubbles. The
thermal spikes also generate shock waves that propagates radially to the
boundary and create point defects in the system. The number of point defects
decreases with cooling through the sink. Since the re-solution observed in the
thermal spike model is substantially greater than in the PKA model, we only
consider the heterogeneous re-solution of Xe atoms in subsequent calculations.

\begin{figure}[ht]
	\centering
	\begin{subfigure}{0.45\textwidth}
		\centering
		\caption{}
		\includegraphics[width=\textwidth]{images/spike1.png}
	\end{subfigure}
	\begin{subfigure}{0.45\textwidth}
		\centering
		\caption{}
		\includegraphics[width=\textwidth]{images/spike2.png}
	\end{subfigure}

	\begin{subfigure}{0.45\textwidth}
		\centering
		\caption{}
		\includegraphics[width=\textwidth]{images/spike3.png}
	\end{subfigure}
	\begin{subfigure}{0.45\textwidth}
		\centering
		\caption{}
		\includegraphics[width=\textwidth]{images/spike4.png}
	\end{subfigure}
	\caption{
		Snapshots of a thermal spike (30 keV/nm) simulation at (a) 0 ps, (b)
		1.5 ps, (c) 44.5 ps, and (d) 114.5 ps. Xe atoms are shown in black
		along with U (red) and Mo (blue) atoms.
	}
	\label{fig:spike}
\end{figure}

From the on-centered thermal spike simulations for different bubble sizes,
re-solution data is gathered and plotted in Figure \ref{fig:resol}(a) as a
fraction of re-solved Xe atoms against the effective energy transferred to the
lattice, $S_{e,eff}$. Thermal spikes re-solve a greater fraction of Xe atoms in
the smaller bubbles compared to the larger ones. Also, the fraction of
re-solved Xe atoms seems to saturate with increasing deposited energy. Thus, an
exponentially saturating function like the following is used to model the
available data.
\begin{align}
	\chi_0 &= 1 - \exp[-\alpha S_{e,eff}]
\end{align}
where $\chi_0$ is the fraction of re-solved atoms due to an on-centered thermal
spike, and $\alpha$ is the saturation factor. The saturation factors for
different bubble sizes are plotted in \ref{fig:resol}(b) against bubble radius.
The graph suggests there is an inverse proportionality between the saturation
factor and the bubble size, which means the larger bubbles are more difficult
to re-solve completely. To express this relationship, the saturation factor is
made an inverse power function of bubble radius as follows:
\begin{align}
	\alpha &= \frac{5.1}{R_{bubble}^{2.2}}
\end{align}

\begin{figure}[ht]
	\centering
	\begin{subfigure}{0.69\textwidth}
		\centering
		\caption{}
		\includegraphics[height=6cm]{images/resolutionVsRadius.pdf}
	\end{subfigure}
	\begin{subfigure}{0.3\textwidth}
		\centering
		\caption{}
		\includegraphics[height=6cm]{images/saturationFactor.pdf}
	\end{subfigure}
	\caption{
		(a) Fraction of re-solved Xe atoms as a function of the energy
		deposited to the lattice. (b) Saturation factor as a function of bubble
		radius.
	}
	\label{fig:resol}
\end{figure}


\FloatBarrier
\subsection{Off-centered thermal spike}

% What is the thermal spike energy?
% Discuss bubble movement for off-centered thermal spikes
% maybe include some data as well

The results discussed so far pertain to on-centered thermal spikes. It is
expected that off-centered thermal spikes will have diminishing effects on
re-solution, as smaller portion of the Xe gas bubble would be enveloped by the
initial thermal spike region. Furthermore, all re-solution is anticipated to
cease when the initial cylindrical region of the thermal spike no longer
contacts the bubble surface. Let's define the off-centered distance $r$ as the
distance between the Xe gas bubble center and the cylindrical axis of the
thermal spike. The farthest distance at which the thermal spike contacts the
bubble can be denoted as $r_{c} \equiv R_{bubble} + R_{spike}$. Simulations
were conducted with three different bubble sizes (15 \r{A}, 25 \r{A}, and 35
\r{A}) using off-centered thermal spikes. The off-centered distances ranged
from 0 \r{A} to $r_{c}$ \r{A} at an interval of 10 \r{A}.

To compare the results from different bubble sizes, both the Xe re-solution
fraction and the off-centered distance were normalized. The fraction of
re-solved Xe atoms $\chi$ is normalized using the fraction of re-solved Xe
atoms for on-centered thermal spike $\chi_0$, and the off-centered distance $r$
is normalized using the $r_{c}$, which varies for different bubble sizes. The
normalization allows for the comparison of re-solution data from different
bubble sizes. Figure \ref{fig:off} depicts the normalized fraction of re-solved
atoms $\chi/\chi_0$ as a function of $r/r_c$, which highlights how the effect
of the relative off-centered distance of the thermal spike is similar for
bubbles of different sizes. The data was fitted to a logistic function due to
the plateaus observed at both ends of the off-centered distance data. The
fitted equation is as follows:
\begin{align}
	\label{eq:off}
	\frac{\chi}{\chi_0}
	&= \frac{1.058}{1 + \exp \big[8.168 \big(\frac{r}{r_c}\big) - 3.331 \big]}
\end{align}

\begin{figure}[ht]
	\centering
	\includegraphics[width=8cm]{images/offcentered.pdf}
	\caption{
		Re-solution due to 15 keV/nm off-centered thermal spikes.
	}
	\label{fig:off}
\end{figure}


\FloatBarrier
\subsection{Re-solution rate}

% write how we are ignoring the PKA results

\begin{figure}[ht]
	\centering
	\includegraphics[height=6cm]{images/coordSystem.pdf}
	\caption{
		Cylindrical coordinate system for calculating the heterogeneous
		re-solution rate.
	}
	\label{fig:coord}
\end{figure}

To represent the overall re-solution behavior, the contributions from all the
fission products, originating at different distances from a bubble and oriented
toward a random direction, need to be summed up by means of a volume integral.
Consider a cylindrical coordinate system where the gas bubble is at the origin.
To make calculations easier, fission product origins can be rotated around the
coordinate system origin so that all fission tracks (and the thermal spikes
they create) point in the same direction. Given the fission product generation
is uniform and isotropic in the material, the aforementioned rotational
transformation will lead to a uniform distribution of the fission products that
are unidirectional. The axial coordinate $x$ is then defined to be parallel to
the fission tracks and the radial coordinate $r$ perpendicular to the tracks.
The behavior with respect to the azimuth is constant because the fission
products are unidirectional after the transformation. If we denote the fission
rate as $\dot{F}$, the number of fission events per second in an infinitesimal
volume $dV = 2 \pi dr dx$ would be $\dot{F} dV$. Each fission product $i$ from
a fission event contributes $\chi_i \dot{F} dV$ to the total re-solution of the
bubble. The heterogeneous re-solution rate can then be expressed as follows.
\begin{align}
	b_{het}
	&= \sum_{i=1}^2 \int_V \chi_i \dot{F} dV \\
	&= \sum_{i=1}^2 \int_{x=0}^{\infty} \int_{r=0}^{\infty}
		\chi_i \dot{F} 2 \pi r dr dx \\
	&= \dot{F} \sum_{i=1}^2 \int_{r=0}^{\infty}
		\bigg( \frac{\chi_i}{\chi_{0,i}} \bigg)
		2 \pi r dr \int_{x=0}^{\infty} \chi_{0,i} dx
\end{align}
where it is assumed that a single fission reaction creates two fission
products, thus ignoring ternary fission reactions. Also, the double integral is
decoupled into two one-dimensional integrals since
$\Big(\frac{\chi_i}{\chi_{0,i}}\Big)$ is dependent on $r$ only and $\chi_{0,i}$
on $x$.

Now, we can compute the $r$ integral using equation \ref{eq:off}. The upper
limit of the integral can be changed to $r=r_c$ since any re-solution is
presumed to be nonexistent for $r > r_c$.
\begin{align}
	\int_{r=0}^{r_c} \bigg( \frac{\chi_i}{\chi_{0,i}} \bigg) 2 \pi r dr
	&= 2 \pi r_c^2 \int_{r/r_c=0}^{1}
		\frac{1.058}{1 + \exp \big[8.168 \big(\frac{r}{r_c}\big) - 3.331 \big]}
		\bigg(\frac{r}{r_c}\bigg) d\bigg(\frac{r}{r_c}\bigg) \\
	&\approx 0.225 \pi r_c^2
\end{align}

Thus, heterogeneous the resolution rate $b_{het}$ can be written as
\begin{align}
	b_{het}
	&= \dot{F} \sum_{i=1}^2 \int_{r=0}^{\infty}
		\bigg( \frac{\chi_i}{\chi_{0,i}} \bigg)
		2 \pi r dr \int_{x=0}^{\infty} \chi_{0,i} dx \\
	&= \dot{F} \sum_{i=1}^2 0.225 \pi r_c^2 \int_{x=0}^{\infty}
		[1 - \exp(-\alpha S_{e,eff,i})] dx \\
	\label{eq:dx}
	&= 0.225 \pi r_c^2 \dot{F} \sum_{i=1}^2 \int_{x=0}^{\infty}
		[1 - \exp(-5.1 \zeta S_{e,i} / R_{bubble}^{2.2})] dx 
\end{align}
where $S_{e,i}$ is the electronic stopping power of the fission product $i$,
and $\zeta = S_{e,eff} / S_{e}$ is the fraction of fission product energy that
is imparted to the lattice. In this work, $\zeta$ is assumed to be an unknown
parameter ranging somewhere between 0.5 and 1. The introduction of this
parameter is necessary to account for the fraction of the fission product
energy that is not transferred to the thermal spike.

We choose to evaluate the $S_e$ values for Xe-140 and Sr-94 in $\gamma$U-10Mo
based on the fission reaction $_0^1n + _{92}^{235}U \rightarrow _{54}^{140}Xe +
_{38}^{94}Sr + Q$. Even though it is one of many possible fission reactions
that may take place, Xe-140 and Sr-94 can still be considered as
representatives of heavy and light fission products. The rationale behind this
is that the two aforementioned species appear close to the two peaks of a
typical fission product yield graphs (citation needed).

$S_{e,Xe}$ and $S_{e,Sr}$ as a function of the distance traversed by the
fission product is shown in Figure \ref{fig:irad}. This data is generated using
the SRIM software. The following model is fit to represent the $S_e$ data.
\begin{align}
	\label{eq:xe}
	S_{e,Xe} &= 21.3 \exp(-0.239 x^{1.78})
		+ 5.23 \exp(-4.67 \times 10^{-8} x^{11}) \\
	\label{eq:sr}
	S_{e,Sr} &= 19.7 \exp(-0.00273 x^{3.71})
		+ 6.8 \exp(-0.424 x^{1.45})
\end{align}

\begin{figure}[ht]
	\centering
	\includegraphics[width=8cm]{images/iradina.jpg}
	\caption{
		Total electronic stopping power ($S_e$) of Xe-140 and Sr-94 in
		$\gamma$U-10Mo calculated with the SRIM software as a function of
		distance traversed by the fission product from the location of the
		fission reaction.
	}
	\label{fig:irad}
\end{figure}

Finally, the re-solution rate can be calculated for different values of $\zeta$
and bubble radius $R_{bubble}$ using equations \ref{eq:dx}, \ref{eq:xe}, and
\ref{eq:sr}. Figure \ref{fig:res} displays the result from this work. The
re-solution rates are computed using numerical integration, and the data can be
extracted from the associated graphs or by recalculating the integrals.
However, an approximate analytical function might serve the higher length scale
models better. To that end, we propose the following function for fitting the
computed re-solution rate.
\begin{align}
	\label{eq:param}
	b_{het} &= \bigg[ \frac{k}{1 + (R_{bubble} / c)^d} \bigg]
		10^{-14} \dot{F}
\end{align}
where $R_{bubble}$ is in \r{A}, $\dot{F}$ is in fissions per cubic centimeter
per second (fiss/cm$^3$/s), and $k$, $c$ and $d$ are adjustable parameters. The
parameter values for different $\zeta$ are listed in Table 1. The parameter $d$
remains consistent up to three decimal places for all $\zeta$. To determine the
re-solution rate of any arbitrary $\zeta$ between 0.55 and 1, linear
interpolation would be sufficient.

\begin{figure}[ht]
	\centering
	\includegraphics[height=7cm]{images/resRate.pdf}
	\caption{
		Xe gas bubble re-solution rate in $\gamma$U-10Mo as a function of
		bubble radius at a fission rate of $10^{14}$ fiss/cm$^3$/s.
	}
	\label{fig:res}
\end{figure}

\begin{table}[ht]
\centering
\caption{Equation \ref{eq:param} parameter fits for different $\zeta$ values.}
\label{tab:param}
\begin{tabular}{llll}
\toprule
$\zeta$     & $k$        & $c$     & $d$      \\
\midrule
0.55        & 0.0167     & 2.717   & 1.225    \\
0.7         & 0.0162     & 3.412   & 1.225    \\
0.85        & 0.0159     & 4.073   & 1.225    \\
1           & 0.0157     & 4.714   & 1.225    \\
\bottomrule
\end{tabular}
\end{table}

The re-solution rate is a crucial parameter for calculating fission gas
swelling in $\gamma$U-10Mo. The mesoscale program Dispersion Analysis Research
Tool (DART) incorporates this parameter for such calculations \cite{ye2023}.
The currently used re-solution rate $b_{dart}$ in DART is defined as follows.
\begin{align}
	b_{dart} &= b_0 \cdot \dot{F} \cdot G \\
	b_0 &= R_{spike}^2 \cdot \mu_{ff} \\
	G &=
	\begin{cases}
		1 & ,R_{bubble} \leq \lambda \\
		1 - (\frac{R_{bubble}-R_{resol}}{R_{bubble}})^3
		  & ,R_{bubble} > \lambda
	\end{cases}
\end{align}
where $b_0$ is the bubble destruction probability, $\dot{F}$ is the fission
rate, and $G$ is a piecewise function representing different resolution modes
for small and large gas bubbles. The parameter $b_0$ can be estimated using the
interaction volume of a thermal spike with bubbles, given by the formula, $b_o
= R_{spike}^2 \times \mu_{ff}$, where $R_{spike}$ is the radius of a thermal
spike and $\mu_{ff}$ is the recoil length of fission fragments. In the
piecewise function G, $R_{bubble}$ is the bubble radius, $\lambda$ is the
gas-atoms knock out distance, and $R_{resol}$ is the thickness of the annulus
within which all gas-atoms are knocked out. The parameters $b_0$, $\lambda$,
and $R_{resol}$ are considered adjustable, and the optimized values are $b_0 =
2 \times 10^{-18}$ cm$^3$, $\lambda = 5 \times 10^{-7}$ cm, and $R_{resol} = 3
\times 10^{-9}$ cm as reported in \cite{ye2023}. Since the parameters $\lambda$
and $R_{resol}$ are not coupled, the re-solution rate is discontinuous at
bubble radius $\lambda$, which is unphysical. Figure \ref{fig:dart} shows the
re-solution rate for bubble radius up to 100 \r{A} as used in DART along with
the re-solution rate computed in this work. Unlike the DART model prediction,
the re-solution rate computed in this work is a smooth decaying function of
bubble radius. The calculated re-solution rate is about two orders of magnitude
higher than the DART prediction for very small bubbles. This difference
increases to about three orders of magnitude for large bubbles. Only for
bubbles of radius around 50 \r{A}, the difference is less than one order of
magnitude. The value of $\zeta$ does not impact this comparison, since the
re-solution rate at $\zeta=1$ is at most double of the rate at $\zeta=0.55$.

\begin{figure}[ht]
	\centering
	\includegraphics[height=7cm]{images/resRate_ext.pdf}
	\caption{
		Comparison of the calculated re-solution rate against DART model
		prediction \cite{ye2023}.
	}
	\label{fig:dart}
\end{figure}


\FloatBarrier
\section{Conclusion}

In this study, MD simulations are utilized to determine the re-solution rate of
Xe gas bubbles in $\gamma$U-10Mo fuel. Both homogeneous and heterogeneous
re-solution processes are simulated using the primary knock-on atom and thermal
spike methods. However, the homogeneous re-solution is found to be negligible
in describing the overall re-solution behavior. The fraction of re-solved Xe
gas bubble atoms appears to be invariant with respect to bubble pressure and Xe
density. This fraction is subsequently used to model the re-solution fraction
as a function of effective energy imparted into the lattice for different
bubble sizes. Simulations of both on-centered and off-centered thermal spikes
are conducted to get the spatial correlation between re-solution and thermal
spike distance. Additionally, the electronic stopping power of fission products
(Xe and Sr) in $\gamma$U-10Mo is simulated to provide distance-dependent
electronic stopping power. All this information is then used to calculate the
overall re-solution rate of Xe gas bubbles in the fuel. The re-solution rate of
Xe gas bubble atoms in $\gamma$U-10Mo is found to be on the order of $10^{-4}$
to $10^{-2}$ s$^{-1}$ for a fission rate of $10^{14}$ fissions/cm$^3$/s. An
analytical form of re-solution rate dependent on bubble radius, fission rate,
and $\zeta$ is also provided. The result from this work will inform higher
length scale models of $\gamma$U-10Mo with a physics-based description of Xe
gas bubble re-solution rate.


\bibliographystyle{unsrt}
\bibliography{ref.bib}

\end{document}

\documentclass{article}
\usepackage[margin=1in]{geometry}
\usepackage{authblk}
\usepackage{graphicx, xcolor, placeins}
\usepackage{amsmath, booktabs, subcaption}

\title{Re-solution of Xe gas bubbles in the U-10Mo fuel}
\author[1]{ATM Jahid Hasan}
\author[2]{Linu Malakkal}
\author[1,2]{Benjamin Beeler}
\author[2]{Mathew Swisher}
\affil[1]{North Carolina State University, Raleigh, NC}
\affil[2]{Idaho National Laboratory, Idaho Falls, Idaho}

\begin{document}
\maketitle


\begin{abstract}
	Molecular dynamics simulations are performed to quantify the Xe gas bubble
	re-solution rate in U-10Mo fuel. PKA and thermal spike models are employed
	to evaluate homogeneous and heterogeneous re-solution, respectively. It is
	found that the homogeneous re-solution is negligible compared to the
	heterogeneous re-solution. Thermal spike simulations are thus performed for
	Xe gas bubbles of different radii ($R_{bubble}$). Thermal spike axis is
	also varied so that both on-centered and off-centered collisions with the
	gas bubble are accounted for. Subsequently, the probability of Xe gas atoms
	getting re-solved into the U-10Mo matrix is evaluated as a function of
	bubble radius ($R_{bubble}$), off-centered distance ($r$), and thermal
	spike energy ($S_{e,eff}$). The electronic stopping power ($S_e$) of
	fission products are also simulated to find the energy imparted by the
	fission products at a certain distance. A fraction ($\zeta$) of the energy
	imparted by fission products is equated to the thermal spike energy ($\zeta
	S_e$). Eventually, all the information is used to compute an overall Xe gas
	bubble re-solution rate in the U-10Mo fuel as a function of bubble radius
	for reasonable $\zeta$ values.
\end{abstract}


\section{Introduction}

------------------------------------
{\color{cyan} Intro is incomplete}
------------------------------------

A U-Mo alloy based monolithic fuel design has been identified as the fuel type
for conversion of the United States High-Performance Research Reactors (HPRRs).
However, the U-Mo monolithic fuel exhibits excessive swelling during operation.
To understand the fuel evolution under irradiation, mesoscale and engineering
level fuel performance models require the knowledge of fundamental mechanistic
behavior of fission products in the fuel. As such, understanding the
progression of Xe gas bubbles in the fuel is crucial. The Xe gas bubbles act as
a sink for individual Xe atoms (trapping of Xe) and subsequently grow in size
after absorption. Under irradiation, the Xe atoms in the gas bubble are
re-solved back into the fuel matrix, and this process reduces the bubble size.
In this work, we investigate the re-solution of xenon gas bubble in U-Mo fuel
with the molecular dynamics simulations. The efficiency of both the homogeneous
and heterogeneous resolution mechanism are investigated via the primary knock
on atom and thermal spike methods, respectively. For low energy interactions,
the collision cascades caused by a single fast moving primary knock on atom
were studied. The higher energy range was studied by the thermal spike method.
The study sheds light on the atomic scale processes at play for fission gas
re-solution in nuclear fuel. The present atomic scale simulations indicate that
low energy interactions are not effective in the resolution process. The high
energy interactions destroy smaller bubbles completely. The re-solution rate
computed in this work, combined with the trapping rate of Xe, provides a full
picture of the evolution of Xe gas bubbles in U-Mo.

To understand the fuel evolution under irradiation, mesoscale and engineering
level fuel performance models require the knowledge of fundamental mechanistic
behavior of fission products in the fuel. As such, understanding the
progression of Xe gas bubbles in the fuel is crucial. The Xe gas bubbles act as
a sink for individual Xe atoms (trapping of Xe) and subsequently grow in size
after absorption. Under irradiation, the Xe atoms in the gas bubble are
re-solved back into the fuel matrix, and this process reduces the bubble size
\cite{ye2023}. In this work, we investigate the re-solution of xenon gas bubble
in U-Mo fuel with the molecular dynamics simulations. The efficiency of both
the homogeneous and heterogeneous resolution mechanism are investigated via the
primary knock on atom and thermal spike methods, respectively.

The thermal spike method is a useful model used to describe the interaction
caused by the swift heavy ions on a target material. These interactions occur
mainly through electronic excitation of the material through a four-step
process. During the timescale of $10^{-17}-10^{-16}$ s, the swift heavy ions
interact with the electrons of the target materials and the deposited energy is
accumulated in the electronic subsystem \cite{toulemonde2002, wang1994}.
Secondly, this energy is redistributed locally in the electronic subsystem in
about $10^{-15}-10^{-14}$ s via electron-electron collisions, heating up the
'cold' electrons. At around $10^{-13}-10^{-12}$ s, the energy from the
electrons is transferred to the lattice via the electron-phonon coupling.
Finally, the energy is transferred among the atoms resulting in a rapid rise of
lattice temperature ($\approx$10$^{4}$ K) in a cylindrical zone of typically
few nm radius called the thermal spike \cite{patra2019}.

The re-solution rate calculated in this work is compared with the rate defined
by the model used in DART \cite{ye2023}. The associated model is described
below.
\begin{align}
	b_{dart} &= b_0 \cdot \dot{F} \cdot G \\
	b_0 &= R_{spike}^2 \cdot \mu_{ff} \\
	G &=
	\begin{cases}
		1 & ,R_{bubble} \leq \lambda \\
		1 - (\frac{R_{bubble}-R_{resol}}{R_{bubble}})^3
		  & ,R_{bubble} > \lambda
	\end{cases}
\end{align}
where $b_0$ is the bubble destruction probability, $\dot{F}$ is the fission
rate, and $G$ is a piecewise function representing different resolution modes
for small and large gas bubbles. $b_0$ can be estimated with the interaction
volume of a thermal spike with bubbles, $b_o = R_{spike}^2 \times \mu_{ff}$,
where $R_{spike}$ is the radius of a thermal spike and $\mu_{ff}$ is the recoil
length of fission fragments. In the piecewise function G, $R_{bubble}$ is the
bubble radius, $\lambda$ is the gas-atoms knock out distance, and $R_{resol}$
is the thickness of the annulus within which all gas-atoms are knocked out.

\begin{figure}[ht]
	\centering
	\includegraphics[width=10cm]{images/onlyDart.pdf}
	\caption{Re-solution rate used in the DART model \cite{ye2023}.}
	\label{fig:dart}
\end{figure}


\FloatBarrier
\section{Computational details}

Two models of re-solution are implemented: primary knock-on atom (PKA) and
thermal spike. Re-solution through PKAs happens via atomic collision cascades
(i.e., due to the nuclear stopping). Re-solution through thermal spikes occurs
due to the electronic stopping, which raises the temperature of the fission
track to a value that is higher than the melting temperature of the material.
For both of these implementations, the LAMMPS software package \cite{lammps}
has been used with a U-Mo-Xe angular-dependent potential (ADP) \cite{beelerADP}
for the MD simulations. This potential is capable of describing the
body-centered cubic phase of U-Mo alloys and is able to reproduce the stable
structure, modulus of elasticity, room temperature density, and melting point.
Since this work focuses on high-energy interactions such as primary knock-on
atom (PKA) and thermal spike events, the distance between the ions and atoms
can become very small, these interactions are not described accurately by the
classical interatomic potential alone. Therefore, the repulsive potential of
Ziegler, Biersack, and Littmark (ZBL) \cite{ziegler1985}, which provides a
realistic description of the ion-ion repulsive interactions, is combined
with the ADP potential for all the simulations.

For both PKA and thermal spike simulations, the system is equilibrated at 400 K
at a pressure of 0 bar in an NPT ensemble using the Nos\'e-Hoover barostat and
thermostat. Using the resulting configurations, spherical bubbles of radius
$R_{bubble}$ are created by removing a stoichiometric number of uranium and
molybdenum atoms and depositing gaseous Xe atoms at a given density within the
resulting void. The system is then further equilibrated in the NPT ensemble.
Finally, simulations are performed using an NVE ensemble with a canonical
sampling thermostat \cite{bussi2007} at the edges of the simulation box to act
as a heat sink. This allows the resulting shock-wave emission and local heating
to be contained within the cell. This also prevents interactions with the
periodic image of the system across the boundaries. Electronic stopping is
incorporated into the simulation by applying a frictional force to each atom.
The frictional force is calculated using the stopping power of the atom in the
fuel. These values are determined using the SRIM software
\cite{ziegler2010srim}. Often the frictional force due to electronic stopping
is applied to atoms having more than a kinetic energy threshold. This threshold
is often set to 10 eV or double the cohesive energy for metals
\cite{nordlund1998, duffy2006}. In our work, electronic stopping is applied
only when an atom has a kinetic energy of more than 9 eV, which is twice the
U-U cohesive energy of 4.5 eV. The cohesive energy is evaluated by Beeler et
al. using the same ADP potential used in this work \cite{beeler2018disp}.
A variable timestep is implemented for all simulations such that the maximum
displacement of any atom between two successive timesteps is less than or
equal to 0.01 \r{A}.

A $160 \times 160 \times 160$ \r{A}$^3$ supercell is used for the PKA model.
This amounts to $\sim$8 million atoms in the system. The thermostatted sink is
implemented in all three directions. Figure \ref{fig:struct} shows the
cross-section of such a supercell. Using this method, both the over- and
under-pressurized bubbles are created. Within a 20 \r{A} gas bubble, 200 and
400 Xe atoms are investigated, which correspond to pressures of 120 MPa and 551
MPa, respectively. It is to be noted that the equilibrium pressure of such a
bubble is 333 MPa \cite{beeler2020improved}. The cascade is initiated by
selecting a single PKA, located approximately 5.5 nm from the center of the
bubble, and scaling its velocity to a kinetic energy of successively higher
energies (up to 500 keV). These simulations are carried out for 120
picoseconds so that the total temperature of all the examined systems fall
below 600 K. For each bubble pressure and PKA energy, 5 simulations are
performed with different initial atomic configuration and PKA direction in
order to gather statistics.

\begin{figure}[ht]
	\centering
	\includegraphics[height=7cm]{images/struct.png}
	\caption{Initial configuration for an MD simulation of the PKA. The
	supercell is sliced through the middle to show the Xe gas bubble
	(black). Red and blue dots represent U and Mo atoms, respectively.}
	\label{fig:struct}
\end{figure}

For the thermal spike model, the simulated supercell is $120 \times 120 \times
50$ \r{A}$^3$, which holds $\sim$1.5 million atoms. The thermal spike axis is
parallel to the side with the shortest dimension. The thermostatted sink is
implemented only in the directions perpendicular to the thermal spike axis. A
Xe gas bubble has been inserted in the middle of the supercell with a
Xe/vacancy ratio of 0.2. According to Beeler et al., this ratio provides an
equilibrium bubble pressure \cite{beeler2020improved}. The atoms present in a
cylindrical zone of 35 \r{A} radius along the thermal spike axis are excited
with a high kinetic energy to simulate local heating due to electronic
stopping. A total energy of up to 30 keV/nm is imparted to the cylindrical
region of the thermal spike. The position of the thermal spike in the supercell
is varied to account for both on-centered and off-centered collisions between
the gas bubble and thermal spike. For on-centered thermal spikes, gas bubbles
of radii randing from 0 \r{A} to 40 \r{A} are simulated. For off-centered
thermal spikes, gas bubbles of radii 15 \r{A}, 25 \r{A}, and 35 \r{A} are
examined. The thermal spike simulations are run for at least 450 picoseconds,
which is enough for all systems to cool down below 800 K. It is verified by
running a few longer timescale simulations that the re-solution behavior does
not change if the system is allowed to cool further below 800 K. Thus, 800 K is
chosen as the cutoff temperature for thermal spike simulations to optimize the
use of computing resources.

Cluster analysis is performed on the Xe atoms in the system using OVITO
\cite{ovito}. The cutoff distance among clusters is chosen to be 10 \r{A}. The
cluster with the largest number of Xe atoms is considered as the original
bubble and all the other atoms are defined as the re-solved atoms. One edge
case arises when the largest cluster has only one atom. To account for that,
the whole bubble is deemed to be re-solved if the largest cluster consists of
only one atom. Handling this edge case is important because the smallest
bubbles can get re-solved completely.


\FloatBarrier
\section{Results and Discussion}

% Address the pressure issue later. The re-solution data for both the
% under-pressurized (200 Xe atoms) and the over-pressurized (400 Xe atoms)
% bubble are analyzed. From observing the data, it can be hypothesized that the
% fraction of re-solved atoms is independent of the bubble pressurization (or
% Xe density).

There is very limited re-solution in the PKA simulations. Only a few of them
showed a single re-solved Xe atom in the system. Snapshots of one such
simulation are shown in Figure \ref{fig:pka}. The black spheres represent Xe
atoms, and red and blue spheres represent U and Mo interstitials respectively.
The imparted energy into the system by the introduction of the PKA propagates
toward the supercell boundary as a shock wave. The shock wave creates many
point defects, the majority of which eventually annihilate. The Xe gas bubble
gets deformed in the immediate aftermath of the PKA initiation. However, the
gas bubble quickly goes back to a stable configuration. In the PKA simulations,
at most one re-solved Xe atom is observed, with most showing no re-solution at
all. Therefore, the homogeneous re-solution is deemed negligible in the
examined energy range of the PKAs (up to 500 keV).

% Explain why PKA model doesn't show any re-solution.
% We are not simulating swift heavy ions because we'd need ttm.
% Where to discuss this? Methods or results?
% What would happen if we did simulate a 70 MeV PKA?
% There can be channeling. There are boundary issues. Also shock wave issues.
% Hard to tell what the imparted energy into the bubble is.
% We need a huge simulation box and that's expensive. We'd need randomized
% bubble and fission product location. The boundary constraint is also
% problemnatic. At that point, it'd be a holistic representation of the real
% phenomenon. Address this later.

\begin{figure}[ht]
	\centering
	\begin{subfigure}{0.45\textwidth}
		\centering
		\caption{}
		\includegraphics[width=\textwidth]{images/pk1.png}
	\end{subfigure}
	\begin{subfigure}{0.45\textwidth}
		\centering
		\caption{}
		\includegraphics[width=\textwidth]{images/pk2.png}
	\end{subfigure}

	\begin{subfigure}{0.45\textwidth}
		\centering
		\caption{}
		\includegraphics[width=\textwidth]{images/pk3.png}
	\end{subfigure}
	\begin{subfigure}{0.45\textwidth}
		\centering
		\caption{}
		\includegraphics[width=\textwidth]{images/pk4.png}
	\end{subfigure}
	\caption{Snapshots of a PKA (500 keV) simulation at (a) 0 ps, (b) 1.5
	ps, (c) 44.5 ps, and (d) 114.5 ps. Xe atoms are shown in black along
	with U (red) and Mo (blue) interstitials.}
	\label{fig:pka}
\end{figure}

On the other hand, the thermal spike model showed significant re-solution of Xe
atoms. Figure \ref{fig:spike} displays a few snapshots of a on-centered thermal
spike simulation. The thermal spike creates a cylindrical liquid zone that
engulfs the entire Xe gas bubble and breaks it apart. However, the region cools
down signficantly after a few picoseconds. While many Xe atoms congregate into
a large gas bubble upon cooldown, a substantial amount of Xe atoms remain in
the lattice or form other smaller bubbles. The thermal spikes also create shock
waves that propagates radially to the boundary and generate point defects in
the system. The number of point defects reduces with cooling through the sink.
Since the re-solution observed in the thermal spike model is way higher than in
the PKA model, we only consider heterogeneous re-solution of Xe atoms in
subsequent calculations.

\begin{figure}[ht]
	\centering
	\begin{subfigure}{0.45\textwidth}
		\centering
		\caption{}
		\includegraphics[width=\textwidth]{images/ev1.png}
	\end{subfigure}
	\begin{subfigure}{0.45\textwidth}
		\centering
		\caption{}
		\includegraphics[width=\textwidth]{images/ev2.png}
	\end{subfigure}

	\begin{subfigure}{0.45\textwidth}
		\centering
		\caption{}
		\includegraphics[width=\textwidth]{images/ev3.png}
	\end{subfigure}
	\begin{subfigure}{0.45\textwidth}
		\centering
		\caption{}
		\includegraphics[width=\textwidth]{images/ev4.png}
	\end{subfigure}
	\caption{Snapshots of a thermal spike (30 keV/nm) simulation at (a) 0 ps,
	(b) 1.5 ps, (c) 44.5 ps, and (d) 114.5 ps. Xe atoms are shown in black
	along with U (red) and Mo (blue) interstitials.}
	\label{fig:spike}
\end{figure}

From on-centered thermal spike simulations for different bubble sizes,
re-solution data is gathered and plotted in Figure \ref{fig:resol}(a) as
fraction of re-solved Xe atoms against effective energy transferred to the
lattice, $S_{e,eff}$. Thermal spikes can re-solve higher fractions of smaller
bubbles than of larger bubbles. Also, the fraction of re-solved Xe atoms seems
to saturate with increasing deposited energy. Thus, an exponentially saturating
function like the following is used to model the available data.
\begin{align}
	\chi_0 &= 1 - \exp[-\alpha S_{e,eff}]
\end{align}
where $\chi_0$ is the fraction of re-solved atoms due to an on-centered thermal
spike, and $\alpha$ is the saturation factor. The saturation factors for
different bubble sizes are plotted in \ref{fig:resol}(b) against bubble radius.
The graph suggests there is an inverse proportionality between the saturation
factor and the bubble size. To express this relationship, the saturation factor
is made an inverse power function of bubble radius as follows.
\begin{align}
	\alpha &= \frac{5.1}{R_{bubble}^{2.2}}
\end{align}

\begin{figure}[ht]
	\centering
	\begin{subfigure}{0.69\textwidth}
		\centering
		\caption{}
		\includegraphics[height=6cm]{images/resolutionVsRadius.pdf}
	\end{subfigure}
	\begin{subfigure}{0.3\textwidth}
		\centering
		\caption{}
		\includegraphics[height=6cm]{images/saturationFactor.pdf}
	\end{subfigure}
	\caption{(a) Fraction of re-solved Xe atoms as a function of the energy
	deposited to the lattice. (b) Saturation factor as a function of bubble
	radius.}
	\label{fig:resol}
\end{figure}

% Discuss bubble movement for off-centered thermal spikes

The results discussed so far are from the on-centered thermal spikes.
Off-centered thermal spikes are expected to have diminishing effects on
re-solution since less portion of the Xe gas bubble would be covered by the
initial thermal spike region, and all re-solution should stop when that initial
cylindrical region of the thermal spike barely touches the bubble surface. Let
us define the off-centered distance $r$ as the distance between the Xe gas
bubble center and the cylindrical axis of the thermal spike. Then the distance
where the thermal spike just touches the bubble can be denoted as $r_{c} \equiv
R_{bubble} + R_{spike}$. Three different bubble sizes (15 \r{A}, 25 \r{A}, and
35 \r{A}) are simulated with off-centered thermal spikes with off-centered
distances ranging from 0 \r{A} to $r_{c}$ \r{A} at an interval of 10 \r{A}.

To compare the results from the different bubble sizes, both Xe re-solution
fraciton and off-centered distance are normalized. Fraction of re-solved Xe
atoms $\chi$ is normalized using the fraction of re-solved Xe atoms for
on-centered thermal spike $\chi_0$, and the off-centered distance $r$ is
normalized using the $r_{c}$, which is different for different bubble sizes.
Such normalization makes it possible to compare the re-solution data from
different bubble sizes. Figure \ref{fig:off} depicts the normalized fraction of
re-solved atoms $\chi/\chi_0$ as a function of $r/r_c$. The data is fitted to a
logistic function since there seems to be plateaus on both ends of the
off-centered distance. The fitted equation is as follows.
\begin{align}
	\label{eq:off}
	\frac{\chi}{\chi_0}
	&= \frac{1.058}{1 + \exp \big[8.168 \big(\frac{r}{r_c}\big) - 3.331 \big]}
\end{align}

\begin{figure}[ht]
	\centering
	\includegraphics[width=10cm]{images/offcentered.pdf}
	\caption{Re-solution due to off-centered thermal spikes.}
	\label{fig:off}
\end{figure}

% cite wahyu et al. profusely

To get the overall re-solution behavior, the contributions from all the fisison
products originating at different distances from a bubble and oriented toward a
random direction need to be summed up by means of a volume integral. Let us
consider a cylindrical coordinate system where the gas bubble is at the origin,
radial coordinate $r$ is perpendicular to the thermal spike axis, and the axial
coordinate $x$ is parallel to the fission track. The azimuth can be ignored in
this case if we assume isotropic production of fission products in the
material. Thus, the source of any fission product can be described by $(r, x)$
coordinates. If we denote the fission rate as $\dot{F}$, the heterogeneous
re-solution rate can then be expressed as follows.
\begin{align}
	b_{het}
	&= \dot{F} \sum_{i=1}^2 \int_{r=0}^{\infty} \int_{x=0}^{\infty}
		2 \pi r \chi_i dr dx \\
	&= \dot{F} \sum_{i=1}^2 \int_{r=0}^{\infty}
		\bigg( \frac{\chi_i}{\chi_{0,i}} \bigg)
		2 \pi r dr \int_{x=0}^{\infty} \chi_{0,i} dx
\end{align}
where the index $i$ denotes the species of the fission product. It is assumed
that a single fission reaction creates two fission products, thus ignoring
ternary fission reactions.

Now, we can compute the $r$ integral using equation \ref{eq:off}. The upper
limit of the integral can be changed to $r=r_c$ since any re-solution is
presumed to be nonexistent for $r > r_c$.
\begin{align}
	\int_{r=0}^{r_c} \bigg( \frac{\chi_i}{\chi_{0,i}} \bigg) 2 \pi r dr
	&= 2 \pi r_c^2 \int_{r/r_c=0}^{1}
		\frac{1.058}{1 + \exp \big[8.168 \big(\frac{r}{r_c}\big) - 3.331 \big]}
		\bigg(\frac{r}{r_c}\bigg) d\bigg(\frac{r}{r_c}\bigg) \\
	&\approx 0.225 \pi r_c^2
\end{align}

Thus, heterogeneous the resolution rate $b_{het}$ can be written as
\begin{align}
	b_{het}
	&= \dot{F} \sum_{i=1}^2 \int_{r=0}^{\infty}
		\bigg( \frac{\chi_i}{\chi_{0,i}} \bigg)
		2 \pi r dr \int_{x=0}^{\infty} \chi_{0,i} dx \\
	&= \dot{F} \sum_{i=1}^2 0.225 \pi r_c^2 \int_{x=0}^{\infty}
		[1 - \exp(-\alpha S_{e,eff,i})] dx \\
	\label{eq:dx}
	&= 0.225 \pi r_c^2 \dot{F} \sum_{i=1}^2 \int_{x=0}^{\infty}
		[1 - \exp(-5.1 \zeta S_{e,i} / R_{bubble}^{2.2})] dx 
\end{align}
where $S_{e,i}$ is the electronic stopping power of the fission product $i$,
and $\zeta = S_{e,eff} / S_{e}$ is the fraction of fission product energy that
is imparted to the lattice. In this work, $\zeta$ is assumed to be an unknown
parameter ranging somewhere between 0.5 and 1. The introduction of this
parameter is necessary to account for the fraction of the fission product
energy that is not transferred to the thermal spike.

We choose to evaluate the $S_e$ values for Xe-140 and Sr-94 in U-10Mo based on
the fission reaction $_0^1n + _{92}^{235}U \rightarrow _{54}^{140}Xe +
_{38}^{94}Sr + Q$. Even though it is one of many possible fission reactions
that may take place, Xe-140 and Sr-94 can still be considered as
representatives of heavy and light fission products. The rationale behind this
is that the two aforementioned species appear close to the two peaks of a
typical fission product yield graphs (citation needed).

$S_{e,Xe}$ and $S_{e,Sr}$ as a function of the distance traversed by the
fission product is shown in Figure \ref{fig:irad}. This data is generated using
the Iradina software \cite{crocombette2019irad}. The following model is fit to
represent the $S_e$ data.
\begin{align}
	\label{eq:xe}
	S_{e,Xe} &= 21.3 \exp(-0.239 x^{1.78})
		+ 5.23 \exp(-4.67 \times 10^{-8} x^{11}) \\
	\label{eq:sr}
	S_{e,Sr} &= 19.7 \exp(-0.00273 x^{3.71})
		+ 6.8 \exp(-0.424 x^{1.45})
\end{align}

\begin{figure}[ht]
	\centering
	\includegraphics[width=10cm]{images/iradina.jpg}
	\caption{Total electronic stopping power ($S_e$) of Xe-140 and Sr-94 in
	U-10Mo calculated with the Iradina software as a function of distance
	traversed by the fission product from the location of the fission
	reaction.}
	\label{fig:irad}
\end{figure}

Finally, the re-solution rate can be calculated for different $\zeta$ and
bubble radius $R_{bubble}$ using equations \ref{eq:dx}, \ref{eq:xe}, and
\ref{eq:sr}. Figure \ref{fig:res} displays the result from this work along with
the prediction from the DART model. Unlike the DART model prediction, the
re-solution rate computed in this work is a smooth decaying function of bubble
radius. The calculated re-solution rate is about two orders of magnitude higher
than the DART prediction for very small bubbles. This difference is about three
orders of magnitude for large bubbles. Only for bubbles of radius around 50
\r{A}, the difference is less than one order of magnitude. The value of $\zeta$
does not impact this comparison since the ratio between the re-solution rate at
$\zeta=1$ is at most double of the rate at $\zeta=0.55$.

\begin{figure}[ht]
	\centering
	\begin{subfigure}{0.49\textwidth}
		\centering
		\caption{}
		\includegraphics[width=\textwidth]{images/resRate.pdf}
	\end{subfigure}
	\begin{subfigure}{0.49\textwidth}
		\centering
		\caption{}
		\includegraphics[width=\textwidth]{images/resRate_ext.pdf}
	\end{subfigure}
	\caption{(a) Xe gas bubble re-solution rate in U-10Mo as a function of
	bubble radius at a fission rate of $10^{14}$ fiss/cm$^3$/s. (b) Comparison
	of the calculated re-solution rate against DART model prediction.}
	\label{fig:res}
\end{figure}

The re-solution rates are computed using numerical integration, and the data
can be extracted from the associated graphs or by recalculating the integrals.
However, an approximate analytical function might serve the higher length scale
models better. To that end, we propose the following function for fitting the
computed re-solution rate.
\begin{align}
	\label{eq:param}
	b_{het} &= \bigg[ \frac{k}{1 + (R_{bubble} / c)^d} \bigg]
		10^{-14} \dot{F}
\end{align}
where $R_{bubble}$ is in \r{A}, $\dot{F}$ is in fiss/cm$^3$/s, and $k$, $c$ and
$d$ are adjustable parameters. Table \ref{tab:param} lists the parameter values
for different $\zeta$. Note that the parameter $d$ is the same for up to 3
decimal places for all $\zeta$. Linear interpolation would be adequate to get
the re-solution rate for any arbitrary $\zeta$ between 0.55 and 1.

\begin{table}[ht]
\centering
\caption{Equation \ref{eq:param} parameter fits for different $\zeta$ values.}
\label{tab:param}
\begin{tabular}{llll}
\toprule
$\zeta$     & $k$        & $c$     & $d$      \\
\midrule
0.55        & 0.0167     & 2.717   & 1.225    \\
0.7         & 0.0162     & 3.412   & 1.225    \\
0.85        & 0.0159     & 4.073   & 1.225    \\
1           & 0.0157     & 4.714   & 1.225    \\
\bottomrule
\end{tabular}
\end{table}


\FloatBarrier
\section{Conclusion}

Molecular dynamics simulations are used in this study to provide a re-solution
rate of Xe gas bubbles in U-10Mo fuel. Both homogeneous and heterogeneous
re-solution have been simulated using the primary knock-on atom and thermal
spike methods. Homogeneous re-solution is found to be negligible in describing
the overall re-solution behavior. The fraction of re-solved Xe gas bubble atoms
seems to be invariant with respect to bubble pressure and Xe density. This
fraction is then used to model the re-solution fraction as a function of
effective imparted energy into the lattice for different bubble sizes. Both
on-centered and off-centered thermal spikes are simulated to get the spatial
correlation between re-solution and thermal spike distance. Also, the
electronic stopping power of fission products (Xe and Sr) in U-10Mo is
simulated to provide distance-dependent electronic stopping power.
Subsequently, all this information is used to calculate the overall re-solution
rate of Xe gas bubbles in the fuel. The re-solution rate of Xe gas bubble atoms
in U-10Mo is found to be on the order of $10^{-4}$ to $10^{-2}$ s$^{-1}$ for a
fission rate of $10^{14}$ fissions/cm$^3$/s. An analytical form of re-solution
rate dependent on bubble radius, fission rate, and $\zeta$ is also provided.
The result from this work will advise higher length scale models of U-10Mo with
a physics-based description of Xe gas bubble re-solution rate.


\bibliographystyle{unsrt}
\bibliography{ref.bib}

\end{document}

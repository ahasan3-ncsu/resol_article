\documentclass{article}
\usepackage[margin=1in]{geometry}
\usepackage{authblk}
\usepackage{graphicx, xcolor, placeins}
\usepackage{amsmath, booktabs, subcaption}

\title{Re-solution of Xe gas bubbles in the U-10Mo fuel}
\author{ATM Jahid Hasan, Benjamin Beeler}
\affil{Department of Nuclear Engineering, North Carolina State University,
Raleigh, NC}

\begin{document}
\maketitle


\begin{abstract}
	An U-Mo alloy based monolithic fuel design has been identified as the fuel
	type for conversion of the United States High-Performance Research Reactors
	(HPRRs). However, the U-Mo monolithic fuel exhibits excessive swelling
	during operation. To understand the fuel evolution under irradiation,
	mesoscale and engineering level fuel performance models require the
	knowledge of fundamental mechanistic behavior of fission products in the
	fuel. As such, understanding the progression of Xe gas bubbles in the fuel
	is crucial. The Xe gas bubbles act as a sink for individual Xe atoms
	(trapping of Xe) and subsequently grow in size after absorption. Under
	irradiation, the Xe atoms in the gas bubble are re-solved back into the
	fuel matrix, and this process reduces the bubble size. In this work, we
	investigate the re-solution of xenon gas bubble in U-Mo fuel with the
	molecular dynamics simulations. The efficiency of both the homogeneous and
	heterogeneous resolution mechanism are investigated via the primary knock
	on atom and thermal spike methods, respectively. For low energy
	interactions, the collision cascades caused by a single fast moving primary
	knock on atom were studied. The higher energy range was studied by the
	thermal spike method. The study sheds light on the atomic scale processes
	at play for fission gas re-solution in nuclear fuel. The present atomic
	scale simulations indicate that low energy interactions are not effective
	in the resolution process. The high energy interactions destroy smaller
	bubbles completely. The re-solution rate computed in this work, combined
	with the trapping rate of Xe, provides a full picture of the evolution of
	Xe gas bubbles in U-Mo.
\end{abstract}


\section{Introduction}

To understand the fuel evolution under irradiation, mesoscale and engineering
level fuel performance models require the knowledge of fundamental mechanistic
behavior of fission products in the fuel. As such, understanding the
progression of Xe gas bubbles in the fuel is crucial. The Xe gas bubbles act as
a sink for individual Xe atoms (trapping of Xe) and subsequently grow in size
after absorption. Under irradiation, the Xe atoms in the gas bubble are
re-solved back into the fuel matrix, and this process reduces the bubble size
\cite{ye2023}. In this work, we investigate the re-solution of xenon gas bubble
in U-Mo fuel with the molecular dynamics simulations. The efficiency of both
the homogeneous and heterogeneous resolution mechanism are investigated via the
primary knock on atom and thermal spike methods, respectively.


\FloatBarrier
\section{Computational details}

A molecular dynamics (MD) study is being conducted using the LAMMPS software
package \cite{lammps} to quantify the Xe re-solution rate in U-Mo. This rate
will be used in engineering-scale models to evaluate fuel evolution. Two models
of re-solution are implemented: primary knock-on atom (PKA) and thermal spike.
Re-solution through PKAs happens via atomic collision cascades (i.e. due to the
nuclear stopping). Re-solution through thermal spike occurs due to the
electronic stopping, which raises the temperature of the fission track to a
value that is higher than the melting temperature of the material. A U-Mo-Xe
angular-dependent potential (ADP) \cite{beelerADP} for the MD simulations. This
potential is capable of describing the body-centered cubic phase of U-Mo alloys
and is able to reproduce the stable structure, modulus of elasticity, room
temperature density, and melting point. Since this work focuses on the high
energy interactions such as primary knock on atom (PKA) and thermal spike
events, the distance between the ions and atoms can become very small, these
interactions are not described accurately by the classical interatomic
potential alone. Therefore, the repulsive potential of Ziegler, Biersack and
Littmark (ZBL) \cite{ziegler1985} a realistic description of the ion-ion
interaction at a very short order is chosen to act at very short inter-atomic
distances for all the potential used.

A 160 × 160 × 160 \r{A}$^3$ supercell is used for the PKA model. This amounts
to $\sim$ 8 million atoms in the system. Figure \ref{fig:struct} shows the
cross-section of such a supercell. The system is equilibrated at 400 K at a
pressure of 0 bar within the NPT ensemble using the Nose-Hoover barostat and
thermostat. Using the resulting configurations, spherical bubbles of radius 2
nm are created by removing a stoichiometric number of uranium and molybdenum
atoms and incorporating gaseous Xe atoms at a given density within in the
resulting void. Using this method, both the over and under pressurized bubbles
are created. For the 2 nm radius bubble, 200 and 400 Xe atoms are investigated.
The simulations are then equilibrated in the NPT ensemble. Finally,
displacement cascades are performed using a constant volume simulation cell
with a Nose-Hover thermostat at the edges of the simulation box. This allows
the resulting shock-wave emission and local heating to be contained with the
cell. This also prevents interactions with the periodic image in the adjacent
simulation cell. The cascade is initiated by selecting a single PKA, located
approximately 5.5 nm from the center of bubble and scaling its velocity to a
kinetic energy of successively higher energies (up to 500 keV). The simulations
are carried out for 120 picoseconds. The Xe atoms within the bubble are
carefully monitored to see whether any gas resolution occurs or not. Xe atoms
that traverse more than 1 nm from the surface of the initial spherical region
are counted as re-solved Xe atoms. The simulation is repeated for each bubble
density and PKA energy using varying initial PKA direction in order to gather
the statistics. Snapshots of such a PKA simulation are provided in Figure
\ref{fig:pka}.

Thermal spike method is a powerful model used to describe the interaction
caused by the swift heavy ions on a target material. These interactions occur
mainly through electronic excitation of the material through a four step
process. During the timescale of $10^{-17}-10^{-16}$ s, the swift heavy ions
interact with the electrons of the target materials and the deposited energy is
accumulated in the electronic subsystem \cite{toulemonde2002, wang1994}.
Secondly, this energy is redistributed locally in the electronic subsystem in
about $10^{-15}-10^{-14}$ s via electron-electron collision thus, heating up
the cold electrons. At around $10^{-13}-10^{-12}$ s the energy from the
electrons are transferred to lattice via the electron-phonon coupling. Finally,
the energy is transferred among the atoms resulting in a rapid rise of lattice
temperature ($\approx$10$^{4}$ K) in a cylindrical zone of typically few nm
radius called the thermal spike \cite{patra2019}. For the thermal spike model,
the supercell is 160 × 160 × 80 \r{A}$^3$ , which holds $\sim$ 4 million atoms.
The atoms present in a cylindrical zone of 2 nm radius along the thermal spike
axis are excited with a high kinetic energy to simulate local heating due to
electronic effects. A xenon bubble has been inserted with its center on the
thermal spike axis using a similar procedure as that used for the PKA
simulations. A total energy of up to 40 MeV is imparted to a cylindrical region
containing the gas bubble. Figure \ref{fig:spike} provides a few snapshots of a
thermal spike simulation.

\begin{figure}[ht]
	\centering
	\includegraphics[height=7cm]{images/struct.png}
	\caption{ Initial configuration for an MD simulation of the thermal spike.
	Only half of the supercell is shown.}
	\label{fig:struct}
\end{figure}

\begin{figure}[ht]
	\begin{subfigure}{0.49\textwidth}
		\centering
		\includegraphics[width=\textwidth]{images/pk1.png}
	\end{subfigure}
	\begin{subfigure}{0.49\textwidth}
		\centering
		\includegraphics[width=\textwidth]{images/pk2.png}
	\end{subfigure}

	\begin{subfigure}{0.49\textwidth}
		\centering
		\includegraphics[width=\textwidth]{images/pk3.png}
	\end{subfigure}
	\begin{subfigure}{0.49\textwidth}
		\centering
		\includegraphics[width=\textwidth]{images/pk4.png}
	\end{subfigure}

	\caption{Snapshots of a PKA (500 keV) simulation at different times. Xe
	atoms are shown in yellow along with U (red) and Mo (blue) point defects.
	(Chronological order is left to right, top to bottom.)}
	\label{fig:pka}
\end{figure}

\begin{figure}[ht]
	\begin{subfigure}{0.49\textwidth}
		\centering
		\includegraphics[width=\textwidth]{images/ev1.png}
	\end{subfigure}
	\begin{subfigure}{0.49\textwidth}
		\centering
		\includegraphics[width=\textwidth]{images/ev2.png}
	\end{subfigure}

	\begin{subfigure}{0.49\textwidth}
		\centering
		\includegraphics[width=\textwidth]{images/ev3.png}
	\end{subfigure}
	\begin{subfigure}{0.49\textwidth}
		\centering
		\includegraphics[width=\textwidth]{images/ev4.png}
	\end{subfigure}

	\caption{Snapshots of a thermal spike (3.5 MeV) simulation at different
	times. Xe atoms are shown in yellow along with U (red) and Mo (blue) point
	defects. (Chronological order is left to right, top to bottom.)}
	\label{fig:spike}
\end{figure}


\FloatBarrier
\section{Results and Discussion}

There is barely any re-solution in the PKA model. Only a few simulations showed
1 re-solved Xe atom in the system. Snapshots of one such simulation are shown
in Figure \ref{fig:pka}. On the other hand, the thermal spike model showed
significant re-solution of Xe atoms. Thus, we only consider heterogeneous
re-solution of Xe atoms from thermal spikes in our model. The re-solution in
this model is dependent on the total amount of effective energy deposited to
the lattice ($S_{e,eff}$). The re-solution data for both the under-pressurized
(200 Xe atoms) and the over-pressurized (400 Xe atoms) bubble are analyzed.
Figure \ref{fig:resol}(a) shows the fraction of re-solved Xe atoms in the
system as a function of deposited energy. From observing the data, it can be
hypothesized that the fraction of re-solved atoms is independent of the bubble
pressurization (or density). An exponentially saturating function of the
following form is used to model the available data.

\begin{align}
	\chi_0 &= 1 - \exp[-\alpha (S_{e,eff} - S_{e,c})]
\end{align}
where $\alpha$ is a tuning parameter and $S_{e,c}$ is the cutoff energy
deposition needed for any re-solution to happen. Setyawan et al.
\cite{setyawan2018} employed a similar functional model for their study of Xe
re-solution in UO$_2$. (The re-solution model developed in this work is heavily
inspired by their work as well.) After fitting this equation to the data, we
found the parameter values to be as follows. The model fit to the data is shown
in Figure \ref{fig:resol}(b).

\begin{align}
	\alpha &= 5.88 \times 10^{-4} \\
	S_{e,c} &= 2.47 \times 10^{-13}
\end{align}

\begin{figure}[ht]
	\begin{subfigure}{\textwidth}
		\centering
		\caption{}
		\includegraphics[width=10cm]{images/resol_frac.png}
	\end{subfigure}

	\begin{subfigure}{\textwidth}
		\centering
		\caption{}
		\includegraphics[width=10cm]{images/resol_curve.png}
	\end{subfigure}

	\caption{Fraction of re-solved Xe atoms as a function of the energy
	deposited to the lattice.}
	\label{fig:resol}
\end{figure}

In this work, only on-centered thermal spikes are simulated so far.
Off-centered thermal spikes are expected to have diminishing effects to the
re-solution as they go farther away from the Xe gas bubble. Let us denote the
fraction of re-solved Xe atoms due to any thermal spike as $\chi$, the distance
between the bubble center and the thermal spike center as $r$, the distance
between the bubble center and origin of the thermal spike as $x$ , and the
fission rate as $\dot{F}$. The heterogeneous re-solution rate can then be
expressed as a volume integral.

\begin{align}
	b_{het}
	&= \sum_{i=1}^2 \int_{r=0}^{\infty} \int_{x=0}^{\infty}
		\chi_i \dot{F} 2 \pi r dr dx \\
	&= \dot{F} \sum_{i=1}^2 \int_{r=0}^{\infty}
		\bigg( \frac{\chi_i}{\chi_{0,i}} \bigg)
		2 \pi r dr \int_{x=0}^{\infty} \chi_{0,i} dx
\end{align}

In the absence of off-centered thermal spike data, we use the expression
$\int_{r=0}^{\infty} \bigg( \frac{\chi_i}{\chi_{0,i}} \bigg) 2 \pi r dr \approx
0.25 \pi r_c^2$ as calculated in \cite{setyawan2018}. Here, $r_c = R_{bubble} +
R_{spike}$ is the cutoff distance for thermal spikes. Re-solution of Xe atoms
are assumed to be improbable for $r \geq r_c$.

\begin{align}
	b_{het}
	&= 0.25 \pi r_c^2 \dot{F} \sum_{i=1}^2 \int_{x=0}^{\infty}
		[1 - \exp(-\alpha (S_{e,eff,i} - S_{e,c}))] dx \\
	&= 0.25 \pi r_c^2 \dot{F} \sum_i y_i \int_{x=0}^{\infty}
		[1 - \exp(-\alpha (\zeta S_{e,i} - S_{e,c}))] dx \label{eq:dx}
\end{align}
where $y_i$ is the independent fission product yield of the fission product $i$
and $\zeta = S_{e,eff} / S_{e}$, the fraction of fission product energy that is
imparted to the lattice. In this work, $\zeta$ is assumed to be 0.73 as per
\cite{govers2012, setyawan2018}. Therefore, $S_e$ for all possible fission
products are needed to be known for the evaulation of the integral in the above
equation.

We choose to evaluate the $S_e$ values for Xe-140 and Sr-94 in U-10Mo based on
the fission reaction $_0^1n + _{92}^{235}U \rightarrow _{54}^{140}Xe +
_{38}^{94}Sr + Q$. Even though it is one of many possible fission reactions
that may take place, Xe-140 and Sr-94 can still be considered as
representatives of heavy and light fission products. The rationale behind this
is that the two aforementioned species appear close to the two peaks of a
typical fission product yield graphs.

$S_{e,Xe}$ and $S_{e,Sr}$ as a function of the distance traversed by the Xe
atom is shown in Figure \ref{fig:irad}. This data is generated using the
Iradina software \cite{crocombette2019irad}. The following model is fit to
represent the $S_e$ data in the integral in equation \ref{eq:dx}.

\begin{align}
	S_{e,Sr} &= 19.7 \exp(-0.00273 x^{3.71})
		+ 6.8 \exp(-0.424 x^{1.45}) \\
	S_{e,Xe} &= 21.3 \exp(-0.239 x^{1.78})
		+ 5.23 \exp(-4.67 \times 10^{-8} x^{11})
\end{align}

\begin{figure}[ht]
	\centering
	\includegraphics[width=10cm]{images/iradina.png}
	\caption{Total electronic stopping power ($S_e$) of Xe in U-10Mo calculated
	with the SRIM code as a function of distance traveled by Xe from the
	location of the fission reaction.}
	\label{fig:irad}
\end{figure}

Finally, after evaluating the integral, we get the following expression for the
re-solution rate of Xe gas bubbles in U-10Mo as a function of fission rate
(fissions/cm$^3$/s).

\begin{align}
	b_{het} = 9 \times 10^{-19} \dot{F}
\end{align}

To compare the re-solution rate calculated in this work with the rate defined by
the model used in DART \cite{ye2023}, the associated model is described below.

\begin{align}
	b_{dart} &= b_0 \cdot \dot{F} \cdot G \\
	b_0 &= R_{spike}^2 \cdot \mu_{ff} \\
	G &=
	\begin{cases}
		1 & ,R_{bubble} \leq \lambda \\
		1 - (\frac{R_{bubble}-R_{resol}}{R_{bubble}})^3 & ,R_{bubble} > \lambda
	\end{cases}
\end{align}
where $b_0$ is the bubble destruction probability, $\dot{F}$ is the fission
rate, and $G$ is a piecewise function representing different resolution modes
for small and large gas bubbles. $b_0$ can be estimated with the interaction
volume of a thermal spike with bubbles $b_o = R_{spike}^2 \times \mu_{ff}$:
where $R_{spike}$ is the radius of a thermal spike and $\mu_{ff}$ is the recoil
length of fission fragments. In the piecewise function G, $R_{bubble}$ is the
bubble radius, $\lambda$ is the gas-atoms knock out distance, and $R_{resol}$
is the thickness of the annulus within which all gas-atoms are knocked out.
Ye et al. reported the optimized values of $b_0, \lambda, R_{resol}$ in
\cite{ye2023}. These optimized values are used to get the following concise
expression for a bubble radius of 2 nm.

\begin{align}
	b_{dart} = 2 \times 10^{-18} \dot{F}
\end{align}

The re-solution rates from the model developed in this work and the model
described in \cite{ye2023} are presented in Figure \ref{fig:comp} for a broad
range of plausible fission rates. The re-solution rate used in DART is 2.2
times larger than the value calculated in this work. However, the re-solution
rate described in this work is only a lower limit since only Xe (a lower energy
fission product) is used as a possible fission product for calculations.
Overall, the result from this work agrees on the order of magnitude of
re-solution rate used in DART.

\begin{figure}[ht]
	\centering
	\includegraphics[width=10cm]{images/comparison.png}
	\caption{Comparison of re-solution rate calculated in this work with the
	rate used in the DART model \cite{ye2023}.}
	\label{fig:comp}
\end{figure}


\FloatBarrier
\section{Conclusion}

Molecular dynamics simulation is used in this study to provide a re-solution
rate of Xe gas bubbles in U-10Mo fuel. Both homogeneous and heterogeneous
re-solution has been simulated using primary knock-on atom and thermal spike
methods. Homogeneous re-solution is found to be negligible in describing the
overall re-solution behavior. The fraction of re-solved Xe gas bubble atoms
seems to an invariant with respect to bubble pressure and density. This
fraction is then used to model re-solution fraction as a function of effective
imparted energy into the lattice. Afterward, electronic stopping power of Xe
fission products in U-10Mo is used to provide a re-solution rate of Xe gas
bubbles. The re-solution rate of Xe gas bubble atoms in U-10Mo is found to be
$b = 9 \times 10^{-19} \dot{F}$, where $\dot{F}$ is the fission rate specified
as fissions/cm$^3$/s.


\bibliographystyle{unsrt}
\bibliography{ref.bib}

\end{document}

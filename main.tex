\documentclass[12pt]{article}
\usepackage[margin=1in]{geometry}
\usepackage{graphicx, subcaption, booktabs}
\usepackage{amsmath, amssymb, siunitx}
\usepackage{placeins, hyperref}

\title{Xe gas bubble re-solution in U-10Mo nuclear fuel}
\author{ATM Jahid Hasan}

\begin{document}

\maketitle

\begin{abstract}
	Nothing here yet
\end{abstract}


\section{Introduction}

Nothing here yet


\section{Computational methods}

MD: LAMMPS, ADP

BCA: RustBCA, K-C


\section{Energy loss of fission fragments in U-10Mo}

Representative fission reaction:
\begin{align}
_0^1n + _{92}^{235}U
\rightarrow
_{39}^{97}Y + _{53}^{136}I + Q
\end{align}

Y and I have a fission product yield of about 0.12 (cite here).
Nuclear and electronic stopping powers of these fission fragments in U-10Mo
are shown in Figure \ref{fig:stopping}.

\begin{figure}[!ht]
\begin{subfigure}{0.49\textwidth}
	\centering
	\includegraphics[width=8cm]{images/Y_stopping.pdf}
\end{subfigure}
\begin{subfigure}{0.49\textwidth}
	\centering
	\includegraphics[width=8cm]{images/I_stopping.pdf}
\end{subfigure}
\caption{
	Nuclear and electronic stopping power
	of (a) $_{39}^{97}Y$ and (b) $_{53}^{136}I$.
}
\label{fig:stopping}
\end{figure}


\FloatBarrier
\section{Re-solution due to electronic stopping}

MD simulations incorporating TTM have been performed
to quantify re-solution due to electonic stopping.
No re-solution has been observed even at 30 keV/nm.
This is consistent with the findings of Kolotova et al.


\section{Re-solution due to nuclear stopping}

\subsection{Model for re-solution calculation}

Rotated fission events \ref{fig:rot}

\begin{figure}[ht]
	\centering
	\includegraphics[width=14cm]{images/rot.pdf}
	\caption{
		Rotation of fission event positions and velocities around the origin
		such that all velocities point to the $-x$ direction.
	}
	\label{fig:rot}
\end{figure}

Volume element \ref{fig:volel}

\begin{figure}[ht]
	\centering
	\includegraphics[width=6cm]{oldimg/coordSystem.pdf}
	\caption{
		placeholder for new illustration
	}
	\label{fig:volel}
\end{figure}

\begin{align}
	dV &= 2 \pi w dw dx \\
	b &= \sum_{k = Y, I} \int_V \xi_k \dot{F} dV \\
	b &= \dot{F} \sum_{k = Y, I} \int_V \xi_k dV
		= \dot{F} \left( \int_V \xi_Y dV + \int_V \xi_I dV \right) \\
	b &= \dot{F} \sum_{k = Y, I} \int_x \int_w \xi_k(x, w) 2 \pi w dw dx
\end{align}

\subsection{Master simulations}

% simulation visuals using vispy here

Master simulations \ref{fig:master}

% the colorbars need labels
\begin{figure}[ht]
	\centering
	\begin{subfigure}{0.49\textwidth}
		\centering
		\caption{}
		\includegraphics[width=8cm]{images/Y_p.pdf}
	\end{subfigure}
	\begin{subfigure}{0.49\textwidth}
		\centering
		\caption{}
		\includegraphics[width=8cm]{images/I_p.pdf}
	\end{subfigure}
	\begin{subfigure}{0.49\textwidth}
		\centering
		\caption{}
		\includegraphics[width=8cm]{images/Y_e.pdf}
	\end{subfigure}
	\begin{subfigure}{0.49\textwidth}
		\centering
		\caption{}
		\includegraphics[width=8cm]{images/I_e.pdf}
	\end{subfigure}
	\begin{subfigure}{0.49\textwidth}
		\centering
		\caption{}
		\includegraphics[width=8cm]{images/Y_a.pdf}
	\end{subfigure}
	\begin{subfigure}{0.49\textwidth}
		\centering
		\caption{}
		\includegraphics[width=8cm]{images/I_a.pdf}
	\end{subfigure}
	\caption{
		Ion incidence probability per surface area of (a) Y and (b) I.
		Average incidence energy of (c) Y and (d) I.
		Average incidence angle of (e) Y and (f) I.
	}
	\label{fig:master}
\end{figure}

\FloatBarrier
\subsection{Fission fragment interactions with Xe gas bubbles}

van der Waals EOS \ref{fig:vdw}

\begin{align}
	n &= \bigg( B + \frac{kT}{p} \bigg)^{-1} \\
	p_{eq} &= \frac{2 \gamma}{R_b} \\
	n_{eq} &= \bigg( B + \frac{kT R_b}{2 \gamma} \bigg)^{-1}
\end{align}

\begin{figure}[ht]
	\centering
	\includegraphics[width=8cm]{images/n_vdw.pdf}
	\caption{
		Equilibrium Xe number density in gas bubbles
		calculated from the van der Waals equation of state.
	}
	\label{fig:vdw}
\end{figure}

\begin{align}
	D &= R_b + \delta \\
	y' &= \mathcal{I}_X (x', X, Y) \\
	\chi(E', \ell)
	   &= \mathcal{I}_{\mathcal{E}}
	   (E', \mathcal{E}, [\chi(E, \ell)]_{E \in \mathcal{E}}) \\
	\chi(E', \ell')
	   &= \mathcal{I}_{\mathcal{L}}
	   (\ell', \mathcal{L}, [\chi(E', \ell)]_{\ell \in \mathcal{L}})
\end{align}

% simulation visuals here using vispy

$\chi$ data \ref{fig:chi}

% use scientific notational for tick labels and make them consistent
\begin{figure}[ht]
	\centering
	\begin{subfigure}{0.49\textwidth}
		\centering
		\caption{}
		\includegraphics[width=8cm]{images/chi_2nm_Y.pdf}
	\end{subfigure}
	\begin{subfigure}{0.49\textwidth}
		\centering
		\caption{}
		\includegraphics[width=8cm]{images/chi_2nm_I.pdf}
	\end{subfigure}
	\begin{subfigure}{0.49\textwidth}
		\centering
		\caption{}
		\includegraphics[width=8cm]{images/chi_64nm_Y.pdf}
	\end{subfigure}
	\begin{subfigure}{0.49\textwidth}
		\centering
		\caption{}
		\includegraphics[width=8cm]{images/chi_64nm_I.pdf}
	\end{subfigure}
	\caption{
		$\chi(E, \ell)$ for bubble radius of 2 nm with incident (a) Y and (b) I.
		$\chi(E, \ell)$ for bubble radius of 64 nm with incident (c) Y and (d) I.
	}
	\label{fig:chi}
\end{figure}

\FloatBarrier
\subsection{Calculation of \texorpdfstring{$\xi$}{xi}}

\begin{align}
	\xi(x, w) &= \sum_{m \in S}
		p(r_m) \frac{A_m}{\cos \alpha(x, w)}
		\chi(E(r_m), ||r_m - r_c||) \\
	b / \dot{F} &= \sum \xi \Delta V \\
		&= \sum_{i,j \in V} \xi(c_{i,j})
		\pi (w_{i,j+1}^2 - w_{i,j}^2) (x_{i+1,j} - x_{i,j})
\end{align}

\subsection{Homogeneous re-solution rate}

Re-solution \ref{fig:res}

\begin{figure}[ht]
	\centering
	\includegraphics[width=8cm]{images/bhom.pdf}
	\caption{
		Homogeneous re-solution rate as a function of bubble radius $R_b$
		at equilibrium Xe number density $n_{eq}$ in U-10Mo.
	}
	\label{fig:res}
\end{figure}

\subsection{Effect of bubble pressure}

Pressure effects \ref{fig:pres}

\begin{figure}[ht]
	\centering
	\includegraphics[width=8cm]{images/pressure.pdf}
	\caption{
		Effect of Xe number density on homogeneous re-solution rate.
	}
	\label{fig:pres}
\end{figure}

\subsection{Re-solution rate as a function of bubble size and pressure}


\section{Discussion}

Comparison with DART \ref{fig:comp}

% also need to compare with uo2, uc and uzr

\begin{figure}[ht]
	\centering
	\includegraphics[width=8cm]{images/comp.pdf}
	\caption{
		Comparison of the DART model prediction for re-solution
		with the results from this work.
	}
	\label{fig:comp}
\end{figure}


\FloatBarrier
\section{Conclusion}


\end{document}
